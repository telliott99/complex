\documentclass[11pt, oneside]{article} 
\usepackage{geometry}
\geometry{letterpaper} 
\usepackage{graphicx}
	
\usepackage{amssymb}
\usepackage{amsmath}
\usepackage{parskip}
\usepackage{color}
\usepackage{hyperref}

\graphicspath{{/Users/telliott/Github/figures/}}

\title{Laurent series}
\date{}

\begin{document}
\maketitle
\Large 

\subsection*{example}
\url{https://math.berkeley.edu/~nikhil/courses/121a/laurent.pdf}

Here is another example where we know the Laurent series:
\[ f(z) = z e^{1/z} \]
Use the known series expansion for the exponential to write:
\[ = z \ [ \ 1 + (1/z) + \frac{(1/z)^2}{2!} + \dots \ ] \]
\[ = z + 1 + \frac{(1/z)}{2!} + \dots \ ] \]

So the integral around 
\[ \oint f(z) \ dz = \oint \frac{(1/z)}{2!} \]
Our formulas are:
\[ b_1 = \lim_{z \rightarrow z_0} (z-z_0) \ f(z)  \]
\[ \oint f(z) \ dz = 2 \pi i \ \sum \text{ Res } \]

so the residue is
\[ b_1 = \lim_{z \rightarrow 0} (z) \ \frac{(1/z)}{2!} = \frac{1}{2} \]
and
\[ \oint f(z) \ dz = 2 \pi i \ \frac{1}{2} = \pi i \]

We don't need to actually know the Laurent series, only that it exists.  The form is
\[ f(z) = \sum_{n=0}^{\infty} a_n(z-z_0)^n + \frac{b_1}{z-z_0} + \dots + \frac{b_m}{(z-z_0)^m} \]

Now there are three steps.  Multiply by $(z - z_0)^m$
\[ (z - z_0)^m f(z) = \sum_{n=0}^{\infty} a_n(z-z_0)^{n+m} + b_1 (z-z_0)^{m-1} + \dots + b_m \]

Second, lose the $b_2, \dots b_m$ terms by differentiating $m-1$ times.  (Note the exponents for terms beyond $b_1$ are \emph{smaller}, so they disappear...
\[ \frac{d^{m-1}}{dz^{m-1}} (z-z_0)^m f(z) = \sum_{n=0}^{\infty} a_n\frac{d^{m-1}}{dz^{m-1}}  (z-z_0)^{n+m} + (m-1)! \ b_1 \]

Kill off all the $a_n$ terms by taking the limit $z \rightarrow z_0$
\[ \lim_{z \rightarrow z_0}  \frac{d^{m-1}}{dz^{m-1}} (z-z_0)^m f(z) = (m-1)! \ b_1 \]
\[ b_1 = \frac{1}{(m-1)!} \ \lim_{z \rightarrow z_0}  \frac{d^{m-1}}{dz^{m-1}} (z-z_0)^m f(z) \]


\end{document}