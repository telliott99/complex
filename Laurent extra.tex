\documentclass[11pt, oneside]{article} 
\usepackage{geometry}
\geometry{letterpaper} 
\usepackage{graphicx}
	
\usepackage{amssymb}
\usepackage{amsmath}
\usepackage{parskip}
\usepackage{color}
\usepackage{hyperref}

\graphicspath{{/Users/telliott/Github/figures/}}

\title{Writing Laurent Series}
\date{}

\begin{document}
\maketitle
\Large

%[my-super-duper-separator]

\subsection*{examples 1 and 2}

\label{sec:ex1L}
\label{sec:ex2L}

Our first two examples are
\[ \int \frac{1}{1 - z} \ dz, \ \ \ \ \ \ \int \frac{1}{1 + z} \ dz \]  

Of course, we already know how to do these integrals.  

They are variants of $\int 1/(z - z_0) \ dz$.  The first one just needs a minus sign:
\[ - \int \frac{1}{z - 1} \ dz \]

So then, if the path encloses the singularity at $z = 1$, the value is $-2 \pi i f(z_0)$, where $f = 1$ (from the numerator).

For the second, we have that $z_0 = -1$ but the form is already correct.  If the path includes the singularity, the value is $2 \pi i$.

Alternatively, make a change of variables.  For the first, $w = 1 - z$, $dw = -dz$ and $\int -dw/w = -2 \pi i$ if the contour includes $w = 0$, that is, $z = 1$.

\subsection*{writing Laurent series}

We will now use these two as an entry point to writing Laurent series.  

Consider the first.  As we know, the geometric series is just

$1A$:
\[ \frac{1}{1 - z} = 1 + z + z^2 + z^3 \dots \]

and it converges for an open disk with $|z| < 1$.  We will refer to this below as $1A$ (for analytic).

There is a trick to getting a \emph{different} geometric series.  Write
\[ \frac{1}{1 - z} = -\frac{1}{z} \cdot \frac{1}{1 - 1/z} \]

That's also a geometric series, but in $1/z$.  It converges when $|1/z| < 1$ which means $|z| > 1$!  That's just what we need.

$1P$ (for principal) is
\[ -\frac{1}{z} \cdot (1 + \frac{1}{z} + \frac{1}{z^2} + \frac{1}{z^3} \dots ) = - \frac{1}{z} + \frac{1}{z^2} - \frac{1}{z^3} \dots \]

We can read the value of the integral as the coefficient $b_1 = -1$.

Obtain positive $z$ by substituting $w = -z$ or $z = -w$, or just substituting $z = -z$ everywhere it appears.  (Odd powers get a change of sign).

$2A$:
\[ \frac{1}{1 + z} = 1 - z + z^2 - z^3 \dots \]

$2P$:
\[ \frac{1}{1 + z} = - \frac{1}{z} + \frac{1}{z^2} - \frac{1}{z^3} \dots \]

Check by multiplying the right-hand side by $z$ and see all the cancellations after the first term.  Similarly, then the coefficient $b_1 = 1$.

We can also deal with 
\[ \frac{1}{a - z} = \frac{1}{a} \cdot \frac{1}{1 - z/a} \]

Then we have the analytic series as 

\[ \frac{1}{a - z} = \frac{1}{a} \cdot \ [ \ 1 +  \frac{z}{a} + (\frac{z}{a})^2 + \dots  \]
\[ \frac{1}{a - z} = \frac{1}{a} +  \frac{z}{a^2} +  \frac{z^2}{a^3} + \dots \]

and the principal part of the Laurent series as
\[ \frac{1}{a - z} = \frac{1}{a} \ [ - \frac{a}{z} + (\frac{a}{z})^2 - (\frac{a}{z})^3 \dots \ ]  \]
\[ \frac{1}{a - z} = - \frac{1}{z} + \frac{a}{z^2} - \frac{a^2}{z^3} + \frac{a^3}{z^4} \dots \ ]  \]

Then
\[ \frac{1}{z - a} = \frac{1}{z} - \frac{a}{z^2} + \frac{a^2}{z^3} - \frac{a^3}{z^4} \dots \ ]  \]

Laurent series is more complicated than these two examples.  The reason is that we must think about the disk or annulus of convergence, and make sure that the series we're using converges in that region.

We will use these results below.

In general, the Laurent series is two parts.  The analytic part is
\[ \sum a_k (z - z_0)^k \]
valid in a disk inside $r = 1$.

The principal part is
\[ \sum b_k (z - z_0)^{-k} \]
valid in the region outside a disk of radius $r = 1$.  If the variable is $a/z$ then the valid radius is $z > a$.

\subsection*{example 16 (Boas)}

\label{sec:ex16L}

\[ f(z) = \frac{12}{z(2 - z)(1 + z)} \]

This function has three isolated singularities (at $z = 0, 2, -1$).

Expanded around $z_0 = 0$, there will be three regions in which we have \emph{different} series:  namely $0 < |z| < 1$, $1 < |z| < 2$ and $|z| > 2$.

\begin{center} \includegraphics [scale=0.5] {Boas_14_4_1b.png} \end{center}

For each region, there will usually be two series adding up to the complete Laurent series.

Also, in this problem, we have two integrands, obtained by partial fractions.  So it's even more complicated!
\[ = \frac{4}{z} \cdot (\frac{1}{1 + z} + \frac{1}{2 - z}) \]

\subsection*{reference}
$1/1-z =$ (1A and 1P):
\[ = 1 + z + z^2 + z^3 \dots \]
\[ -\frac{1}{z} \cdot (1 + \frac{1}{z} + \frac{1}{z^2} + \frac{1}{z^3} \dots ) \]
$1/1+z =$ (2A and 2P):
\[ = 1 - z + z^2 - z^3 \dots \]
\[ \frac{1}{z} \cdot (1 - \frac{1}{z} + \frac{1}{z^2} - \frac{1}{z^3} \dots ) \]

\subsection*{inner disk}

Start with the inner punctured disk.  We want convergence for the region $|z| < 1$.  Since it's less than, we use standard geometric series for each of the two terms.

For $1/(1+z)$ we get just 2A:
\[ \frac{1}{1 + z} = 1 - z + z^2 - z^3 + z^4 \dots \]

For $1/(2 - z)$ we must first factor:
\[ \frac{1}{2 - z} = \frac{1}{2} \cdot \frac{1}{1 - z/2} \]

Recall $1A$:
\[ \frac{1}{1 - z} = 1 + z + z^2 + z^3 + z^4 \dots \]
substitute:
\[ = \frac{1}{2} \ [ \ 1 + \frac{z}{2} + \frac{z^2}{4} + \frac{z^3}{8} + \frac{z^4}{16} \dots \ ] \]

Multiply 
\[ = \frac{1}{2} + \frac{z}{4} + \frac{z^2}{8} + \frac{z^3}{16} + \frac{z^4}{32} \dots \ ] \]

add them together 
\[ = \frac{3}{2} - \frac{3z}{4} +  \frac{9z^2}{8} - \frac{15z^3}{16} + \frac{33z^4}{32}  \dots \]

and finally, multiply by $4/z$ to obtain:
\[ = \frac{6}{z} - 3 + \frac{9z}{2} - \frac{15z^2}{4} \dots \]

This is the Laurent series valid in the innermost region.

Alternatively, for the second, go back to
\[ \frac{1}{a - z} = \frac{1}{a} +  \frac{z}{a^2} +  \frac{z^2}{a^3} + \dots \]

\subsection*{outer region}

For the outer region, we need the principal part only.  The first one is good as it is.

$2P$:
\[ \frac{1}{z} \cdot (1 - \frac{1}{z} + \frac{1}{z^2} - \frac{1}{z^3} \dots ) \]

For the second, we have the rearranged version:
\[ \frac{1}{2 - z} = \frac{1}{2} \cdot \frac{1}{1 - z/2} \]

$1P$:
\[ -\frac{1}{z} \cdot (1 + \frac{1}{z} + \frac{1}{z^2} + \frac{1}{z^3} \dots ) \]

so we need to substitute $z/2$ for $z$ in $1P$:
\[ \frac{1}{2 - z} = \frac{1}{2} \cdot (-\frac{2}{z}) (1 + \frac{2}{z} + \frac{4}{z^2} + \frac{8}{z^3} + \frac{16}{z^4} \dots ) \]

multiply:
\[ = -\frac{1}{z} -\frac{2}{z^2} - \frac{4}{z^3} - \frac{8}{z^4} - \frac{16}{z^4} \dots ) \]

Add them together:
\[  - \frac{3}{z^2} - \frac{3}{z^3} - \frac{9}{z^4} \dots  ) \]

Recall the leading factor of $4/z$
\[ - \frac{12}{z^3} - \frac{12}{z^4} - \frac{36}{z^5} \dots )  \]
\[ = \frac{12}{z^3} \ (-1 -\frac{1}{z} - \frac{3}{z^2} \dots )\]

\subsection*{middle region}

The third part is the annulus in the middle.  For this we want convergence for $|z| > 1$ and for $|z| < 2$.  Hence we want the principal part for $1/(1+z)$ and the analytic part for $1/(2-z)$.

\[ \frac{1}{1 + z} =  \frac{1}{z} - \frac{1}{z^2} + \frac{1}{z^3} - \frac{1}{z^4} \dots \]

\[ \frac{1}{2 - z} = \frac{1}{2} + \frac{z}{4} + \frac{z^2}{8} + \frac{z^3}{16} + \frac{z^4}{32} \dots \]

add them together, which isn't necessary since there are no shared powers.

Finally, there is the leading factor of $4/z$.

I try to verify one power, of course I pick $z^{-1}$.  I get $6/z$ from the inner disk, and $-2/z$ from the middle, giving $4/z$.

Looking ahead to the residue theory, I try to verify the sum of residues:
\[ 12 \ [ \ \frac{1}{(1+z)(2-z)} \bigg |_{z=0} + \frac{1}{z(1+z)} \bigg |_{z=2} + \frac{1}{z(2-z)} \bigg |_{z=-1} \ ] \]
I get
\[ 12 \ (\frac{1}{2} + \frac{1}{6} - \frac{1}{3} ) = 4 \]

Even easier:
\[ f(z) = \frac{4}{z} \cdot \ [ \ \frac{1}{1 + z} + \frac{1}{2 - z} \ ]\]
I get
\[ 4 \ [ \ \frac{1}{z} \bigg |_{-1} + \frac{1}{z+1} \bigg |_{0} + \frac{1}{z} \bigg |_{2} + \frac{1}{2-z} \bigg |_{0} \ ] \]
\[ = 4(-1 + 1 + \frac{1}{2} + \frac{1}{2}) = 4 \]

Maybe we got that right!

(Boas has $2$ as the coefficient of the $1/z$ term in the series.  I think that's possibly an error).

\subsection*{example 18}

\label{sec:ex18L}

Consider:
\[ f(z) = \frac{z}{(z-1)(z-3)} \]

Suppose we want the series around $0 \le | z - 1 | \le 2$, also written as $C[1,2]$, a circle of radius $2$ around the point $z_0 = 1$.

I really prefer this notation
\[ \text{C[origin, radius]} \]
since there is no need to mentally change the sign on the second term of $z - z_0$ to get $z_0$.

One way to do this is to simplify by partial fractions.
\[ f(z) = \frac{z}{2} \ (\frac{1}{z - 3} - \frac{1}{z - 1}) \]

Another way is to make a substitution $w = z - 1$, so $z = w + 1$ and we have
\[  = \frac{w+1}{(w)(w-2)} \]
\[ = \frac{1}{w - 2} + \frac{1}{w(w-2)}  \]
(this is easy to restore at the end by multiplying by $1/w$).

Since
\[ \frac{1}{z - a} = \frac{1}{z} - \frac{a}{z^2} + \frac{a^2}{z^3} - \frac{a^3}{z^4} \dots \ ]  \]
The first term is $1/w$ and the second is
\[ \frac{1}{w} 

\subsection*{xxx}

And then our goal is to get something like $1/1-r$.  

That means we want $w - 2$ on top
\[ = \frac{1}{w} \ ( \frac{w - 2 +  3}{w - 2} ) \]
\[ = \frac{1}{w} \ ( 1 + \frac{ 3}{w-2} ) \]
\[ = \frac{1}{w} \ ( 1 - \frac{ 3}{2-w} ) \]
\[ = \frac{1}{w} \ ( 1 - \frac{ 3/2}{1 - w/2} ) \]
\[ = \frac{1}{w} \ ( 1 - \frac{3}{2} \cdot \frac{1}{1 - w/2} ) \]

and now the 
\[ \frac{1}{1 - w/2} \]
can be expanded because that's a geometric series $\sum (w/2)^n$

so 
\[ \frac{1}{1 - w/2} = 1 + \frac{w}{2} + (\frac{w}{2} )^2 + \dots \]
which gives
\[ = \frac{1}{w} \ [ 1 - \frac{3}{2} \cdot (1 + \frac{w}{2} + (\frac{w}{2} )^2 + \dots) \ ] \]

Now, multiplying through by $1/w$ gives
\[ -\frac{1}{2w} + \dots \]
and \emph{nothing else matters}.  Reverse the change of variable:
\[ = -\frac{1}{2(z-1)} + \dots \]

which we will integrate as
\[ \oint \frac{-1/2}{z-1} \ dz \]
 over $C[1,2]$.
 
We can look ahead to the formula for residues:
\[ b_1 = \lim_{z \rightarrow z_0} (z-z_0) \ f(z)  \]

So
\[ \text{Res }(1) = \lim_{z \rightarrow 1} (z-1) \ \frac{-1/2}{z-1} = -\frac{1}{2} \]
Multiply by $2 \pi i$ to obtain $I = -\pi i$.

As a check on this go back to 
\[ f(z) = \frac{z}{(z-1)(z-3)} \]
\[ \text{Res }(1) = \lim_{z \rightarrow 1} (z-1) \ \frac{z}{(z-1)(z-3)} \]
\[ =  \lim_{z \rightarrow 1} \ \frac{z}{z-3}  = -\frac{1}{2} \]
and we just bypassed the manipulation.

\subsection*{example 17}
These examples can get complicated.  Here is one from

\url{http://zimmer.csufresno.edu/~doreendl/128.13f/handouts/Lseriesex.pdf}

\label{sec:ex17L}

\[ f(z) = \frac{1}{(z-2)(z-1)} \]

This function has poles at $z = 1$ and $z = 2$.  If we are asked to write expansions around $z_0 = 0$, then we have three regions of interest and three different expansions.

The first region is the circle of radius $1$:  $|z| < 1$, the second is $1 < |z| < 2$ and then finally $|z| > 2$.

\subsection*{region 1}
Use partial fractions to write:
\[ \frac{1}{(z-2)(z-1)} = \frac{1}{z-2} - \frac{1}{z-1} \]
Considering the second term, we bring the minus sign inside
\[ = \frac{1}{z-2} + \frac{1}{1-z} \]
We have a geometric series for the second term, and for the other one
\[ \frac{1}{z - 2} = - \frac{1}{2 - z} \]

Our goal is to convert this into something like the geometric series.  Factor out the $2$ on the bottom like so
\[ = - \frac{1}{2} \ [ \  \frac{1}{1 - z/2} \ ]  \]

We can do a formal substitution or recognize that this is the geometric series 
\[ = - \frac{1}{2} \ [ \   \sum_{n=0}^{\infty} (z/2)^n \ ]  \]

We can rewrite this slightly by combining $2^n$ on the bottom with the $2$ out front:
\[ = \sum_{n=0}^{\infty} \ [ \ \frac{-1}{2^{n+1}} \ ] \  z^n \  \]
which converges for $0 < |z/2| < 1 \Rightarrow 0 < |z| < 2$.

Our series is the sum of these two series, which can be combined as
\[ \sum_{n=0}^{\infty} \ [ \ 1 - \frac{1}{2^{n+1}} \ ] \  z^n \  \]

\subsection*{region 2}
This is the annulus $1 < |z| < 2$.  Thus
\[ | \frac{1}{z} | < 1 \  \ \ \text{ and } \ \ \  |\frac{z}{2} | < 1 \]

What they do is to work on the right-hand term of
\[ \frac{1}{(z-2)(z-1)} = \frac{1}{z-2} - \frac{1}{z-1} \]
and, as we saw in the previous section transform it into something containing $1/z$, which will be valid in the region $|z| > 1$.

So let's do it:
\[ \frac{1}{1 - z} = - \frac{1}{z - 1}  \]
\[ = - \frac{1}{z} \cdot \frac{1}{1 - 1/z}  \]
leaving aside the leading factor this is
\[ = 1 + \frac{1}{z} + \frac{1}{z^2} \dots \]
\[ = \sum_{n=0}^{\infty} \frac{1}{z^n} \]
add back that factor
\[ -\frac{1}{z} \ \sum_{n=0}^{\infty} \frac{1}{z^n} \]

The left-hand term is exactly what we had before:
\[ \sum_{n=0}^{\infty} \ [ \ \frac{-1}{2^{n+1}} \ ] \  z^n \  \]
so we combine them
\[ \sum_{n=0}^{\infty} \ [ \ \frac{-1}{2^{n+1}} \ ] \  z^n -\frac{1}{z} \ \sum_{n=0}^{\infty} \frac{1}{z^n} \]
and then just bring that $z$ in the second term inside
\[ \sum_{n=0}^{\infty} \ [ \ \frac{-1}{2^{n+1}} \ ] \  z^n - \ \sum_{n=0}^{\infty} \frac{1}{z^{n+1}} \]
or change the index
\[ = \sum_{n=0}^{\infty} \ [ \ \frac{-1}{2^{n+1}} \ ] \  z^n - \ \sum_{n=1}^{\infty} \frac{1}{z^{n}} \]

\subsection*{region 3}
We do the $1/z$ trick with both terms
\[ \frac{1}{z-2} - \frac{1}{z-1} \]
Start with the first one:
\[ \frac{1}{z-2} = \frac{1}{z} \cdot \frac{1}{1 - 2/z} \]
The series is
\[ \frac{1}{z} \ \cdot \ \sum_{n=0}^{\infty} \ [ \ \frac{2}{z} \ ]^n \]
\[ = \sum_{n=0}^{\infty} \ \frac{2^n}{z^{n+1}} \]

The second term is (leaving off the factor of $-1$)
\[ \frac{1}{z-1} = \frac{1}{z} \cdot \frac{1}{1 - 1/z} \]
The series is
\[ \frac{1}{z} \ \cdot \ \sum_{n=0}^{\infty} \ [ \ \frac{1}{z} \ ]^n \]
\[ = \sum_{n=0}^{\infty} \ \frac{1}{z^{n+1}} \]

Combining the two results and bringing back the factor we get
\[ \sum_{n=0}^{\infty} \ \frac{2^n}{z^{n+1}} -  \sum_{n=0}^{\infty} \ \frac{1}{z^{n+1}}  \]
\[ = \sum_{n=0}^{\infty} \ (2^{n} - 1) \ \frac{1}{z^{n+1}} \]
adjust the index
\[ = \sum_{n=1}^{\infty} \ (2^{n-1} - 1) \ \frac{1}{z^{n}} \]

\subsection*{example 3}

\label{sec:ex3L}

\[ f(z) = \frac{1}{z(z+2)} \]
Suppose the region of interest is an annulus centered on $z = 1$ with $1 < |z-1| < 3$.
\begin{center} \includegraphics [scale=0.5] {writeseries1.png} \end{center}

The first thing to do is make a substitution that translates the region so that it becomes centered on the origin:  $w = z - 1$.  Then the function becomes
\[  \frac{1}{(w + 1)(w + 3)} \]
The next thing is to write partial fractions.  For the numerator we get
\[ A(w+3) + B(w+1) = 1 \]
\[ A = - B = \frac{1}{2} \]
Hence
\[ \frac{1}{2} \cdot \ [ \ \frac{1}{w+1} - \frac{1}{w+3} \ ] \]
The third step is to convert each of these fractions into something like $1/1-x$.
\[ \frac{1}{w+1} = \frac{1}{1 - (-w)} \]
\[ \frac{1}{w+3}  = \frac{1}{3} \cdot \frac{1}{1 - (-w/3)} \]
And then the fourth step is to write the series, recalling that we want different forms depending on whether we are in a circle or an annulus.

\[ \frac{1}{1 - (-w)} \]
\[ = \sum_{n=0}^{\infty} (-w)^n = \sum_{n=0}^{\infty} (-1)^n (w)^n , \ \ \ |w| < 1 \]
\[ = -\sum_{n=1}^{\infty} \frac{1}{(-w)^n} =  -\sum_{n=1}^{\infty} \frac{(-1)^n}{w^n}, \ \ \ |w| > 1 \]
We pick the second form because our region is $1 < |z-1| < 3$

A similar thing can be done for the other term.  We show only the first series since we are inside the circle.
\[  \frac{1}{3} \cdot \frac{1}{1 - (-w/3)} \]
\[ = \frac{1}{3} \cdot  \sum_{n=0}^{\infty} (-1)^n (\frac{w}{3})^n , \ \ \ |w| < 3 \]
\[ = \sum_{n=0}^{\infty} (-1)^n \ \frac{1}{3^{n+1}} \ w^n , \ \ \ |w| < 3 \]

Add the two series together (remembering the minus sign on the second term)
\[ -\sum_{n=1}^{\infty} \frac{(-1)^n}{w^n} - \sum_{n=0}^{\infty} (-1)^n \ \frac{1}{3^{n+1}} \ w^n \]
and then picking up the leading factor from 
\[ \frac{1}{2} \cdot \ [ \ \frac{1}{w+1} - \frac{1}{w+3} \ ] \]
so
\[ \frac{1}{2} \ [ \ -\sum_{n=1}^{\infty} \frac{(-1)^n}{w^n} - \sum_{n=0}^{\infty} (-1)^n \ \frac{1}{3^{n+1}} \ w^n \ ] \]

The last step is to reverse the substitution:  $w = z - 1$ and bring the minus sign out front
\[ f(z) = - \frac{1}{2} \ [ \ \sum_{n=1}^{\infty} \frac{(-1)^n}{(z-1)^n} \sum_{n=0}^{\infty} (-1)^n \ \frac{1}{3^{n+1}} \ (z-1)^n \ ] \]

I don't know if I could ever learn to do this well, but at least the explanations make sense.

Now, if we were to integrate $f(z)$, we would have only one term that gives a non-zero result, namely the first term with $n=1$
\[ - \frac{1}{2} (-1) \frac{1}{z-1} \]
The residue is the cofactor of that term:
\[ \text{Res }(1) = \lim_{z \rightarrow 1} \frac{1}{2} = \frac{1}{2} \]
Multiply by $2 \pi i$ to obtain $\pi i$.

\subsection*{example 3, partial fractions}

On the other hand, I would just write the partial fraction:
\[ \int \frac{1}{2} \ [ \ \frac{1}{z} - \frac{1}{z + 2} \ ] dz \]

The curve $C[1,3]$ includes the singularity at $z = 0$, but $z = -2$ is on the boundary, not inside the region.  The curve $C[1,1]$ includes neither point.

So for the second curve the integral is zero and for the first one only the $\int 1/z \ dz$ matters.  It's an old friend, with value $2 \pi i$, the value of the integral is thus $\pi i$.

\subsection*{example 6 (Brown and Churchill)}

\label{sec:ex6L}

\[ \int \frac{1}{z(z-2)^4} \ dz \]
with singularities at $z = 0$ and $z = 2$.

Use the $1/z$ part to get a geometric series:
\[ \frac{1}{z(z-2)^4} = \frac{1}{(z-2)^4} \cdot \frac{1}{(2 + z - 2)} \]
\[ = \frac{1}{(z-2)^4} \cdot \frac{1}{2} \cdot \frac{1}{(1 - (-(z-2))/2 )} \]

The third term gives the geometric series with common ratio $(z-2)/2$.  Those $z-2$ terms will cancel the leading factor.  The only term that matters is the cube, which gives:
\[  \frac{1}{(z-2)} \cdot \frac{1}{2} \cdot \frac{1}{(-2)^3} \]

We have that $b_1 = -1/16$ so $I = -\pi i/8$.

\subsection*{example 19 (Brown and Churchill)}

\label{sec:ex19L}

The second one is:
\[ \int e^{1/z^2} \ dz \]
We use the standard series for $e^z$
\[ e^z = 1 + z + \frac{z^2}{2!} + \frac{z^3}{3!} + \dots \]
substituting $1/z^2$
\[ 1 + \frac{1}{z^2} + \frac{1}{2! \ z^4} + \frac{1}{3! \ z^6} + \dots \]

Since there's no $z$ with a power $n=-1$, the value of the integral is just zero.

\end{document}