\documentclass[11pt, oneside]{article} 
\usepackage{geometry}
\geometry{letterpaper} 
\usepackage{graphicx}
	
\usepackage{amssymb}
\usepackage{amsmath}
\usepackage{parskip}
\usepackage{color}
\usepackage{hyperref}

\graphicspath{{/Users/telliott/Github/figures/}}
% \begin{center} \includegraphics [scale=0.4] {gauss3.png} \end{center}

\title{Estimation lemma}
\date{}

\begin{document}
\maketitle
\Large

%[my-super-duper-separator]

There is a basic theorem that is used for several different proofs in the theory of complex functions, but it's not that sexy, so it's usually called a Lemma.

\subsection*{the estimation lemma}

We have seen the (\hyperref[sec:tri_inequality]{\textbf{Triangle inequality}}), which says that for two complex numbers $z$ and $w$:
\[  |z + w| \le |z| + |w| \]

We can write a similar inequality for integrals
\[ | \int_a^b g(t) \ dt | \le \int_a^b | g(t) | \ dt \]

\emph{Proof.}

Since we approximate the integral as a Riemann sum.

\[ |\int_a^b g(t) \ dt| \approx | \sum g(t_k) \ \Delta t | \le \sum | g(t_k) | \ dt \approx \int_a^b |g(t)| \ dt \]

Using that result, we can apply it to the integral along a curve or contour:
\[ | \int_\gamma f(z) \ dz |  \le \int_{\gamma} | f(z) | \ dz \]

\emph{Proof.}

\[ | \int_{\gamma} f(z) \ dz | = | \int_a^b f(\gamma(t)) \gamma'(t) \ dt | \]
\[ \le \int_a^b | f(\gamma(t)) | | \gamma'(t) | \ dt \]
\[ = \int_{\gamma} | f(z) | \ dz \]

So what that means is that if $|f(z)| < M$ along $C$, then
\[ | \int_C f(z) \ dz | \le M \cdot L \]
where $L$ is the length of $C$.

\emph{Proof.}

let $\gamma(t), a \le t \le b$ be a paramtrization of $C$.  Using the triangle inequality:

\[ | \int_C f(z) \ dz | \le \int_C |f(z)| | dz | = \int_a^b |f(\gamma(t))| |\gamma'(t)| dt \]
\[ \le \int_a^b M |\gamma'(t)| \ dt = M \cdot \text{length of C} \]

where we have used the fact that
\[ |\gamma'(t) | \ dt = ds \]
the arc length element.

\end{document}