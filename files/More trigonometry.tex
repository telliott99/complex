\documentclass[11pt, oneside]{article} 
\usepackage{geometry}
\geometry{letterpaper} 
\usepackage{graphicx}
	
\usepackage{amssymb}
\usepackage{amsmath}
\usepackage{parskip}
\usepackage{color}
\usepackage{hyperref}

\graphicspath{{/Users/telliott/Github/figures/}}

\title{Trig 2}
\date{}

\begin{document}
\maketitle
\Large

\subsection*{analyticity}
We proved before that the complex exponential obeys the CRE, which means that it is analytic.  There is a theorem that says that if we add two analytic functions together, the result is also analytic.  Hence, the trigonometric functions are analytic.

But, just to check this result, let's write them out in terms of $u$ and $v$ and see whether the partial derivatives follow the CRE conditions:

\[ \sin z = \sin x \cosh y + i \cos x \sinh y \]
Taking the derivatives:
\[ u(x,y) =  \sin x \cosh y \]
\[ u_x = \cos x \cosh y \]
\[ u_y = \sin x \sinh y \]
and
\[ v(x,y) = \cos x \sinh y \]
\[ v_x = - \sin x \sinh y \]
\[ v_y = \cos x \cosh y \]
So we see that indeed
\[ u_x = v_y \]
\[ u_y = -v_x \]

The CRE are satisfied and therefore, the complex sine is analytic.

Similarly we have that 
\[ \cos z = \cos (x + iy) \]
\[ = \cos x \cos iy - \sin x \sin iy \]
\[ = \cos x \cosh y - i \sin x \sinh y \]
So
\[ u(x,y) =  \cos x \cosh y \]
\[ u_x = - \sin x \cosh y \]
\[ u_y = \cos x \sinh y \]
and
\[ v(x,y) = -\sin x \sinh y \]
\[ v_x = -\cos x \sinh y \]
\[ v_y = -\sin x \cosh y \]
So we see that
\[ u_x = v_y \]
\[ u_y = v_x \]
Thus the complex cosine is also analytic.

We can also prove that:

\[ \sin^2 z + \cos^2 z = 1 \]
The easy way is
\[ \cos^2 z + \sin^2 z = \ [ \frac{e^{iz} + e^{-iz}}{2} \ ]^2 +  [ \frac{e^{iz} - e^{-iz}}{2i} \ ]^2 \ ] \]
\[= \frac{e^{2iz} + 2 + e^{-2iz} - e^{2iz} + 2 - e^{-2iz} }{4} \]
\[ = 1 \]

\subsection*{series}
On the other hand, Shankar defines the trig functions and the exponential using series in the same way as the real versions:

\[ \sin z = z - \frac{z^3}{3!} + \frac{z^5}{5!} - \frac{z^7}{7!} \dots = \sum_0^{\infty} (-1)^n \frac{z^{2n+1}}{(2n +1)!} \]
\[ \cos z = 1 - \frac{z^2}{2!} + \frac{z^4}{4!} - \frac{z^6}{6!} \dots = \sum_0^{\infty} (-1)^n \frac{z^{2n}}{(2n)!} \]

\[ \sinh z = \sum_0^{\infty} \frac{z^{2n+1}}{(2n +1)!} \]
\[ \cosh z = \sum_0^{\infty} \frac{z^{2n}}{(2n)!} \]
and showing that they converge for any $z$.

\subsection*{complex hyperbolics}

The definition is analogous to the real case:
\[ \cos z = \frac{1}{2} \ [ \ e^{z} + e^{-z} \ ]  \]
\[ = \frac{1}{2} \ [ \ e^{i(x+iy)} + e^{-i(x + iy)} \ ]  \]
\[ = \frac{1}{2} \ [ \ e^{ix - y} + e^{-ix + y} \ ]  \]

Double the top and the bottom
\[ = \frac{e^{ix - y} + e^{-ix + y} + e^{ix - y} + e^{-ix + y}}{4}  \]

The pattern in the exponents is 
\[ +- \ \ -+ \ \ +- \ \ -+ \]

We reach a new pattern by first switching the order to 
\[ -+ \ \ +- \ \ -+ \ \ +- \]
\[ = \frac{e^{-ix + y} + e^{ix - y} + e^{-ix + y}  + e^{ix - y}}{4} \]

then add and subtract terms with $++$ and $--$, like this:
\[ = \frac{e^{ix + y} + e^{-ix + y} + e^{ix - y} + e^{-ix - y}}{4} - \frac{e^{ix + y} - e^{-ix + y}  - e^{ix - y} + e^{-ix - y}}{4}  \]

Now we realize that we can factor the first term as:
\[ = \frac{(e^y + e^{-y})}{2} \ \frac{(e^{ix} + e^{-ix})}{2}  \]
\[ = \cosh y \cos x \]

The second term is:
\[ = - \frac{(e^y - e^{-y})}{2} \ \frac{(e^{ix} - e^{-ix})}{2}  \]
\[ = -i \frac{(e^y - e^{-y})}{2} \ \frac{(e^{ix} - e^{-ix})}{2i}  \]
\[ = -i \sinh y \sin x \]

Putting it all together:
\[ \cos z = \cos x \cosh y - i \sin x \sinh y \]

That required a lot of bookkeeping, and now we have to go back and repeat it all for the sine.  And this is basically a repeat of the derivation in the last chapter.  It's just nice to see the factoring trick.

\end{document}