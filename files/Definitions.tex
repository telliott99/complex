\documentclass[11pt, oneside]{article} 
\usepackage{geometry}
\geometry{letterpaper} 
\usepackage{graphicx}
	
\usepackage{amssymb}
\usepackage{amsmath}
\usepackage{parskip}
\usepackage{color}
\usepackage{hyperref}

\graphicspath{{/Users/telliott/Github/figures/}}

\title{Definitions}
\date{}

\begin{document}
\maketitle
\Large

%[my-super-duper-separator]

This is mainly here for reference, although the first part is crucial.  And it is placed early in the book so that you'll notice it.

We will often want to write an expression for the distance between two points in the complex plane.  Suppose $z$ and $w$ are those two points, with $z = x + iy$ and $w = s + it$.

\begin{center} \includegraphics [scale=0.45] {z_minus_w.png} \end{center}

Subtract:
\[ z - w = (x - s) + i (y - t) \]
Now, since $x < s$ and $y < t$ there is a minus sign for both terms.  As you can seen $z - w = r$ lies in the third quadrant.

But the sign issue goes away when we take the modulus as
\[ |z - w| = \sqrt{(x - s)^2 + (y - t)^2} \]

This is the distance between $z$ and $w$ by the Pythagorean theorem.  It is the modulus of the complex number $r$.  $|r| = |z - w| = |w - z|$.

\subsection*{neighborhood}

A neighborhood is an open disk of radius $r$ (or $\delta$) around a point $z_0$.  The points in the disk can be defined as 
\[ z:  |z - z_0| < \delta \]
all the points $z$ such that the distance between them is less than $r$.

This is an open disk.  It does not include the points on the boundary, defined by $z:  |z - z_0| = \delta$.

\subsection*{deleted neighborhood}
A \emph{punctured disk} or deleted neighborhood is all the points in the neighborhood of $z_0$ except $z_0$ itself:
\[ z:   0 < |z - z_0| < \epsilon \]

Sometimes, the missing points are more than just one, but lie in a smaller disk around $z_0$.  The region is called an annulus:
\[ z:   r < |z - z_0| < R \]

\subsection*{boundaries}

There is a funny kind of language used when talking about sets of points that are either inside, on the boundary of, or outside a set of points.  

$\bullet$ \ $w$ is an \emph{interior} point of $\mathbf{S}$ if $w \in \mathbf{S}$ and there exists a neighborhood of $w$ that includes no boundary points of $\mathbf{S}$.  Of course, \emph{all} points $\in \mathbf{S}$ not on the boundary have this property, because there is no closest number to a number, whether real or rational).

A boundary point is one for which each neighborhood contains points both $\in \mathbf{S}$ and not $\in \mathbf{S}$.

An exterior point of a set is a point for which there exists a neighborhood with no points in $\mathbf{S}$.

A set is open if it does not contain any of its boundary point, closed if it contains all of its boundary points.  The punctured disk is neither open nor closed.

An open set is \emph{connected} if each pair of points can be joined by a polygonal line.  An open set that is connected is called a \emph{domain}.  Any neighborhood is a domain.

A set $\mathbf{S}$ is \emph{bounded} if every point of the set lies within a circle $|z| = R$.

\subsection*{bounded set}
A set $\mathbf{S} \subset \mathbf{C}$ is \emph{bounded} if there is some $M > 0$ such that all $z \in \mathbf{S}$ have the property $|z| < M$.

\subsection*{accumulation point}
A point $z_0$ is called a limit point, cluster point or accumulation of a point set $\mathbf{S}$ if every deleted $\delta$ neighborhood of $z_0$ contains points of $\mathbf{S}$.  Since $\delta$ can be any positive number, it follows that $\mathbf{S}$ must have infinitely many points.  

Note that $z_0$ may or may not belong to the set $\mathbf{S}$.

A set $\mathbf{S}$ is closed if $\mathbf{S}$ if and only if it contains all of its accumulation points.

\subsection*{example}
Show that $0$ is an accumulation point of the set $\mathbf{S}$ $= \{ 1/n, n \in \mathbf{N} \}$.

$\circ$ \ Given $\epsilon > 0$, choose $ n \in \mathbf{N}$ large enough so that $n > 1/\epsilon$

$\circ$ \ So $ 0 < |0 - 1/n| = 1/n < \epsilon$

$\circ$ \ So $1/n \in \mathbf{S}$ is in the deleted neighborhood of $0$.

So for this \emph{finite} set of points, $0$ is an accumulation point, since every deleted neighborhood of the $0$ contains a point in the set.

\subsection*{curves and paths}
Typically we have paths (often designated $\gamma$) which are parametrized curves $\gamma(t)$ for $a \le t \le b$.  The curve generates points $x,y$ for each $t$.

The value of an integral over a curve does not depend on the particular parametrization.

The length of the path is just $\int_a^b |\gamma'(t) \ dt$.

We often designate closed circular paths in the complex plane as $C[w,r]$ where  $r$ is the radius and $w$ is the center of the curve $z: |z - w| = r$.

\subsection*{Limit}
The distance between two points also shows up in limits.  The limit
\[ \lim_{z \rightarrow z_0} \ f(z) = L \]
is defined as follows:

Given $\epsilon > 0$, it is possible to find $\delta > 0$ such that $|z - z_0| < \delta$ implies $|f(z) - L < \epsilon|$.

We can also use the language of neighborhoods:  if for any neighborhood defined for $f(x) - L$ in terms of $\epsilon$ we can guarantee that if $|z-z_0|$ is in the neighborhood defined by radius $\delta$, the limit exists.

Let a function $f$ be defined at all points $z$ in some deleted neighborhood of $z_0$, then the statement
\[ \lim_{z \rightarrow z_0} f(z) = w_0 \]
means that the point $w = f(z)$ can be made arbitrarily close to $w_0$ if we choose the point $z$ close enough to $z_0$ (though distinct from it).
  
Formally, for each positive number $\epsilon$, there exists a positive number $\delta$ such that
\[ |z - z_0| < \delta \ \ \Rightarrow |f(z) - w_0| < \epsilon \]
\begin{center} \includegraphics [scale=0.5] {Brown_Fig23.png} \end{center}
If the limit of a function exists at a point, it is unique.

In the figure we see one of the big consequences of defining complex numbers in terms of two real numbers:  the numbers $z$ and $w$ can lie anywhere in the plane, so rather than show a function as a curve (two numbers in $\mathbb{R}^1$ mapping to $\mathbb{R}^2$, we show a complex number $z$ in $\mathbb{R}^2$ mapping to a new plane containing all possible $w = f(z)$.

\subsection*{Limit of a sequence}
An infinite sequence of complex numbers $z_1 \dots z_n$ has a limit $L$ if, for each positive number $\epsilon$, there exists a positive integer $n_0$ such that $|z_n - L| < \epsilon$ whenever $n>n_0$.

\subsection*{Continuity}
A function $f$ is continuous as a point $z_0$ if:
\[ \lim_{z \rightarrow z_0} f(z) = f(z_0) \]
which of course requires both the left- and right-hand sides:  $f(z_0)$ and $\lim_{z \rightarrow z_0} f(z)$ both exist.

Any polynomial in $z$ is continuous everywhere while any rational function is continuous everywhere except at the zeroes of the denominator.

\subsection*{Differentiable}
A function $f$ is said to be differentiable at $z_0$ if the function's domain includes a neighborhood of $z_0$ and the derivative exists:
\[ f'(z_0) = \lim_{z \rightarrow z_0} \frac{f(z) - f(z_0)}{z - z_0} \]
The existence of the derivative at $z_0$ implies that the function is continuous at that point;  however, the converse is not necessarily true.

\subsection*{Analytic}
A function is analytic at a point if it has a derivative at that point.

\subsection*{Holomorphic}
A function is analytic at a point if it can be represented by a convergent power series in a region around that point.

To be analytic implies holomorphism, and vice-versa.

\subsection*{Entire}
An entire function is a function that is analytic at every point in the entire finite plane.

\subsection*{Singular point}
A point $z_0$ is called a singular point of a function $f$ if $f$ fails to be analytic at $z_0$  but is analytic at some point in every neighborhood of $z_0$ . 

A singular point $z_0$  is said to be isolated if, in addition, there is a deleted neighborhood  of $z_0$  throughout which $f$ is analytic.

\subsection*{Pole}
An isolated singular point is called a pole.  For example
\[ \frac{b_1}{z - z_0} \]
has a pole at $z_0$, since it is undefined there.  A pole of order $m$ would be
\[ \frac{b_1}{(z - z_0)^m} \]

\subsection*{Holomorphic and meromorphic}
Holomorphic is used as a synonym for analytic.  A function $f$ is said to be meromorphic in a domain $D$ if it is analytic throughout $D$ except for poles.

\end{document}