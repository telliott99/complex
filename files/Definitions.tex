\documentclass[11pt, oneside]{article} 
\usepackage{geometry}
\geometry{letterpaper} 
\usepackage{graphicx}
	
\usepackage{amssymb}
\usepackage{amsmath}
\usepackage{parskip}
\usepackage{color}
\usepackage{hyperref}

\graphicspath{{/Users/telliott/Github/figures/}}

\title{Distance}
\date{}

\begin{document}
\maketitle
\Large

%[my-super-duper-separator]

We will often want to write an expression for the distance between two points in the complex plane.  Suppose $z$ and $w$ are those two points, with $z = x + iy$ and $w = s + it$.

Subtract:
\[ z - w = (x - s) + i (y - t) \]

If we take the modulus of this as
\[ |z - w| = \sqrt{(x - s)^2 + (y - t)^2} \]

This is the distance between $z$ and $w$ by the Pythagorean theorem.  It is the modulus of the complex number $z - w$.

\subsection*{neighborhood}

One very useful  concept is that of the neighborhood.  A neighborhood is an open disk of radius $r$ (or $\epsilon$) around a point $z_0$.  The points in the disk can be defined as 
\[ z:  |z - z_0| < \epsilon \]
all the points $z$ such that the distance between them is less than $r$.

This is an open disk.  It does not include the points on the boundary, defined by $z:  |z - z_0| = \epsilon$.

\subsection*{punctured disk}
A punctured disk or deleted neighborhood is all the points in the neighborhood of $z_0$ except $z_0$ itself:
\[ z:   0 < |z - z_0| < \epsilon \]

Sometimes, the missing points are more than just one, but lie in a smaller disk around $z_0$.  The region is called an annulus:
\[ z:   r < |z - z_0| < R \]

\subsection*{boundaries}

There is a funny kind of language used when talking about sets of points that are either inside, on the boundary of, or outside a set of points.  $w$ is an \emph{interior} point of $\mathbf{S}$ if $w \in \mathbf{S}$ and there exists a neighborhood of $w$ that includes no boundary points of $\mathbf{S}$.  (This can happen because there is no closest number to a number, whether real or rational).

A boundary point is one for which each neighborhood contains points both $\in \mathbf{S}$ and not $\in \mathbf{S}$.

An exterior point of a set is a point for which there exists a neighborhood with no points in $\mathbf{S}$.

A set is open if it does not contain any of its boundary point, closed if it contains all of its boundary points.  The punctured disk is neither open nor closed.

An open set is connected if each pair of points can be joined by a polygonal line.  An open set that is connected is called a \emph{domain}.  Any neighborhood is a domain.

A set $\mathbf{S}$ is \emph{bounded} if every point of the set lies within a circle $|z| = R$.

A point is an \emph{accumulation} point of $\mathbf{S}$ if each deleted neighborhood of the point contains at least one point of $\mathbf{S}$,  A closed set contains all of its accumulation points.  A set is closed if an only if it includes all of its accumulation points.  [more work needed]

\subsection*{curves and paths}
Typically we have paths (often designated $\gamma$) which are parametrized curves $\gamma(t)$ for $a \le t \le b$.  The curve generates points $x,y$ for each $t$.

The value of an integral over a curve does not depend on the particular parametrization.

The length of the path is just $\int_a^b |\gamma'(t) \ dt$.

We often designate closed circular paths in the complex plane as $C[w,r]$ where  $r$ is the radius and $w$ is the center of the curve $z: |z - w| = r$.

\subsection*{Limit}
The distance also shows up in limits.  The limit
\[ \lim_{z \rightarrow z_0} \ f(z) = L \]
is defined by saying that given $\epsilon > 0$ and the statement $|f(z) - L < \epsilon|$, then one can find $\delta > 0$ such that $|z - z_0| < \delta$ implies that conclusion.

We can also use the language of neighborhoods:  the limit exists if for any neighborhood defined for $f(x) - L$ in terms of $\epsilon$ we can guarantee that if $|z-z_0|$ is in the neighborhood defined by radius $\delta$, the conclusion is true.

Let a function $f$ be defined at all points $z$ in some deleted neighborhood of $z_0$, then the statement
\[ \lim_{z \rightarrow z_0} f(z) = w_0 \]
means that the point $w = f(z)$ can be made arbitrarily close to $w_0$ if we choose the point $z$ close enough to $z_0$ (though distinct from it).
  
Formally, for each positive number $\epsilon$, there exists a positive number $\delta$ such that
\[ |z - z_0| < \delta \ \ \Rightarrow |f(z) - w_0| < \epsilon \]
\begin{center} \includegraphics [scale=0.5] {Brown_Fig23.png} \end{center}
If the limit of a function exists at a point, it is unique.

\subsection*{Continuity}
A function $f$ is continuous as a point $z_0$ if all three conditions hold:
\[ \lim_{z \rightarrow z_0} f(z) = f(z_0) \]
which of course requires
\[ f(z_0) \text{ exists} \]
\[ \lim_{z \rightarrow z_0} f(z) \text{ exists} \]

\subsection*{Differentiable}
A function $f$ is said to be differentiable at if the function's domain includes a neighborhood of $z_0$ and the derivative exists:
\[ f'(z_0) = \lim_{z \rightarrow z_0} \frac{f(z) - f(z_0)}{z - z_0} \]
The existence of the derivative at $z_0$ implies that the function is continuous at that point;  however, the converse is not true.

\subsection*{Analytic}
A function is analytic at a point if it has a derivative at that point.

\subsection*{Entire}
An entire function is a function that is analytic at each point in the entire finite plane.

\subsection*{Singular point}
A point $z_0$ is called a singular point of a function $f$ if $f$ fails to be analytic at $z_0$  but is analytic at some point in every neighborhood of $z_0$ . 

A singular point $z_0$  is said to be isolated if, in addition, there is a deleted neighborhood  of $z_0$  throughout which $f$ is analytic.

\subsection*{Pole}
An isolated singular point is called a pole.  For example
\[ \frac{b_1}{z - z_0} \]
has a pole at $z_0$, since it is undefined there.  A pole of order $m$ would be
\[ \frac{b_1}{(z - z_0)^m} \]

\subsection*{Holomorphic and meromorphic}
Holomorphic is used as a synonym for analytic.  A function $f$ is said to be meromorphic in a domain $D$ if it is analytic throughout $D$ except for poles.

\subsection*{Limit of a sequence}
An infinite sequence of complex numbers $z_1 \dots z_n$ has a limit $L$ if, for each positive number $\epsilon$, there exists a positive integer $n_0$ such that $|z_n - L| < \epsilon$ whenever $n>n_0$.

\subsection*{Cauchy sequence}



\end{document}