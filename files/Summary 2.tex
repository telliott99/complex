\documentclass[11pt, oneside]{article} 
\usepackage{geometry}
\geometry{letterpaper} 
\usepackage{graphicx}
	
\usepackage{amssymb}
\usepackage{amsmath}
\usepackage{parskip}
\usepackage{color}
\usepackage{hyperref}

\graphicspath{{/Users/telliott/Github/figures/}}

\title{Basic integration}
\date{}

\begin{document}
\maketitle
\Large

%[my-super-duper-separator]

A general complex function $f(z)$ takes a complex number $z$, which is really just an ordered pair $(x,y)$ and feeds that number to a pair of real functions of two real variables, which each output a single real number.  So
\[ z = x + iy \]
\[ f(z) = u(x,y) + v(x,y) \]
\[ dz = dx + i dy \]

We compute integrals as line integrals along a curve (or contour) by doing
\[ \int f(z) \ dz = \int (u + iv)(dx + i dy) \]
\[ = \int u \ dx - v \ dy + i \ [ \ v \ dx + u \ dy \ ] \]
These don't look like it but they are integrals in a single variable, because $x$ and $y$ are related.

There are two kinds of complex functions:  \emph{analytic} and otherwise.  The analytic functions are \emph{good} functions, they follow the rules we know from basic calculus, and can be differentiated and integrated in analogous forms.  We did some examples in $x$ and $y$ like square and triangular paths.

We discovered that integration of analytic functions along a closed path gives a result of zero, except when the function is not defined at some point in the region.

More commonly, integration around a circular contour is of interest, often on a unit circle.  In that case, we have a parameter $t$ and the function said to be parametrized.
\[ \int_C f(z) \ dz = \int_a^b f \ [ \ z \ ] \ z'(t) \ dt \]

For example:
\[ z = r e^{it} \]
On a unit circle around the origin, $r$ is a constant and
\[ dz = r(i e^{it}) \ dt = iz \ dt \]
So, for example, the inverse function $1/z$ gives
\[ \int \frac{1}{z} \ iz \ dt = i \int dt = 2 \pi i \]
Around a closed path, the value of the integral is $2 \pi i$.  

This occurs despite the fact that $1/z$ obeys the CRE and is analytic.  The problem is that it is undefined at the origin and not analytic there.

The factor of $2 \pi i$ will come up repeatedly from this point.

Another way to explain this is to say 
\[ \frac{1}{z} = \frac{z^*}{zz^*} \]

The denominator is $x^2 + y^2 = r^2$ which is constant for any circular path, so we have
\[ \int \frac{1}{z} \ dz = k \int z^* \ dz \]
and $z*$ is definitely not analytic since $z^* = x - iy$ and $u_x = 1 \ne v_y = -1$.

We also did some other examples, such as $1/z^2$.  On the unit circle
\[ z = e^{it} \]
\[ dz = iz \ dt \]
so the integral is
\[ \int \frac{1}{z^2} \ iz \ dt = \int \frac{1}{z} \ i \ dt \]
\[ = i \int e^{-it} \ dt = i \ \frac{1}{-i} \ e^{-it} = - e^{-it} \]

From Euler
\[ e^{ix} = \cos x + i \sin x \]
but evaluated around a closed path, this is zero because the sine and cosine have a period of $2 \pi$.

\subsection*{Cauchy's integral theorem}

Cauchy's first theorem says that:
\[ \oint_C f(z) \ dz = 0 \] 
for an analytic function around a region without any singularity.

We proved this theorem, it follows very simply from Green's theorem.

A corollary of this theorem is that the result of an integral between any two points over two different paths, is equal.

\subsection*{Cauchy's integral formula}

With these two results, complex analysis starts to get a bit wild.  

If we can write an integral in this form:
\[ \oint_{C} \frac{f(z)}{z-z_0} \ dz \]

where $f(z)$ is analytic and defined everywhere in the domain we care about, with this composite function of course not defined at $z = z_0$.

We parametrize the curve as a circle of radius $r$ around the point $z_0$
\[ z = z_0 + re^{it} \]
$z_0$ is a constant so
\[ dz = ri e^{it} \ dt =  i(z-z_0) \ dt \] 

and then we can simplify the integral as
\[ i \oint_{C} f(z) \ dt \]

It is easy to show that the value of this integral does not depend on the radius of the path, so we let the radius shrink and approach zero.  

The magic thing is that $f(z) \rightarrow f(z_0)$, but $f(z_0)$ is a \emph{constant}.  It can come out from under the integral:
\[ = i f(z_0) \int_{C} dt \]

This integral is just $2\pi$, so the whole thing is $2 \pi i f(z_0)$ and we can write:
\[ \oint_C \frac{f(z)}{z-z_0} \ dz = 2 \pi i \cdot f(z_0) \]

This is Cauchy's integral formula.

A simple example is the inverse $1/z$, the numerator is $f(z) = 1$ and $z_0 = 0$ so the result is $2 \pi i$ times the value of the function $1$ at the origin, which is $2 \pi i$ and matches what we got by direct computation.

\subsection*{extension}

A specific extension of the Cauchy Integral formula is
\[ f'(z_0) = \frac{1}{2 \pi i} \int_C \frac{f(z)}{(z - z_0)^2} \ dz \]

Generally:
\[ f^n(z_0) = \frac{n!}{2 \pi i} \int_C \frac{f(z)}{(z - z_0)^{n+1}} \ dz \]

Rearranged:
\[ \frac{2 \pi i}{n!} \ f^n(z_0) = \int_C \frac{f(z)}{(z - z_0)^{n+1}} \ dz \]

A function $f$ is said to be differentiable at $z_0$ if the function's domain includes a neighborhood of $z_0$ and the derivative exists:
\[ f'(z_0) = \lim_{z \rightarrow z_0} \frac{f(z) - f(z_0)}{z - z_0} \]

\subsection*{more}

The existence of the derivative at $z_0$ implies that the function is continuous at that point;  however, the converse is not necessarily true.

A function is \emph{analytic} at a point if it has a derivative at that point.

Cauchy's theorem says that the integral around a closed path is zero for a function which is analytic everywhere in a domain.

If such a function is undefined at a limited number of points (e.g. because such values produce zero in the denominator), then those points are called poles or singularities and Cauchy's formula can be used to calculate the value of the integral (called a residue) from the value of the function at those points.

\end{document}