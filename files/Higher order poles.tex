\documentclass[11pt, oneside]{article} 
\usepackage{geometry}
\geometry{letterpaper} 
\usepackage{graphicx}
	
\usepackage{amssymb}
\usepackage{amsmath}
\usepackage{parskip}
\usepackage{color}
\usepackage{hyperref}

\graphicspath{{/Users/telliott/Github/figures/}}

\title{Higher order poles}
\date{}

\begin{document}
\maketitle
\Large

%[my-super-duper-separator]

\subsection*{poles of higher order}

Here is the corollary to Cauchy's integral formula:

\[ f^{(n)} \ (z_0) = \frac{n!}{2 \pi i} \ \oint_C \frac{f(z)}{(z-z_0)^{n+1}} \ dz \]

I find this confusing, since we would usually write $\oint f(z) \ dz$ and include the denominator as part of the function.  So let's rewrite it:
\[ f(z) = \frac{g(z)}{(z - z_0)^{n+1}} \]

To compute $\oint f(z) \ dz$, the first thing is to clear the $(z-z_0)^{n+1}$ term from the denominator.  Consider only $g(z)$.

Then compute $g'(z)$ and evaluate that in the limit as $z \rightarrow z_0$.  \

That's the residue.  Don't forget the factor of the factorial.  Remember that the argument of the factorial and the order of the derivative are the same, and they are $1$ less than the power of the factor in the denominator that is causing the trouble.

Here is Rule II from Kaplan.

At a pole of order $N$ (where $N = 2, 3, \dots$),
\[ \text{Res } [f(z),z_0] = \lim_{z \rightarrow z_0} (z-z_0) \frac{g^{(N-1)}(z)}{(N-1)!} \]
where 
\[ g(z) = (z-z_0)^N f(z) \]

Notice that rather than $n+1$ in the denominator they have $N$, and rather than $n$ in the derivative and the factorial, they have $N-1$.

\subsection*{steps to follow}

Double pole. $n = 2$.  Multiply first:
\[ g(z) = (z - z_0)^2 \cdot f(z) \]
Compute $g'(z)$.  (The n-1 derivative).  Then evaluate $\lim_{z \rightarrow z_0} g'(z)$.

Triple pole. $n = 3$  Multiply first:
\[ g(z) = (z - z_0)^3 \cdot f(z) \]
Compute $g''(z)$.  (The n-1 derivative).  Then evaluate $2 \cdot \lim_{z \rightarrow z_0} g'(z)$.  
Remember the factor of $(n-1)!$, the same as the degree of the derivative. 

These are residues, don't forget the factor of $2 \pi i$ for the integral.

\subsection*{example 5}

This can be done in many ways.

\subsection*{example 5, partial fractions}

\label{sec:ex5PF}

As a check, let's do this by partial fractions.
\[ \frac{1}{z(z-2)^2} = \frac{A}{z} +  \frac{B}{z(z-2)} +  \frac{C}{(z-2)^2}  \]
Hence in putting all terms over a common denominator, for the numerator we have
\[ A(z^2 - 4z + 4) + B(z^2 - 2z) + Cz = 1 \]

From which we get three equations:
\[ z^2:  A + B = 0 \]
\[ z^1: -4A + -2B + C = 0 \]
\[ z^0:  4A = 1 \]

Hence, $A = -B$ and $A = 1/4$. Then $-1 + 1/2 + C = 0$, so $C = 1/2$.

\[ \frac{1}{z(z-2)^2} = \frac{1/4}{z} - \frac{1/4}{z-2} +  \frac{1/2}{(z-2)^2}  \]

which we check by doing
\[ 1/4 \cdot (z^2 - 4z + 4) - 1/4 \cdot (z^2 - 2z) + 1/2 z = 1 \]
So how to deal with 
\[ \frac{1/4}{z} - \frac{1/4}{z-2} +  \frac{1/2}{(z-2)^2}  \]

The last term has a pole of order $2$ at $z = 2$.  We remove that factor and compute the $N-1$ (first) derivative of what's left, which is just zero.  Alternatively, substitute $w = z - 2$, $dw = dz$.  Then $\oint 1/z^2 \ dz = 0$.

For the rest, we have two simple poles at $z=0$ and $z=2$.  The residues are
\[ \text{Res } (0) = \lim_{z \rightarrow 0} \ \ z \cdot \frac{1/4}{z} = \frac{1}{4} \]
\[ \text{Res } (2) = \lim_{z \rightarrow 2} \ \ (z-2) \cdot \frac{-1/4}{z-2} = - \frac{1}{4} \]
which add up to zero.

\subsection*{example  11}

\label{sec:ex11R}

\[ f(z) = \frac{1}{z^4 + z^3 - 2z^2} \]
where $C$ is the circle $|z| = 3$ with positive orientation.

The denominator can be factored as
\[ z^2(z^2 + z - 2) = z^2(z + 2)(z - 1) \]
so
\[ f(z) = \frac{1}{z^2(z + 2)(z - 1)} \]
There is a pole of order 2 at the origin and simple poles at 1 and -2.  All of these lie within the contour $|z| = 3$.
\[ \text{Res }(1) = \lim_{z \rightarrow 1} (z-1) \ f(z) = \lim_{z \rightarrow 1} \frac{1}{z^2 (z + 2)} = \frac{1}{3} \]
\[ \text{Res }(-2) = \lim_{z \rightarrow -2} (z+2) \ f(z) = \lim_{z \rightarrow -2} \frac{1}{z^2 (z - 1)} = -\frac{1}{12} \]
For the double pole, we remove the factor of $1/z^2$, take the first derivative of what's left
\[ \frac{d}{dz} \ \frac{1}{z^2 + z - 2} = \frac{(-1)(2z + 1)}{(z^2 + z - 2)^2} \]
Evaluate at zero and obtain.
\[ \text{Res }(0) = - \frac{1}{4} \]
The total of the residues is
\[ \frac{1}{3} -\frac{1}{12} - \frac{1}{4} = 0 \]

As Mathews and Howell say:
\begin{quote}The value 0 for the integral is not an obvious answer, and all of the preceding calculations are required to find it.\end{quote}

\end{document}