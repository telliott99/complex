\documentclass[11pt, oneside]{article} 
\usepackage{geometry}
\geometry{letterpaper} 
\usepackage{graphicx}
	
\usepackage{amssymb}
\usepackage{amsmath}
\usepackage{parskip}
\usepackage{color}
\usepackage{hyperref}

\graphicspath{{/Users/telliott/Github/figures/}}

\title{Higher order poles}
\date{}

\begin{document}
\maketitle
\Large

%[my-super-duper-separator]

\subsection*{poles of higher order}

Here is the corollary to Cauchy's integral formula:

\[ f^{(n)} \ (z_0) = \frac{n!}{2 \pi i} \ \oint_C \frac{f(z)}{(z-z_0)^{n+1}} \ dz \]

I find this confusing, since we would usually write $\oint f(z) \ dz$ and include the denominator as part of the function.  So let's rewrite it:
\[ f(z) = \frac{g(z)}{(z - z_0)^{n+1}} \]

To compute $\oint f(z) \ dz$, the first thing is to clear the $(z-z_0)^{n+1}$ term from the denominator.  Consider only $g(z)$.

Then compute $g'(z)$ and evaluate that in the limit as $z \rightarrow z_0$.  \

That's the residue.  Don't forget the factor of the factorial.  Remember that the argument of the factorial and the order of the derivative are the same, and they are $1$ less than the power of the factor in the denominator that is causing the trouble.

Here is Rule II from Kaplan.

At a pole of order $N$ (where $N = 2, 3, \dots$),
\[ \text{Res } [f(z),z_0] = \lim_{z \rightarrow z_0} (z-z_0) \frac{g^{(N-1)}(z)}{(N-1)!} \]
where 
\[ g(z) = (z-z_0)^N f(z) \]

Notice that rather than $n+1$ in the denominator they have $N$, and rather than $n$ in the derivative and the factorial, they have $N-1$.

\subsection*{steps to follow}

Double pole. $n = 2$.  Multiply first:
\[ g(z) = (z - z_0)^2 \cdot f(z) \]
Compute $g'(z)$.  (The n-1 derivative).  Then evaluate $\lim_{z \rightarrow z_0} g'(z)$.

Triple pole. $n = 3$  Multiply first:
\[ g(z) = (z - z_0)^3 \cdot f(z) \]
Compute $g''(z)$.  (The n-1 derivative).  Then evaluate $2 \cdot \lim_{z \rightarrow z_0} g'(z)$.  
Remember the factor of $(n-1)!$, the same as the degree of the derivative. 

These are residues, don't forget the factor of $2 \pi i$ for the integral.

\subsection*{example 20}

\label{sec:ex20R}

Here is one we can do many different ways:
\[ f(z) = \frac{e^z}{z^2} \]

There is obviously a double pole at $z = 0$.  (1) Multiply by $z^2$.  (2) Take the derivative, obtaining $e^z$.  Evaluate at $z = 0$, obtaining $1$ for the residue.

\label{sec:ex20L}

Another way is to write a Laurent series:
\[ = \frac{1}{z^2} (1 + z + \frac{z^2}{2!} + \frac{z^3}{3!} \dots ) \]
The only term with a non-zero integral is\
\[ \frac{1}{z} \]
so the cofactor $b_1$ is just $1$, which is also the residue, and the value of the integral is $2 \pi i$.

\subsection*{example 5}

This can be done is many ways.

\label{sec:ex5R}

\[ f(z) = \frac{1}{z(z-2)^2} \]
We have a pole of first order at $z_0=0$ and one of second order at $z_0=2$.  

At the first, multiply by $(z - 0)$ and evaluate the limit of what's left:
\[ \text{Res }(0) = \lim_{z \rightarrow 0} \frac{1}{(z-2)^2} = \frac{1}{4} \]

For the other, multiply by $(z-2)^2$ and compute the $N-1$ (first) derivative of what's left
\[ [ \frac{1}{z} ]' \ =  - \frac{1}{z^2} \]
\[ \text{Res } (2) = \lim_{z \rightarrow 2} -\frac{1}{z^2} = -\frac{1}{4} \]
Don't forget to divide by $(N-1)!$, which is just $1$ in this case.  The total is just zero.

\subsection*{example 5, partial fractions}

\label{sec:ex5PF}

As a check, let's do this by partial fractions.
\[ \frac{1}{z(z-2)^2} = \frac{A}{z} +  \frac{B}{z(z-2)} +  \frac{C}{(z-2)^2}  \]
Hence in putting all terms over a common denominator, for the numerator we have
\[ A(z^2 - 4z + 4) + B(z^2 - 2z) + Cz = 1 \]

From which we get three equations:
\[ z^2:  A + B = 0 \]
\[ z^1: -4A + -2B + C = 0 \]
\[ z^0:  4A = 1 \]

Hence, $A = -B$ and $A = 1/4$. Then $-1 + 1/2 + C = 0$, so $C = 1/2$.

\[ \frac{1}{z(z-2)^2} = \frac{1/4}{z} - \frac{1/4}{z-2} +  \frac{1/2}{(z-2)^2}  \]

which we check by doing
\[ 1/4 \cdot (z^2 - 4z + 4) - 1/4 \cdot (z^2 - 2z) + 1/2 z = 1 \]
So how to deal with 
\[ \frac{1/4}{z} - \frac{1/4}{z-2} +  \frac{1/2}{(z-2)^2}  \]

The last term has a pole of order $2$ at $z = 2$.  We remove that factor and compute the $N-1$ (first) derivative of what's left, which is just zero.  Alternatively, substitute $w = z - 2$, $dw = dz$.  Then $\oint 1/z^2 \ dz = 0$.

For the rest, we have two simple poles at $z=0$ and $z=2$.  The residues are
\[ \text{Res } (0) = \lim_{z \rightarrow 0} \ \ z \cdot \frac{1/4}{z} = \frac{1}{4} \]
\[ \text{Res } (2) = \lim_{z \rightarrow 2} \ \ (z-2) \cdot \frac{-1/4}{z-2} = - \frac{1}{4} \]
which add up to zero.

\subsection*{example 5, change of variables to allow series}

Change of variables can also help here.
\[ \frac{1}{z(z-2)^2} \]
Let $w = z - 2$, $dw = dz$, $z = w + 2$
\[ \oint \frac{1}{w^2(w+2)} dw \]

This allows the possibility of a series solution, but we must ask where is the contour?

Suppose the contour includes both of the original poles.  Because of that factor of $1/w^2$, we need the $A$ series for $1/(w+2)$, which is \hyperref[sec:a_plus_z]{\textbf{here}}.

\[ \frac{1}{2} \cdot \ [ \ 1 - \frac{w}{2} + \frac{w^2}{4} \dots \ ]    \]
With the factor, the relevant term is $b = -1/4$, which matches what we had above.

We would have a different series and a different answer for a smaller contour only enclosing one pole.

\subsection*{example 26}

\label{sec:ex26R}

\[ \oint_C \frac{e^z}{z^3 - z^2 - 5z - 3} \ dz \]

The factorization was given in the original problem, but suppose we don't have it.  Guess:
\[ (z ...)(z^2 ... ) \]
The next term is $-z^2$ so maybe 
\[ (z -3)(z^2 + 2z ... ) \]
Then we'll need $-3$ at the end:
\[ (z - 3)(z^2 + 2z + 1) \]
and we get $-5z$, so that works!

Nicely, the quadratic also factors:
\[ \oint_C \frac{e^z}{(z+1)^2 (z-3)} \]

A double pole at $z = -1$ and a single one at $z = 3$.

Recall the general approach for a double pole, construct
\[ g(z) = (z - z_0)^2 \ f(z) \]
Here
\[ = \frac{e^z}{(z-3)} \]

Then compute the first derivative:
\[ \frac{e^z \cdot (z-3) - (e^z \cdot 1)}{(z - 3)^2} \]
\[ = \frac{(z-4)e^z}{(z-3)^2} \]

Evaluate the limit of that as $z \rightarrow z_0$.
\[ \lim_{z \rightarrow -1} \ \frac{(z-4)e^z}{(z-3)^2} = -\frac{5}{16} \ \frac{1}{e} \]

Remember the extra factor of $1/(n-1)!$ to get the residue, and $2 \pi i$ to get the value of the integral.
\[ I = -\frac{5 \pi i}{8 e} \]

\subsection*{example  11}

\label{sec:ex11R}

\[ f(z) = \frac{1}{z^4 + z^3 - 2z^2} \]
where $C$ is the circle $|z| = 3$ with positive orientation.

The denominator can be factored as
\[ z^2(z^2 + z - 2) = z^2(z + 2)(z - 1) \]
so
\[ f(z) = \frac{1}{z^2(z + 2)(z - 1)} \]
There is a pole of order 2 at the origin and simple poles at 1 and -2.  All of these lie within the contour $|z| = 3$.
\[ \text{Res }(1) = \lim_{z \rightarrow 1} (z-1) \ f(z) = \lim_{z \rightarrow 1} \frac{1}{z^2 (z + 2)} = \frac{1}{3} \]
\[ \text{Res }(-2) = \lim_{z \rightarrow -2} (z+2) \ f(z) = \lim_{z \rightarrow -2} \frac{1}{z^2 (z - 1)} = -\frac{1}{12} \]
For the double pole, we remove the factor of $1/z^2$, take the first derivative of what's left
\[ \frac{d}{dz} \ \frac{1}{z^2 + z - 2} = \frac{(-1)(2z + 1)}{(z^2 + z - 2)^2} \]
Evaluate at zero and obtain.
\[ \text{Res }(0) = - \frac{1}{4} \]
The total of the residues is
\[ \frac{1}{3} -\frac{1}{12} - \frac{1}{4} = 0 \]

As Mathews and Howell say:
\begin{quote}The value 0 for the integral is not an obvious answer, and all of the preceding calculations are required to find it.\end{quote}

\subsection*{example 21}

\label{sec:ex21R}

\[ f(z) = \frac{1 + e^z}{z^2} + \frac{2}{z} \]
We can break this up into its two component parts.  For the first term, the pole is of order $m = 2$ at $z_0 = 0$.  We remove the $z^2$ term and take the derivative
\[ (1 + e^z)' = e^z \]

The factorial term is just $1$, leaving $e^z$ which is evaluated at the pole giving a residue
\[ \text{Res }(0) = e^0 = 1 \]

The other term is just $2$ times the standard
\[ \oint \frac{1}{z} \ dz = 2 \pi i \]
Here $I = 4 \pi i$ and the residue is $2$.  Alternatively just use
\[ I = 2 \pi i f(z_0) = 4 \pi i \]
where $f = 2$.

The total of the residues is $3$ and the value of the integral is $6 \pi i$.

\subsection*{example 28}

\label{sec:ex28R}

\[ f(z) = \frac{e^z}{z(z-1)^2} \]
The denominator is one we saw above, but now there is an extra factor of $e^z$.

We have a pole of first order at $z=0$ and one of second order at $z=1$.  At the first
\[ \text{Res } [f(z),z=0] = \lim_{z \rightarrow 0} \frac{e^z}{(z-1)^2} = 1 \]

For the other one, remove the factor of $1/(z-1)^2$ and compute the $N-1$ (first) derivative of what's left
\[ \text{Res } [f(z),z=1] = \lim_{z \rightarrow 1} \ [ \ \frac{e^z}{z} \ ]' \ \]
\[ = \frac{e^z z - e^z}{z^2} = \frac{e^z(z - 1)}{z^2} \ \bigg |_1 =  0 \]
Hence
\[ \oint f(z) \ dz = 2 \pi i \ [ \ \sum  \text{Res } \ ] \ = 2 \pi i \]

\subsection*{example 6}

\label{sec:ex6R}

\[ f(z) = \frac{1}{z(z-2)^4} \]
We have a pole of first order at $z=0$ and one of fourth order at $z=2$.  

At the first
\[ \text{Res } [f(z),z=0] = \lim_{z \rightarrow 0}   \frac{1}{(z-2)^4} \]
\[ =  \lim_{z \rightarrow 0} \ \frac{1}{(z-2)^4}  = \frac{1}{(-2)^4} =  \frac{1}{16} \]

For the other pole recall that
\[ \frac{2 \pi i}{n!} f^n(a) = \oint_C \frac{f(z)}{(z-a)^{n+1}} \ dz \]

We remove the factor of $1/(z-2)^4$ leaving $f(z) = 1/z$ and then compute the $N-1$ (third) derivative of what's left

\[ \text{Res } [f(z),z=2] = \frac{1}{n!} \ \lim_{z \rightarrow 2} \ [ \ \frac{1}{z} \ ]''' \ \]
\[ f(z) = z^{-1} \]
\[ f'(z) = - z^{-2} \]
\[ f''(z) = 2 z^{-3} \]
\[ f'''(z) = -6 z^{-4} \]
\[  \lim_{z \rightarrow 2} \ [ \ \frac{1}{z} \ ]''' = - \frac{6}{16} \]
Don't forget to divide by $(N-1)!$, which is $3! = 6$ in this case.  That leaves
\[ \text{Res } [f(z),z=2] = - \frac{1}{16} \]

The total of the residues is just zero.  

This problem is from Brown and Churchill, which they work by doing Laurent series.  They get a different answer, namely $-\pi i/8$.  

The reason is that they integrate over the contour $0 < | z - 2 | < 2$, that is, $C[2,2]$, which includes the second pole but not the first.  

Multiplying by $2 \pi i$ gives their result.

Going back to example 6
\[ f(z) = \frac{1}{z(z-2)^4} \]

To expand this, write:
\[ -\frac{1}{16z} \cdot \frac{1}{(1 - z/2)^4} \]
which gives a geometric series in $z/2$.  
\[ = -\frac{1}{16z} (1 + \frac{z}{2} + (\frac{z}{2})^2 \dots)^4 \]

We want the term with $1/z$, which is just the first one of the power series to the fourth power:  $1$.  Thus that term is
\[   -\frac{1}{16z}  \]
The residue is $-1/16$.

We do need to think about the disk of convergence.  We are on $C[2,2]$.  The series has $r = z/2$ which converges if $|z - z_0/2| < 1$, so $|z - z_0| < 2$, which is fine.

The way they do it is to write:
\[ \frac{1}{z(z - 2)^4} = \frac{1}{(z-2)^4} \cdot \frac{1}{2 + (z - 2)} \]
\[ =  \frac{1}{2(z-2)^4} \cdot \frac{1}{1 - (-(z - 2)/2)} \]
which is a geometric series in $-(z-2)/2$.

The term we want preserves a factor with $z^{-1}$ power, that is:
\[ \ [ \ -(z-2)/2 \ ]^3 \]
We have one factor of $1/2$ in the lead, and then a cube of $-1/2$, giving $-1/16$, which agrees with the other methods.

Note:  the example is number 3 in the section on Residues from the 8th edition, which I found online, but it is not available in my paper copy (6th ed.).

\end{document}