\documentclass[11pt, oneside]{article} 
\usepackage{geometry}
\geometry{letterpaper} 
\usepackage{graphicx}
	
\usepackage{amssymb}
\usepackage{amsmath}
\usepackage{parskip}
\usepackage{color}
\usepackage{hyperref}

\graphicspath{{/Users/telliott/Github/figures/}}

\title{Extra}
\date{}

\begin{document}
\maketitle
\Large

%[my-super-duper-separator]

In what follows $z = re^{is}$ or $z = x + iy$, as convenient.

We also use $w = \rho e^{it}$ or $w = u + iv$.

Properties of the conjugate:
P1
\[ (z + w)^* = x - iy + u - iv = z^* + w^* \]
P2
\[ (zw^*)^* = (re^{is}  \rho e^{-it})^* = (r \rho e^{i(s - t)})^* = r \rho e^{i(t - s)} = z^*w \]
P3
\[ z + z* = x + iy + x - iy = 2x = 2 Re \ z \]
P4
\[ |z| = |z^*| \]
Length of a product:
\[ |z| |w| = |re^{is}| \  |\rho e^{-it}| = r \ \rho= |zw| \]

\subsection*{Triangle Inequality}

\begin{center} \includegraphics [scale=0.4] {tri_inequality.png} \end{center}

\[ |z + w|^2 = (z + w)(z + w)^* \]\
$z + w$ is a complex number.  The square of the length of a complex number is equal to the number multiplied by its modulus.
\[ = (z + w)(z^* + w^*) \]
P1 above.
\[ = zz^* + zw^* + z^*w + ww^* \]
Distributivity of multiplication.
\[ = zz^* + zw^* + (zw^*)^* + ww^* \]
P2 above.
\[ = |z|^2 + 2 Re (zw*) + |w|^2 \]
From the definition of the conjugate and P3.

Now we transition to the inequality
\[ \le |z|^2 + 2 |zw^*| + |w|^2 \]
since the real part of a complex number is less than or equal to its length (only equal if it is purely real).
\[ = |z|^2 + 2 |z| |w^*| + |w|^2 \]
See the length of a product, above.
\[ = |z|^2 + 2 |z| |w| + |w|^2 \]
P4 above.
\[ = (|z| + |w|)^2 \]
from basic multiplication.

The inequality follows from taking the square root of the first and last expressions:
\[ |z + w| \le |z| + |w| \]

$\square$

\subsection*{reverse triangle equality}

Surprisingly tricky...
\[ |z_2 - z_1| \ge | \ |z_2| - |z_1| \ | \]

\url{https://math.stackexchange.com/questions/127372/reverse-triangle-inequality-proof}

By the standard triangle inequality:
\[ |x| + |y-x| \ge |x + y - x| = |y| \]
By subtraction
\[ |y-x| \ge |y| - |x| \]

By the same logic or just substituting symbols
\[ |y| + |x-y| \ge |y + x - y| = |x| \]
\[ |x-y| \ge |x| - |y| \]

Let 
\[ t = |y-x|  \]
But $|a| = |-a|$ so also
\[ t = |x-y|  \]

Let $u = |y| - |x|$  We have
\[ t \ge u \]
\[ t \ge -u \]

Since $|a| = |-a|$:
\[ |t| \ge |u| \]
Then
\[ |y-x| \ge | \ |y| - |x| \ | \]

$\square$

\end{document}