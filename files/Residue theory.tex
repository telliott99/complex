\documentclass[11pt, oneside]{article} 
\usepackage{geometry}
\geometry{letterpaper} 
\usepackage{graphicx}
	
\usepackage{amssymb}
\usepackage{amsmath}
\usepackage{parskip}
\usepackage{color}
\usepackage{hyperref}

\graphicspath{{/Users/telliott/Github/figures/}}

\title{Residue theory}
\date{}

\begin{document}
\maketitle
\Large

%[my-super-duper-separator]

\subsection*{summary}

\[ b_1 = \lim_{z \rightarrow z_0} (z-z_0) \ f(z)  \]
\[ \oint f(z) \ dz = 2 \pi i \ \sum \text{ Res } \]

Boas \textbf{definitions}:

Consider the Laurent series for $f(z)$ inside some $C$ centered on $z_0$.  Let $z_0$ be either a regular point or an isolated singular point and there are no other singular points inside $C$.  Then:

$\circ$ \ If all the $b$ coefficients are zero, $f(z)$ is analytic at $z = z_0$ and we call $z_0$ a \emph{regular point}.

$\circ$ \ If $b_n \ne 0$ but all the $b$'s after $b_n$ are zero, then $f(z)$ is said to have a \emph{pole of order n} at $z = z_0$.  If $n=1$ it is called a \emph{simple pole}.

$\circ$ \ If there are an infinite number of $b$'s different from zero, then $f(z)$ has an \emph{essential singularity} at $z = z_0$.

$\circ$ \ The coefficient $b_1$ of $1/(z - z_0)$ is called the \emph{residue} of $f(z)$ at $z = z_0$.

\subsection*{example}

\[ e^z = 1 + z + \frac{z^2}{2!} + \frac{z^3}{3!} + \dots \]
There are no $b$'s, $e^z$ is analytic, and the residue at $z = z_0$ is $0$.

\[ \frac{e^z}{z^3} = \frac{1}{z^3} + \frac{1}{z^2} + \frac{1}{2!z} + \frac{1}{3!} \dots \]
The pole is of order $3$, the residue at $z = z_0$ is $1/2!$.

\[ e^{1/z} = 1 + \frac{1}{z} + \frac{1}{2!z^2} + \frac{1}{3!z^3} + \dots \]
There is an essential singularity at $z = z_0$, and the residue at $z = z_0$ is $1$.

\subsection*{The residue theorem}

Let $z_0$ be an isolated singular point of $f(z)$, and expand the Laurent series about $z = z_0$ for $f$.  We want to find $\oint f(z) \ dz$.  By Cauchy's integral theorem, the integral of the analytical part of the Laurent series is zero (non-negative exponents for $z - z_0$).

It is easy to show that the negative exponents of power $-2$ and higher also give rise to zero when integrated, because they retain some power of $e^{i \theta}$ which multiplies everything by zero for a closed path.

The exception is the $(z-z_0)$ term, where terms cancel leaving $\int d \theta$:

\[ \oint \frac{b_1}{z - z_0} \ dz = b_1 \int_0^{2 \pi} \frac{ri e^{i \theta}}{r e^{i \theta}} \ d \theta = 2 \pi i \ b_1 \]
$b_1$ is called the residue of $f(z)$ at the singular point inside $C$.  If there is more than one isolated singularity, the value of the integral is $2 \pi i$ times the sum of the residues.

The trick of course, is to know what $b_1$ is equal to.

If we have the Laurent series, then $b_1$ is the coefficient of the $\frac{1}{z - z_0}$ term.

\subsection*{simple pole}

If $f(z)$ has a simple pole at $z = z_0$ we find the residue by a simple consequence of Cauchy's integral formula:
\[ \int \frac{f(z)}{z - z_0} \ dz = 2 \pi i \cdot f(z_0) \]

in the limit that $z \rightarrow z_0$ it can come out from the integral sign:
\[ \int f(z) \ dz = \lim_{z \rightarrow z_0} (z - z_0) \cdot 2 \pi i \cdot f(z_0) \]

Stated in terms of the residue: 
\[ R = \lim_{z \rightarrow z_0} (z - z_0) f(z_0) \]

If there is more than one such point
\[ \oint f(z) \ dz = 2 \pi i \ \sum \text{ Res } \]
The value of the integral is $2 \pi i$ times the sum of all the residues enclosed by the path.

There is no longer a factor of $1/z-z_0$ in the integral, just $f(z)$.

Practically, what we do is to look at our function $f$ as
\[ \int f(z) \ dz = \int \frac{g(z)}{z - z_0} \ dz \]
Now remove that denominator $z - z_0$ and write:
\[ b = g(z) \bigg |_{z=z_0} \]

That's the residue.

\subsection*{Removable singularities}
If the residue turns out to be equal to zero, that is called a removable singularity.  
\[ I = \int_C \frac{\sin \pi z}{z^2 - 1} \ dz \]

The denominator can be factored into 
\[ z^2 - 1 = (z + 1)(z - 1) \]
Suppose $C$ includes only $z = 1$, then
\[ \text{Res}(1) = \lim_{z \rightarrow 1}  (z-1) \ \frac{\sin \pi z}{(z + 1)(z - 1)} \]
\[ = \lim_{z \rightarrow 1} \ \frac{\sin \pi z}{(z + 1)} = \frac{\sin \pi}{2} =  0 \]

Here's a trick:
\[ f(z) = z^2 \sin \frac{1}{z} \]
Compute Res$(0)$
\[ \sin \frac{1}{z} = \frac{1}{z} - \frac{1}{3!} \ \frac{1}{z^3} + \frac{1}{5!} \ \frac{1}{z^5} \dots \]
\[ z^2 \sin \frac{1}{z} = z - \frac{1}{3!} \ \frac{1}{z} + \frac{1}{5!} \ \frac{1}{z^3} \dots \]
The only non-zero integral term is 
\[ - \frac{1}{3!} \ \frac{1}{z} \]
and the residue there is
\[ \lim_{z \rightarrow 0} \ (z - 0) (- \frac{1}{3!} \ \frac{1}{z}) \]
\[ = - \frac{1}{3!} = - \frac{1}{6} \]
which we could have just read off the cofactor in the Laurent series.

Since we know series for the trig and exponential functions, this trick comes up a lot:
\[ f(z) = \frac{\sin z}{z} = \frac{1}{z} (z - \frac{z^3}{3!} + \frac{z^5}{5!} + \dots) \]
\[ = 1 - \frac{z^2}{3!} + \frac{z^4}{5!} + \dots \]
$f$ has a removable singularity of $z = 0$, and Res$(f,0) = 0$.

\end{document}