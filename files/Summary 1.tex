\documentclass[11pt, oneside]{article} 
\usepackage{geometry}
\geometry{letterpaper} 
\usepackage{graphicx}
	
\usepackage{amssymb}
\usepackage{amsmath}
\usepackage{parskip}
\usepackage{color}
\usepackage{hyperref}

\graphicspath{{/Users/telliott/Github/figures/}}

\title{Summary of the derivative}
\date{}

\begin{document}
\maketitle
\Large

%[my-super-duper-separator]

We have expressions $f(z) = u(x,y) + i v(x,y)$  for all standard complex functions.

Powers can be computed easily 
\[ z = x + iy \]
\[ (x + iy)^2 = x + 2x(iy) + (iy)^2 \]
\[ (x + iy)^3 = x + 3x^2(iy) + 3x(iy)^2 + (iy)^3 \]
\[ \dots \]

We worked with the inverse function using the complex conjugate $z* = x - iy$ so
\[ \frac{1}{z} = \frac{1}{z} \cdot \frac{z*}{z*} \]

Exponential:
\[ e^z = e^{x + iy} = e^x (\cos x + i \sin x) \]

Sine and cosine:
\[ \cos z = \frac{e^{iz} + e^{-iz}}{2} \]
\[ = \cos x \cosh y - i \sin x \sinh y \]

\[ \sin z = \frac{e^{iz} - e^{-iz}}{2i} \]
\[ = \sin x \cosh y + i \cos x \sinh y \]

Logarithm:
\[ w = \log z = \ln r + i \theta \]

\subsection*{derivatives}

We showed that all of these functions are analytic, with $u_x = v_y$ and $u_y = - v_x$, so therefore, their derivatives can be computed as $f'(z) = u_x + i v_x$.

When this is done, they turn out to be just what you'd want:

\[ (z^n)' = n z^{n-1} \]
\[ (e^z)' = e^z \]
\[ (\sin z)' = \cos z \]
\[ (\cos z)' = - \sin z \]
\[ (\log z)' = \frac{1}{z} \]

The only ones we haven't done are the roots.  

We will not do a general proof, but let's go through the square root.  It will remind us of the special features of polar CRE and derivatives.

\subsection*{square root}
\[ f(z) = \sqrt{z} \]
Use polar notation so $z = re^{i \theta}$ and then
\[ f(z) = \sqrt{z} =  \sqrt{r} e^{i \theta/2} \]

Using Euler:
\[ = \sqrt{r}(\cos \theta/2 + i \sin \theta/2) \]
\[ u = \sqrt{r} \cos \theta/2 \]
\[ v = \sqrt{r} \sin \theta/2 \]

The partials are:
\[ u_r = \frac{1}{2 \sqrt{r}} \cos \theta/2 \]
\[ u_{\theta} = \frac{\sqrt{r}}{2} (- \sin \theta/2) \]
and
\[ v_r = \frac{1}{2 \sqrt{r}} \sin \theta/2 \]
\[ v_{\theta} = \frac{\sqrt{r}}{2} \cos \theta/2 \]

At first we're worried ($u_r \ne v_{\theta}$), but then we recall the polar CRE have an extra factor of $r$:
\[ r u_r = v_{\theta} \]
\[ r v_r = - u_{\theta} \]

So the CRE do obtain, and we can get the derivative.  

Next, we recall the second unusual thing about the polar derivative:
\[ f'(z) = e^{-i\theta} (u_r + i v_r) \]

Leave aside the factor of $e^{-i\theta}$ out front and just combine:
\[ u_r + i v_r = \frac{1}{2 \sqrt{r}} \cos \theta/2 + \frac{1}{2 \sqrt{r}} \sin \theta/2 \]
\[ = \frac{1}{2 \sqrt{r}} \ (\cos \theta/2 + i \sin \theta/2) \]
\[ = \frac{e^{i \theta/2}}{2 \sqrt{r}} \]

which appears problematic, but the extra factor gives us just what we need
\[ f'(z) = e^{-i\theta} \cdot \frac{1}{2 \sqrt{r}} \ e^{i \theta/2} \]
\[ = \frac{1}{2 \sqrt{r}} \ e^{-i \theta/2} \]
\[ = \frac{1}{2 \sqrt{r}} \cdot \frac{1}{e^{i \theta/2}} \]
\[ = \frac{1}{2 \sqrt{z}} \]

Just exactly analogous to the real function.

\end{document}