 \documentclass[11pt, oneside]{article} 
\usepackage{geometry}
\geometry{letterpaper} 
\usepackage{graphicx}
	
\usepackage{amssymb}
\usepackage{amsmath}
\usepackage{parskip}
\usepackage{color}
\usepackage{hyperref}

\graphicspath{{/Users/telliott/Github/figures/}}

% \begin{center} \includegraphics [scale=0.4] {gauss3.png} \end{center}


\title{Integration}
\date{}

\begin{document}
\maketitle
\Large

%[my-super-duper-separator]

\subsection*{summary}
\[ \int f(z) \ dz \]
\[ = \int (u + iv)(dx + i dy) \]
\[ = \int u \ dx - v \ dy + i \ [ \ u \ dy + v \ dx \ ] \]
where $x$ and $y$ are \emph{related}, either as parametric equations in $t$ or $\theta$, or simply because $y = f(x)$.

\subsection*{Introduction}

Complex functions are differentiated and integrated in a way that is similar to real functions, but with some differences.  We've already seen that the derivatives are like their real cousins, and the integrals will be too.

However, there are some differences.  A main one is that the integrals are line integrals.  We will set up integrals which look like multi-variable integrals, but the two variables are connected because they lie on a path.  Because they are line integrals and most integrals will evaluate to something like $e^{it}$, going around on a circle in a closed path usually gives a value of zero.

We often but not always restrict our attention to functions that are analytic, paying attention to points in the complex plane where they have poles (or singularities).  

A complex function is a function that produces a complex number as the result. The most general case is that the input is a complex number as well.  

Se we could write:
\[ w = f(z) \]
where both $w$ and $z$ are complex numbers.

The parts of $f$ that generate the real and imaginary parts of $w$ as separate functions of two \emph{real} variables $x$ and $y$:
\[ w = f(z) = u(x,y) + iv(x,y) \]

\subsection*{getting started}
The function is
\[ w = f(z) = u(x,y) + iv(x,y) \]

The input, the complex variable $z$ is
\[ z = x + i y \]
\[ dz = dx + i dy \]

The key is to write the integral as
 \[ \int f(z) \ dz = \int (u + i v) (dx + i dy) \]
 
Now group the pieces as:
\[ = \int u \ dx - v \ dy + iv \ dx + iu \ dy \]

The integral of a complex function is defined as a sum of integrals of two real variables.  Just as with line integrals for real functions of $x$ and $y$, the variables are related by the curve over which we will integrate.

Recall that for the work integral
\[ \int_C \mathbf{F} \cdot d \mathbf{r} = \int_C M \ dx + N \ dy \]
we parametrize the curve to get the integral over a single variable.

We can view $y$ as a function of $x$ or perhaps, we will be able to parametrize both $x$ and $y$ as functions of $t$.

\subsection*{example: $z$}

Suppose our function is simply $z = x + iy$.  The integral is 
\[ \int z \ dz = \int (x + iy) (dx + i dy) \]
\[ = \int x \ dx - y \ dy + i x \ dy + i y \ dx \]

Now we must get $y$ in terms of $x$ from the curve.  Suppose the curve goes from $(1,i)$ to $(3,i)$, then to $(3,3i)$ and finally back to where we started.  

\begin{center} \includegraphics [scale=0.5] {complex_int_1.png} \end{center}

We have three segments.  Along the first part, we are moving in the positive $x$ direction, with no change in $y$, $y = 1$, a constant, so $dy=0$, and the integral is (writing the non-zero parts only):
\[ = \int x \ dx + i \cdot 1 \ dx \]
\[ = \int_{x=1}^{x=3} \ x + i  \ dx \]
\[ = \frac{x^2}{2} + ix \ \bigg |_1^3 \]
\[ = 4 + 2i \]

Along the second part, we are moving in the positive $y$ direction with $dx = 0$ and $x = 3$ so
\[ = \int_{y=1}^{y=3} - y \ dy + 3 i \ dy \]
\[ = -\frac{y^2}{2} + 3iy \ \bigg |_1^3 \]
\[ = -4 + 6i \]

Along the third path, both $dx$ and $dy$ are non-zero, so we must actually do the parametrization.  The curve is $y=x$.  Hence $dy = dx$.
\[ = \int x \ dx - x \ dx + i x \ dx + i x \ dx \]
\[ = 2i \int x \ dx \]

For the closed path, where we end up back at the starting point, $C3$ should be moving from $(3,3)$ to $(1,1)$ so we have 
\[ 2i \ \frac{x^2}{2} \ \bigg |_3^1 = 2i ( - \frac{8}{2}) = -8i \]

Notice that 
\[ \int_{C1} + \int_{C2} = 8i  = - \int_{C3}\]
If we follow the curve $C3$ from $(3,3)$ to $(1,1)$, the whole thing is just zero.  We'll see that this is not a coincidence.

\subsection*{example 2}
Suppose the function is
\[ f(z) = y - x - i3x^2 \]
So $u = (y-x)$ and $v = -3x^2$ and the integral is
\[ = \int u \ dx - v \ dy + i \ [ \ u \ dy + v \ dx \ ] \]
\[ \int (y - x) dx + 3x^2 dy + i \ [ \ (-3x^2) dx + (y-x) dy \ ]  \]

We proceed from the origin to the point $z = 1 + i$ either directly ($C$) or by first going up vertically ($C1$) and then across ($C_2$).

For the vertical part ($C_1$) we have that $x = 0$ and $dx = 0$.
\[ \int (y - x) dx + 3x^2 dy + i (-3x^2) dx + i (y-x) dy \]
\[ I_1 = \int i (y-x) dy = \int i y dy \]

It's important to recognize that although we are proceeding from the point $z=0$ to the point $z = i$, the upper bound on this integral is not $i$ but $y = 1$!  Hence
\[ I_1 = i\frac{y^2}{2} \ \bigg |_0^1 = \frac{i}{2} \]

For the horizontal part ($C_2$) we have that $y=1$ and $dy = 0$ so
\[ I = \int (y - x) dx + i (-3x^2) dx  \]
\[ = \int (1 - x) dx + i (-3x^2) dx \]

$x$ goes from $0$ to $1$
\[ = x - \frac{x^2}{2} - ix^3 \ \bigg |_0^1 = \frac{1}{2} - i \]

Therefore the total
\[ I = \frac{i}{2} + \frac{1}{2} - i = \frac{1}{2} \ (1 - i) \]

When going directly from the origin to $1 + i$ we relate $x$ to $y$ by the equation of the line $y=x$ so $dy = dx$ and
\[ I = \int (y - x) dx + 3x^2 dy + i (-3x^2) dx + i (y-x) dy \]
\[ = \int 3x^2 dx + i (-3x^2) dx  \]

\[ =  x^3 \  -i  \ x^3 \ \bigg |_0^1  = 1  - i \]
And around the closed curve going backward along $C$:
\[ I1 = i \ \frac{1}{2} \]
\[ I2 = \frac{1}{2} - i \]
\[ I3 = 1 - i \]
That last one is in the direction $0 \rightarrow 1$ so we must subtract it:
\[ \oint f(z) \ dz = i \frac{1}{2} + \frac{1}{2} - i - (1  - i) =  -\frac{1}{2} + i \ \frac{1}{2} \]

This time, even though we returned to our starting point (traversing a \emph{closed} path), the result is not zero.

Notice that
\[ f(z) = y - x - i3x^2 \]
\[ u_x = -1 \ne v_y = 0 \]
This function is not analytic.

\subsection*{example: $z$, revisited}
Above we wrote:
\[ \int z \ dz = \int x \ dx - y \ dy + i x \ dy + i y \ dx \]
And then said:  now we must get $y$ in terms of $x$ from the curve.

But what if we don't worry about the curve?  

Just write $y = f(x)$ and $dy = f'(x) \ dx$ and see what happens:
\[ \int x \ dx - f(x) \ f'(x) \ dx + i \ [ \ x \  f'(x) \ dx + f(x) \ dx \]

It helps that we know the answer:
\[ \frac{1}{2} z^2 = \frac{1}{2} (x^2 - y^2 + i 2xy) \]

\[ \int x \ dx = \frac{x^2}{2} \]
then
\[ \int f(x) \ f'(x) \ dx \]
but this is
\[ \frac{1}{2} \ [f(x)]^2 = \frac{1}{2} y^2 \]
We're on to something!

For the imaginary part:
\[ \int x \  f'(x) \ dx + f(x) \ dx \]
the integrand is the derivative
\[ \ [ \ x \ f(x) \ ]' \]
by the product rule.  But that's just $xy$, so this is a match. 




\end{document}