\documentclass[11pt, oneside]{article} 
\usepackage{geometry}
\geometry{letterpaper} 
\usepackage{graphicx}
	
\usepackage{amssymb}
\usepackage{amsmath}
\usepackage{parskip}
\usepackage{color}
\usepackage{hyperref}

\graphicspath{{/Users/telliott/Github/figures/}}

\title{Arithmetic}
\date{}

\begin{document}
\maketitle
\Large

%[my-super-duper-separator]

\subsection*{definition}

The \emph{complex} numbers $\mathbb{C}$ can be defined as \emph{ordered pairs} of real numbers with the operations of addition:

\[ (x,y) + (a,b) = (x + a, y + b) \]
and multiplication:
\[ (x,y) \cdot (a,b) = (xa - yb, xb + ya) \]
(note the sign of $-yb$).

Then, $\mathbb{C}, +, \cdot$ is a \emph{field}.

We can think of the \emph{real} numbers as ordered pairs whose second coordinate is zero.
\[ (x,0) + (a,0) = (x + a, 0) \]
\[ (a,0) \cdot (x,y) = (ax,ay) \]
The real numbers are a subset of the complex numbers.

\[ \mathbb{R} \subset \mathbb{C} \]
or, as is written, that $\mathbb{R}$ is a subset of $\mathbb{C}$.

Later, we'll look at complex numbers as points in a plane, and the real numbers as those values lying along the $x$-axis.

By the definition of multiplication from above:
\[ (0,1) \cdot (0,1) = (-1,0) \]

This implies that we can write:
\[ (x,y) = (x,0) + (0,y) = (x,0) \cdot (1,0) + (y,0) \cdot (0,1) \]

Think of $(x,y)$ as a \emph{linear combination} of numbers of two kinds.  The first are just real numbers, such as $x \cdot (1,0)$.  The second are exemplified by the real number $y$ times the special number $(0,1)$.  

Give that number a name, $i$, and write
\[ x + iy \]

$x$ is the real part, and $y$ the imaginary part of the complex number $z$.
\[ x = Re \{z\} \]
\[ y = Im \{z\} \]
$y$ is a \emph{real} number, it does not include the $i$.

We can rewrite the addition and multiplication rules and find that
\[ (x + iy)(a + ib) = xa - yb + i(xb + ya) \]
and then our example from above becomes
\[ (0,1) \cdot (0,1) = (-1,0) \]
i.e. $i \cdot i = i^2 = -1$.

Two useful identities come from factoring $i^2 = -1$:
\[ i = -\frac{1}{i}  \]
\[ -i = \frac{1}{i} \]

\subsection*{square root of $-1$}

Consider the functions
\[ x^2 + 1 = 0 \]
and
\[ x^2 + x + 1 = 0 \]

For the first equation, it is easy to see that there is no solution among the real numbers since $x^2$ is always positive or zero.  So adding $1$ to $x^2$ cannot bring the sum back to zero.

Visualizing the same function geometrically, this is just the simple parabola $y=x^2$ shifted up by one unit, moving its vertex from $(0,0)$ to $(0,1)$.  Plotting shows that the graphs of both the above functions never cross the $x$-axis---there are no values that lie on the curve and also on the line $y=0$.

It is often said that complex numbers arose in the context of finding solutions to such polynomials, however, as Nahin writes in his book \emph{An imaginary tale}, this is not really true.

What really happened is that people didn't take negative square roots seriously until they arose in the context of solving cubic equations.  You remember that every cubic, containing a power of $x^3$, crosses the $y$-axis at least once.  Yet negative square roots arise naturally in the solution of problems with real solutions.  This motivated further work.

The ingenious solution to this problem was to invent a new kind of number
\[ i = \sqrt{-1} \]
\[ i^2 = -1 \]

Once we accept that $i = \sqrt{-1}$
then we can factor
\[ (x + i)(x - i) = x^2 - i^2 \]
\[ = x^2 - (-1) = x^2 + 1 \]

so $x = \pm \ i$ are both solutions to the equation 
\[ (x + i)(x - i) = 0 \]

For the second one
\[ x^2 + x + 1 = 0 \]
we can plot it, or we can recall the quadratic formula for solutions to
\[ ax^2 + bx + c = 0 \]
for real constants $a$, $b$ and $c$.  The formula is
\[ \frac{-b \pm \sqrt{b^2 - 4ac}}{2a} \]
When $4ac > b^2$, then the solutions to the quadratic formula involve the square root of a negative number.  Here the formula gives
\[ x = \frac{1}{2} (-1 \pm \sqrt{-3}) \]

Take the positive root and square it
\[ x^2 = \frac{1}{4} (-1 + \sqrt{-3})^2 \]
\[ = \frac{1}{4} (-2 - 2 \sqrt{-3}) \]
\[ =  \frac{1}{2} (-1 - \sqrt{-3})  \]

Adding this to $x+1$ we obtain
\[ x^2 + x + 1 \]
\[ = \frac{1}{2} (-1 - \sqrt{-3}) +  \frac{1}{2} (-1 + \sqrt{-3}) + 1 \]
the terms with $\sqrt{-3}$ cancel, giving
\[ = -\frac{1}{2} - \frac{1}{2} + 1 = 0 \]

In fact, now that we have $i$ available, any square root like $\sqrt{-(a^2)}$, where $a$ is a real number, can be factored as $\sqrt{-1} \ \sqrt{a^2} = ia$.

\subsection*{warning}

Note that the converse is not necessarily true.  Consider
\[ i^2 = \sqrt{-1} \cdot \sqrt{-1} \stackrel{?}{=} \sqrt{(-1)\cdot (-1)} = \sqrt{1} \]

Now, $\sqrt{1}$ has two solutions or roots (since $-1 \times -1$ and $1 \times 1$ are both equal to $1$), but we choose the positive root when thinking about $\sqrt{x}$ as a \emph{function}.  However, $i^2$ was defined to be equal to $-1$, not $1$.  What's the deal?

The problem is that the equality with a question mark is not valid
\[ \sqrt{-1} \cdot \sqrt{-1} \ne \sqrt{(-1)\cdot (-1)} \]
which explains why this "proof" is erroneous.

Expressions that involve the square root of a negative real number, like $\sqrt{-1} = i$ and $\sqrt{-3} = \sqrt{3}\ i$, are called imaginary (or \emph{purely} imaginary).  

Numbers that contain both a real and an imaginary part, like $1 + i$, are termed complex numbers, and imaginary numbers are considered to be complex numbers with the real part equal to $0$.

It turns out that for much of what is done with complex numbers the fact that $i$ equals $\sqrt{-1}$ is not even relevant.

Instead, we simply think of \emph{ordered pairs} of real numbers $(a,b)$ and the $i$ notation is a bookkeeping device, a marker to remind us that when we multiply two complex numbers
\[ (a + ib) (c + id) = ac + iad + ibc + i^2bd \]
the last term gets a minus sign:
\[ ib \cdot id = -bd \]
The result of multiplying $ib \cdot id$ is a real number with the sign flipped, while a real number $a$ times an imaginary number $id$ is equal to $iad$ and
\[ (a + ib) (c + id) = ac -bd + i(ad + bc) \]

\subsection*{dual equality}

Two complex numbers $z_1 = a + ib$ and $z_2 = c + id$ are equal 
\[ z_1 = z_2 \iff a = c \text{ and } b = d \]
\emph{if and only if} both the real and the imaginary parts of $z_1$ and $z_2$ are equal.

\subsection*{matrix form}

Another idea to keep track of the same information is in matrix form, namely:
\[
z = \begin{bmatrix}
a & -b \\
b &  \ \ a
\end{bmatrix}
\]
Such matrices can be added and multiplied in the normal way and give the desired results for complex numbers.  Thus:
\[
\begin{bmatrix}
a & -b \\
b &  \ \ a
\end{bmatrix} \times
\begin{bmatrix}
c & -d \\
d &  \ \ c
\end{bmatrix} =
\begin{bmatrix}
ac - bd & -ad - bc \\
ad + bc &  \ \ ac - bd
\end{bmatrix} 
=
\begin{bmatrix}
 u & -v \\
v &  \ \ u
\end{bmatrix}
\]
\subsection*{Geometric interpretation}
Yet another powerful way to think about complex numbers is to use the complex plane (sometimes called the Argand plane), where points are plotted with the real part along the horizontal axis and the imaginary part along the vertical axis.

This figure is from Brown \& Churchill.
\begin{center} \includegraphics [scale=0.6] {Brown6.png} \end{center}

Looking at the graph, the distance of any point from the origin is denoted by $r$, and $\theta$ is the angle the ray makes with the positive $x$-axis in a CCW direction.  This should be familiar from standard polar coordinates.

Switching notation to
\[ z = x + iy \]
To plot the complex number $z$ we go out $x$ units along the real (horizontal) axis and then up $y$ units along the imaginary (vertical) axis.

The statement that $\mathbb{R} \subset \mathbb{C}$ is equivalent to the observation that the Argand plane contains the horizontal axis.  Real numbers have the form $z = x + i \cdot 0 = x$.

More generally, though
\[ x = r \cos \theta \]
\[ y = r \sin \theta \]
and
\[ x + iy = r \cos \theta + ir \sin \theta\]
\[ = r(\cos \theta + i \sin \theta) \]
\[ = re^{i\theta} \]
where the last part makes use of Euler's famous equation.  $r$ is called the \textbf{modulus} and $\theta$ is called the \textbf{argument} or \textbf{phase}.

If you look very carefully at the figure above the argument $\theta$ is actually $\theta + 2 \pi$.  

All multiples $k \cdot 2 \pi$ for $k \in 0, \pm 1, \pm 2 \dots$ are valid.

Depending on the calculation one form is often easier to handle.  

Addition is simpler with $a + ib$ (the Cartesian format) since
\[ (a + ib) + (c + id) = (a+c) + i (b + d) \]

 while multiplication is more straightforward with the polar format.  
 
 Matrices work well for both addition and multiplication.
 
Here is multiplication in polar coordinates
\[ r e^{i\theta} \ \rho e^{i\phi} = r \rho \ e^{i (\theta + \phi)} \]
We multiply the distances and add the angles.  Here is the square function:
\[ (r e^{i\theta})^2 = r^2 e^{i2\theta} \]

Multiplication of $z_1 = r_1 e^{i\theta_1}$ by $z_2 = r_2 e^{i\theta_2}$ stretches $r_1$ (the length of $z_1$) by the factor $r_2$ (the length of $z_2$), and rotates $z_1$ by adding a phase shift of $\theta_2$ to the original angle $\theta_1$.
\begin{center} \includegraphics [scale=0.6] {Brown9.png} \end{center}

The person who originally discovered this representation was Caspar Wessel.

Since the calculations can be tedious, I wrote a Python script to do the calculations for roots and powers.

\url{https://gist.github.com/telliott99/916bc75a73e515968debe48ef418d738}

\end{document}