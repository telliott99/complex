\documentclass[11pt, oneside]{article} 
\usepackage{geometry}
\geometry{letterpaper} 
\usepackage{graphicx}
	
\usepackage{amssymb}
\usepackage{amsmath}
\usepackage{parskip}
\usepackage{color}
\usepackage{hyperref}

\graphicspath{{/Users/telliott/Github/figures/}}

\title{Simple proofs}
\date{}

\begin{document}
\maketitle
\Large

%[my-super-duper-separator]

\subsection*{No ordering of imaginary or complex numbers}
Suppose $z$ and $w$ are complex numbers.

In general, it makes no sense to ask if $z > w$ or $w > z$.  Equality is OK, that requires both the real and imaginary parts to be equal.

Obviously, we can say that $|z| > |w|$, the length of $z$ or its modulus is greater than the corresponding value for $w$.  But those are real numbers.

And we will often say $|z - z_0| < r$, the distance between $z$ and $z_0$ is less than $r$, where $r$ is a real number.

One can also say that $Re \{ z \} > Re \{ w \}$, although this is not very useful.

Consider $i$.

If $i > 0$ then $i^2 > 0$ but that means $-1 > 0$.

If $i < 0$ then $-i > 0$ and so $(-i)^2 > 0$, but that also means $-1 > 0$.

Either way,  $-1 > 0$ is not a property we want to have.

\subsection*{Properties of the conjugate}

\label{sec:conjugate_properties}

In what follows $z = re^{is}$ or $z = x + iy$, as convenient.  We also use $w = \rho e^{it}$ or $w = u + iv$.

We will use these in the proof of the \hyperref[sec:tri_inequality]{\textbf{triangle inequality}}.

P1
\[ (z + w)^* = x - iy + u - iv = z^* + w^* \]
P2
\[ (zw^*)^* = (re^{is}  \rho e^{-it})^* = (r \rho e^{i(s - t)})^* = r \rho e^{i(t - s)} = z^*w \]
P3
\[ z + z* = x + iy + x - iy = 2x = 2 Re \ z \]
P4
\[ |z| = |z^*| \]

\subsection*{Lemma:  Product of Moduli (Karkhar)}

\label{sec:product_of_moduli}

The modulus of a complex product is the product of moduli:
\[ z = x + iy, \ \ \ \ \ \ w = s + it  \ \ \rightarrow \ \  |zw| = |z||w| \]

Here is one way:

Length of a product:
\[ |z| |w| = |re^{is}| \  |\rho e^{-it}| = r \ \rho= |zw| \]

\emph{Proof}.

\[ zw = (x + iy)(s + it) \]
\[ = xs - yt + i \ [ xt + ys \ ] \]
\[ |zw| = \sqrt{ (xs - yt)^2 + (xt + ys)^2 } \]

work without the square root for a moment:
\[ (xs - yt)^2 + (xt + ys)^2 \]
\[ = (xs)^2 - 2xsyt + (yt)^2 + (xt)^2 + 2xtys + (ys)^2 \]
\[ = (xs)^2 + (yt)^2 + (xt)^2 + (ys)^2 \]
\[ = x^2 (s^2 + t^2) + y^2 (s^2 + t^2) \]
\[ = (x^2 + y^2)(s^2 + t^2) \]
so
\[ |zw| = \sqrt{(x^2 + y^2)(s^2 + t^2)} \]

But
\[ = |z| |w| \]
\[ = \sqrt{|z|^2} \cdot \sqrt{|w|^2} \]
\[ = \sqrt{(x^2 + y^2)} \cdot \sqrt{(s^2 + t^2)} \]
\[ = |zw| \]

\subsection*{Lemma:  length of a curve}

\label{sec:length_of_curve}

If we have a curve parametrized by $\gamma(t)$ where
\[ \gamma(t) = x(t) + i y(t) \]
\[ \gamma'(t) = x'(t) + i y'(t) \]

Then we will be interested in
\[ L = \int_a^b |\gamma'(t)| \ dt \]

where $|\gamma'(t)| = \sqrt{x'(t)^2 + y'(t)^2}$ so
\[ L = \int_a^b \sqrt{x'(t)^2 + y'(t)^2} \ dt \]
which should look familiar from multivariable calculus.
\[ = \int_a^b \sqrt{(dx)^2 + (dy)^2 } \]
\[ = \int_a^b \ ds  \]
We see that $L$ is the length of the curve.

\end{document}