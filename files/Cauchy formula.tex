\documentclass[11pt, oneside]{article} 
\usepackage{geometry}
\geometry{letterpaper} 
\usepackage{graphicx}
	
\usepackage{amssymb}
\usepackage{amsmath}
\usepackage{parskip}
\usepackage{color}
\usepackage{hyperref}

\graphicspath{{/Users/telliott/Github/figures/}}
% \begin{center} \includegraphics [scale=0.4] {gauss3.png} \end{center}

\title{Cauchy's formula}
\date{}

\begin{document}
\maketitle
\Large

%[my-super-duper-separator]

If we can write an integral in this form:
\[ \oint_{C} \frac{f(z)}{z-z_0} \ dz \]

where $f(z)$ is analytic and defined everywhere in the domain we care about, with this composite function of course not defined at $z = z_0$, we can show that the value of the integral is 
\[ \oint_C \frac{f(z)}{z-z_0} \ dz = 2 \pi i \ f(z_0) \]
This is called the Cauchy Integral formula.

\subsection*{setup}
Suppose $f(z)$ is analytic and defined everywhere within some region \emph{except} at a singularity, $z_0$.  For example, suppose we have
\[ \frac{f(z)}{z-z_0} \]

We integrate this function around a special closed path in the region of analyticity:
\[ \oint \frac{f(z)}{z-z_0} \ dz \]

\begin{center} \includegraphics [scale=0.5] {keyhole.png} \end{center}

It's not labeled (I didn't draw the figure) but the singularity $z_0$ is at the center of the two concentric circles.  The "keyhole" excludes $z_0$ so $f$ is analytic everywhere in the region enclosed by the path.

Cauchy's integral theorem tells us that the total integral is zero.

The straight line segments are identical but traversed in opposite directions, so the net contribution from them is zero.

Therefore, we have that the integral around the outer ring counter-clockwise + the integral around the inner ring clockwise add up to zero.

But reversing the direction of integration on the inner ring (so both paths go in the counter-clockwise direction) changes the sign of the value, hence we have that

\[ \oint_{C \ \text{outer}} \frac{f(z)}{z-z_0} \ dz - \oint_{C \ \text{inner}} \frac{f(z)}{z-z_0} \ dz = 0 \]
and
\[ \oint_{C \ \text{outer}} \frac{f(z)}{z-z_0} \ dz = \oint_{C \ \text{inner}} \frac{f(z)}{z-z_0} \ dz \]

We haven't said anything about the radius of these rings.  

What that means is that the value of the integral around a ring enclosing a singularity is not zero, but its value is independent of the radius.

\subsection*{derivation}

We can parametrize this path by realizing that each point on one of these curves is given by
\[ z = z_0 + r e^{i\theta}, \ \ \ 0 \le \theta \le 2 \pi \]
\[ z - z_0 = r e^{i\theta} \]
This is a circle with the center displaced to $z_0$.

Since $z_0$ is a constant
\[ dz = i r e^{i \theta} d \theta \]

so, substituting for $r e^{i\theta}$ above we obtain
\[ dz = i(z - z_0) \ d \theta \]
and
\[ \oint \frac{f(z)}{z - z_0} \ dz = \oint f(z) \ i \ d \theta \]
\[ = i \int_0^{2\pi}  f(z) \ d \theta \]

This holds for every circular path enclosing $z_0$.  We may choose $r$ as small as we like, and so we choose it very small ($r \rightarrow 0$) so
\[ f(z) \rightarrow f(z_0) = \text{constant} \]

and in that limit, since it's constant we can bring it out from under the integral sign!
\[ i \int_0^{2\pi}  f(z) \ d \theta \]
\[ = i f(z_0) \int_0^{2\pi} d \theta = 2 \pi i \ f(z_0) \]

An alternative (equivalent) approach is to say that for a small enough circle
\[ \int \frac{f(z)}{z - z_0} \ dz \]

the value of $f(z)$ approaches $f(z_0)$, a constant so this becomes
\[ = f(z_0) \int \frac{1}{z - z_0} \ dz \]
We know that integral, it is $2 \pi i$.

Summarizing:
\[ \oint \frac{f(z)}{z - z_0} \ dz = 2 \pi i \ f(z_0) \]
What this means is that we can evaluate the integral in question by simply plugging in the value of the function at $z_0$ and multiplying that by $2 \pi i$.

\subsection*{average}
Since
\[ f(z_0) = \frac{1}{2 \pi i} \oint \frac{f(z)}{z - z_0} \ dz \]

and $z$ can be parametrized as $z - z_0 = re^{i \theta}$ so that $dz = i r e^{i \theta} \ d \theta$:

\[ f(z_0) = \frac{1}{2 \pi i} \oint \frac{f(z_0 + re^{i \theta})}{re^{i \theta} } \ i r e^{i \theta} \ d \theta \]
\[ = \frac{1}{2 \pi} \oint f(z_0 + re^{i \theta}) \ d \theta \]
\[ = \frac{1}{2 \pi} \oint f(z) \ d \theta \]

\emph{The value of an analytic function at the center of a circle equals the average (arithmetic mean) of the values on the circumference.}

\end{document}