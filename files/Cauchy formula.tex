\documentclass[11pt, oneside]{article} 
\usepackage{geometry}
\geometry{letterpaper} 
\usepackage{graphicx}
	
\usepackage{amssymb}
\usepackage{amsmath}
\usepackage{parskip}
\usepackage{color}
\usepackage{hyperref}

\graphicspath{{/Users/telliott/Github/figures/}}
% \begin{center} \includegraphics [scale=0.4] {gauss3.png} \end{center}

\title{Cauchy's formula}
\date{}

\begin{document}
\maketitle
\Large

%[my-super-duper-separator]

For an integral in this form:
\[ \oint_{C} \frac{f(z)}{z-z_0} \ dz \]

where $f(z)$ is analytic and defined everywhere in the domain we care about, with this composite function of course not defined at $z = z_0$,

\begin{center} \includegraphics [scale=0.4] {cauchy1.png} \end{center}
If the contour of integration includes $z_0$, then the value of the integral is 
\[ \oint_C \frac{f(z)}{z-z_0} \ dz = 2 \pi i \ f(z_0) \]

This is called the Cauchy Integral formula.  We will prove it in this chapter.  Of course, if there were no singularity, the integral would just be zero.

It's a really amazing result.  Rewrite:
\[ f(z_0) = \frac{1}{2 \pi i} \oint_C \frac{f(z)}{z-z_0} \ dz \]

What this says is that the value of the function $f$ at $z = z_0$ is equal to that integral.  It's likely non-zero, except in special cases.

The value of the function at an interior point \emph{is determined by the values} of this other function $f/(z - z_0)$ \emph{along a curve around it}.  And that's true for \emph{any} curve around $z_0$.  

\subsection*{setup}
Suppose $f(z)$ is analytic and defined everywhere within some region \emph{except} at a singularity, $z_0$.  For example, suppose we have
\[ \frac{f(z)}{z-z_0} \]

We integrate this function around a special closed path in the region of analyticity:
\[ \oint \frac{f(z)}{z-z_0} \ dz \]
\begin{center} \includegraphics [scale=0.4] {keyhole2.png} \end{center}

The singularity $z_0$ is at the center of the two concentric circles.  The "keyhole" excludes $z_0$ so $f$ is analytic everywhere in the region enclosed by the path, which always lies on the left as we traverse.

Cauchy's integral theorem tells us that the total integral is zero.

The straight line segments are identical but traversed in opposite directions, so the net contribution from them is zero.

Therefore, we have that the integral around the outer ring counter-clockwise + the integral around the inner ring clockwise add up to zero.

But reversing the direction of integration on the inner ring (so both paths go in the counter-clockwise direction) changes the sign of the value, hence we have that

\[ \oint_{C \ \text{outer}} \frac{f(z)}{z-z_0} \ dz - \oint_{C \ \text{inner}} \frac{f(z)}{z-z_0} \ dz = 0 \]
and
\[ \oint_{C \ \text{outer}} \frac{f(z)}{z-z_0} \ dz = \oint_{C \ \text{inner}} \frac{f(z)}{z-z_0} \ dz \]

What that means is that the value of the integral around a ring enclosing a singularity is not zero, but its value is \emph{independent of the radius}.

\subsection*{derivation}

We can parametrize this path.  Each point on one of these curves is given by
\[ z = z_0 + r e^{it} \]
with $0 \le t \le 2 \pi$ and then
\[ z - z_0 = r e^{it} \]
This is a circle with the center displaced to $z_0$.

Since $z_0$ is a constant
\[ dz = i r e^{it} \ dt = i (z - z_0) \ dt \]

We will obtain a cancelation exactly like what we saw for the inverse function $1/z$ previously.  We can keep everything in terms of $e^{it}$

\[ \int_{\gamma} \frac{f(z)}{z - z_0} \ dz = \int_{\gamma} \frac{f(z)}{re^{it}} \ ire^{it} \ dt = i \int_{\gamma} f(z) \ dt \]

or use the other form for $dz$ from above
\[ dz = i (z - z_0) \ dt \]
and
\[ \int_{\gamma} \frac{f(z)}{z - z_0} \ dz = \int_{\gamma} \frac{f(z)}{z - z_0} \ i(z - z_0) \ dt = i \int_{\gamma} f(z) \ i \ dt \]

The big idea is that this holds for \emph{every} circular path enclosing $z_0$, by Cauchy's Integral theorem.  

In that case, we may choose $r$ as small as we like, and so we choose it very small ($r \rightarrow 0$) so
\[ f(z) \rightarrow f(z_0) = \text{constant} \]

and in that limit, since it's constant we can bring it out from under the integral sign!
\[ i \int_0^{2\pi}  f(z) \ d \theta = i f(z_0) \int_0^{2\pi} d \theta = 2 \pi i \ f(z_0) \]

What this means is that we can evaluate the integral in question by simply plugging in the value of the function at $z_0$ and multiplying that by $2 \pi i$.

\[ \int_{\gamma} \frac{f(z)}{z - z_0} \ dz = 2 \pi i \ f(z_0) \]

This is Cauchy's Integral \emph{formula}.

In the next chapter, we will see that it is a special case of a more general formula, namely:

\[ \int_{\gamma} \frac{f(z)}{(z - z_0)^{n+1}} \ dz = \frac{2 \pi i}{n!} \ f^{n} (z_0) \]

\subsection*{average}

\emph{The value of an analytic function at the center of a circle equals the average (arithmetic mean) of the values on the circumference.}

Since
\[ f(z_0) = \frac{1}{2 \pi i} \oint \frac{f(z)}{z - z_0} \ dz \]

and $z$ can be parametrized as $z - z_0 = re^{i \theta}$ so that $dz = i r e^{i \theta} \ d \theta$:

\[ f(z_0) = \frac{1}{2 \pi i} \oint \frac{f(z_0 + re^{i \theta})}{re^{i \theta} } \ i r e^{i \theta} \ d \theta \]
\[ = \frac{1}{2 \pi} \oint f(z_0 + re^{i \theta}) \ d \theta \]
\[ = \frac{1}{2 \pi} \oint f(z) \ d \theta \]


\end{document}