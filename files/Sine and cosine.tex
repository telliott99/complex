\documentclass[11pt, oneside]{article} 
\usepackage{geometry}
\geometry{letterpaper} 
\usepackage{graphicx}
	
\usepackage{amssymb}
\usepackage{amsmath}
\usepackage{parskip}
\usepackage{color}
\usepackage{hyperref}

\graphicspath{{/Users/telliott/Github/figures/}}

\title{Sine and cosine}
\date{}

\begin{document}
\maketitle
\Large

%[my-super-duper-separator]

\subsection*{cosine and sine}

Start by recalling Euler's formula for \emph{real} x:
\[ e^{ix} = \cos x + i \sin x \]
Substitute $-x$ for $x$
\[ e^{-ix} = \cos -x + i \sin -x \]
\[ = \cos x - i \sin x \]

Addition gives:
\[ 2 \cos x = e^{ix} + e^{-ix} \]
\[ \cos x = \frac{1}{2} \ (e^{ix} + e^{-ix}) \]

Subtraction:
\[ 2i \sin x = e^{ix} - e^{-ix} \]
\[ \sin x = \frac{1}{2i} \ (e^{ix} - e^{-ix}) \]

Our old friends:
\[ \cosh x =  \frac{1}{2} \ (e^{x}+ e^{-x}) \]
\[ \sinh x =  \frac{1}{2} \ (e^{x} - e^{-x}) \]

\subsection*{complex versions}

The complex counterparts of the real trigonometric functions can be explained by saying that Euler's formula is also good for a complex number $z$ (a math book would define them by their power series).  

By the same algebra, this gives

\[ \cos z = \frac{1}{2} \ (e^{iz} + e^{-iz}) \]
\[ \sin z = \frac{1}{2i} \ (e^{iz} - e^{-iz}) \]

Now we see that the complex sine and cosine have properties just like their real cousins.

We will do the complex hyperbolic functions in the next chapter.

\subsection*{period}

The above definition of cosine is
\[ \cos z = \frac{1}{2} \ (e^{iz} + e^{-iz}) \]
then
\[ cos (z + 2 \pi) = \frac{1}{2} \ (e^{iz} e^{i2 \pi} + e^{-iz} e^{-i 2\pi}) \]
but
\[ e^{i 2 \pi} = \cos 2 \pi + i \sin 2 \pi = 1 \]
and the same for $e^{-i 2 \pi}$, so
\[ \cos (z + 2\pi) = \cos z \]

The \emph{period} of the complex cosine and sine is $2 \pi$, just as for the real function.

\subsection*{derivatives}

Take derivatives is straightforward:
\[ \sin z = \frac{1}{2i} \ (e^{iz} - e^{-iz}) \]
\[ \frac{d}{dz} \ \sin z = i \cdot \frac{1}{2i} \ (e^{iz} + e^{-iz}) = \cos z \]

Similarly
\[ \cos z = \frac{1}{2} \ (e^{iz} + e^{-iz} ) \]
\[ \frac{d}{dz} \ \cos z = \frac{i}{2} \ ( e^{iz} - e^{-iz} ) \]
\[ =- \frac{1}{2i} \ ( e^{iz} - e^{-iz} ) = -\sin z \]

Also
\[ \sin -z = \frac{1}{2i} \ (e^{-iz} - e^{iz}) = - \sin z \]
\[ \cos -z = \frac{1}{2} \ (e^{-iz} + e^{iz} ) = \cos z \]

\subsection*{separating real and imaginary parts of trig functions}

Since
\[ \cos z =  \frac{1}{2} \ (e^{iz} + e^{-iz}) \]
if we let $z = iy$ then
\[ \cos iy =  \frac{1}{2} \ (e^{i^2y} + e^{-i^2y}) \]
\[ =  \frac{1}{2} \ (e^{-y} + e^{y}) = \cosh y \]

Similarly
\[ \sin iy =   \frac{1}{2i} \ (e^{i^2y} - e^{-i^2y}) \]
\[ =   \frac{1}{2i} \ (e^{-y} - e^{y}) \]
\[ = - \frac{1}{2i} \ (e^y - e^{-y}) \]
\[ = - \frac{1}{i} \sinh y  = i \sinh y \]

Hence 
\[ \cos iy = \cosh y \]
\[ \sin iy =  i \sinh y \]

So now if we let $z = x + iy$ and use the standard addition formula
\[ \cos z = \cos (x + iy) \]

gives
\[ \cos z = \cos x \cos iy - \sin x \sin iy \]

Since $\cos iy = \cosh y$ and $\sin iy = i \sinh y$:
\[ = \cos x \cosh y - i \sin x \sinh y \]

and what's nice about this is that we have the real and imaginary parts of the complex cosine easily visible.  

Similarly
\[ \sin z = \sin(x + iy) \]
\[ \sin z = \sin x \cos iy + \cos x \sin iy \]
\[ = \sin x \cosh y + i \cos x \sinh y \]

These are very similar to the sum of angles results for real numbers.  It's just that $z = x + iy$ means the $y$ gets the hyperbolic functions and $\sinh$ has a leading factor of $i$.

\subsection*{using the exponential to get cos $z$ and sin $z$}

We can obtain the same results by working through the formulas using the complex exponential.  

Work backward from the answer:

\[ \cos x \cosh y = \frac{(e^{ix} + e^{-ix})(e^{y} + e^{-y})}{4} \] 
\[ =  \frac{e^{ix}e^{y} + e^{ix}e^{-y} + e^{-ix}e^{y} + e^{-ix}e^{-y}}{4} \]

and then also
\[ i \sin x \sinh y =   \frac{(e^{ix} - e^{-ix})(e^{y} - e^{-y})}{4} \] 
\[ =  \frac{(e^{ix}e^{y} - e^{ix}e^{-y} - e^{-ix}e^{y} + e^{-ix}e^{-y})}{4} \]

\emph{Subtraction} gives cancelations:
\[ =  \frac{e^{ix}e^{-y} + e^{-ix}e^{y}}{2} \]

And now there's a trick.  The exponents of the first product in the numerator add to give
\[ ix - y = i(x + iy) = iz \]
the second is
\[ -ix + y = -(ix - y) = -iz \]

So we have just
\[ \frac{e^{i(x + iy)} + e^{-i(x+iy)}}{2} = \cos z \]

The sine was
\[ \sin z = \sin x \cosh y + i \cos x \sinh y \]

Working with one term at a time, we have
\[ \sin x \cosh y = \frac{(e^{ix} - e^{-ix})}{2i} \cdot \frac{(e^{y} + e^{-y})}{2} \] 
\[ =  \frac{e^{ix}e^{y} + e^{ix}e^{-y} - e^{-ix}e^{y} - e^{-ix}e^{-y}}{4i} \]

and
\[  i \cos x \sinh y =  - \frac{\cos x \sinh y}{i} \]
\[ =  - \frac{1}{i} \cdot \frac{(e^{ix} + e^{-ix})}{2} \cdot \frac{(e^{y} - e^{-y})}{2} \] 
\[ =  - \frac{e^{ix}e^{y} - e^{ix}e^{-y} + e^{-ix}e^{y} - e^{-ix}e^{-y}}{4i} \]
\[ =  \frac{-e^{ix}e^{y} + e^{ix}e^{-y} - e^{-ix}e^{y} + e^{-ix}e^{-y}}{4i} \]

\emph{Addition} gives cancelations:
\[ = \frac{e^{ix}e^{-y} - e^{-ix}e^{-y}}{2i} \]

Recall from above that $ix - y = iz$ and $ix + y = -iz$ so
\[ = \frac{e^{iz} - e^{-iz}}{2i} \]
\[ = \sin z \]

\subsection*{zeroes}

\begin{center} \includegraphics [scale=0.35] {sinhcosh.png} \end{center}
$\cosh$ is never zero, while only $\sinh 0 = 0$.

So if we look again at 
\[ \sin z = \sin x \cosh y + i \cos x \sinh y \]
and ask, where is this function equal to zero?  

Both parts must vanish.  Since $\cosh$ is never zero, $\sin x$ must be zero.  This happens for $x = 2k \pi$.  

The cosine of this $x$ is equal to $1$, that means $\sinh y$ must be $0$ which only happens for $y = 0$.

So the zeroes of the complex sine function are at $z = 2k \pi + 0i$.

Alternatively, go back to the original definition:
\[ \sin z = \frac{1}{2i} (e^{iz} - e^{-iz}) \]
which vanishes only for 
\[ e^{iz} = e^{-iz} = \frac{1}{e^{iz}} \]
\[ e^{2iz} = \ [ \ e^{iz} \ ]^2 \ = 1 \]
\[ e^{iz} = \pm 1 \]
\[ e^{i(x + iy)} = \pm 1 \]
\[ e^{-y}e^{ix} = \pm 1 \]
\[ e^{-y} (\cos x + i \sin x) = \pm 1 \]
The imaginary part must be zero, so $x = 2k \pi$.

The other part must be equal to $\pm 1$, so $y = 0$ and $\cos 2k \pi = 1$, which works.

For the cosine
\[ \cos z = \frac{1}{2} (e^{iz} + e^{-iz}) \]
This is equal to zero when
\[ e^{iz} = - e^{-iz} = -\frac{1}{e^{iz}} \]
\[ e^{2iz} = -1 \]
\[ e^{i(x + iy)} = \pm i \]
\[ e^{-y} (\cos x + i \sin x) = \pm i \]
In this case we need $\cos x = 0$ and then $y=0$ and $\sin x = 1$ will work.  $x = (2k + 1)\pi / 2$.

Recall that
\[ \cos z = \cos x \cosh y - i \sin x \sinh y \]
Since $\cosh$ is never zero, $\cos x$ must be zero.  Then either $\sin x = 0$ or $\sinh y = 0$.  Only the latter works for the non-imaginary part, so we have that $y = 0$.

\subsection*{summary}

The definition:
\[ \cos z = \frac{1}{2} \ (e^{iz} + e^{-iz}) \]
\[ \sin z = \frac{1}{2i} \ (e^{iz} - e^{-iz}) \]

\[ \cosh x =  \frac{1}{2} \ (e^{x}+ e^{-x}) \]
\[ \sinh x =  \frac{1}{2} \ (e^{x} - e^{-x}) \]

A pair of identities
\[ \cos iy = \cosh y \]
\[ \sin iy =  i \sinh y \]

By the sum of angles formula, or by manipulating the exponential
\[ \cos z = \cos x \cosh y - i \sin x \sinh y \]
\[ \sin z = \sin x \cosh y + i \cos x \sinh y \]

\end{document}