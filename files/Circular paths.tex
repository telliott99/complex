\documentclass[11pt, oneside]{article} 
\usepackage{geometry}
\geometry{letterpaper} 
\usepackage{graphicx}
	
\usepackage{amssymb}
\usepackage{amsmath}
\usepackage{parskip}
\usepackage{color}
\usepackage{hyperref}

\graphicspath{{/Users/telliott/Github/figures/}}

\title{Circular paths}
\date{}

\begin{document}
\maketitle
\Large

%[my-super-duper-separator]

If the contour (curve) of integration $C$ is parametrized in terms of $t$, then
\[ \int_C f(z) \ dz = \int_a^b f[z(t)] \ z'(t) \ dt \]

A particularly important parametrization is for circular paths.  On such a path, $z$ takes on values with constant $r$ and the only change is in $\theta$.

Often we are interested in closed circular paths, where we go around the whole circle and end up where we started.

\subsection*{example:  $z$}

Suppose we have
\[ z = re^{i \theta} \]
\[ dz = r i e^{i \theta} d \theta \]

so 
\[ \int z \ dz = \int re^{i \theta} \ r i e^{i \theta} d \theta \]
\[ = r^2 i \int e^{i 2\theta}  \ d \theta \]
\[ = r^2 i \ \frac{1}{2i} e^{i 2 \theta} = \frac{1}{2} \ r^2 e^{i 2 \theta} \]
\[ = \frac{1}{2} \ z^2 \]

For any closed path, the starting and ending $z$ are the same, so the value is zero.  Alternatively, the result can be written $\rho e^{i \phi}$, but that exponential is really two trig functions, that have a period of $2 \pi$ so evaluated over any closed path, the result is zero.

For the unit circle, evaluated between $\theta = 0 \rightarrow \pi/2$ we get
\[ \frac{1}{2} \ e^{i 2 \theta} \big |_0^{\pi/2} = \frac{1}{2} ( \ e^{i \pi} - 1) \]
\[ = \frac{1}{2} (\cos \pi - i \sin \pi - 1) \]
\[ = \frac{1}{2} (-1 - 1) = - 1 \]

\subsection*{example:  $z$ in terms of $(x,y)$}

Let's repeat the calculation for $\int z \ dz$ in terms of $u,v$ and $dx,dy$:
\[ \int u \ dx - v \ dy + i \ [ \ u \ dy + v \ dx \ ] \]
Substitute from $u = x, v = y$:
\[ \int x \ dx - y \ dy + i \ [ \ x \ dy + y \ dx \ ] \]

Consider a path around the unit circle.  Then
\[ x = \cos t \]
\[ y = \sin t \]
\[ dx = -\sin t \ dt \]
\[ dy = \cos t \ dt \]

substitute in the integral
\[ \int \cos t (- \sin t \ dt) - \sin t \cos t \ dt + i \ [ \ \cos^2 t \ dt - \sin^2 t \ dt \ ] \]
\[ -2 \int \sin t \cos t \ dt + i \int  \cos^2 t - \sin^2 t \ dt \]
If this were not a unit circle, each term would have a factor of $r$, one for the $u,v$ part and one for the differential, so we would just have a leading factor of $r^2$.

The real part is $\int u \ du = u^2/2$ so
\[ I_{re} = - \sin^2 t \]

If you can't remember $\int \cos^2$ why don't we just guess:
\[ [ \ \sin t \cos t \ ]' = \cos^2 t - \sin^2 t \]
Looks pretty good!  So
\[ I_{im} = \sin t \cos t \]
Putting it together we get
\[ \int z \ dz = - \sin^2 t + i \sin t \cos t \]

Since the trig functions have a period of $2 \pi$, any path that is closed has that period, and the value of the integral will be zero.

Otherwise, in say, the first quadrant where $t = 0 \rightarrow \pi/2$, at the upper bound we have $-1$ and at the lower bound we have $0$, so the value of the integral is $-1$, which matches what we got before.

\subsection*{example:  $\int z^*$}

Consider $f(z) = z*$.

This function is of course \emph{not} analytic, because it involves $z*$ rather than $z$, and also because
\[ z^* = x - iy \]
so
\[ u_x = 1 \ne v_y = - 1 \]
The CRE do not hold.

Suppose our curve is the circle of radius $r$ centered at the origin, and we proceed between the endpoints $z = -ri \rightarrow ri$ (from due south to east to due north).  On this half-circle 
\[ z = re^{i \theta} \]
we have then
\[ dz = i \ re^{i \theta} \ d \theta \]
In radial coordinates
\[ z^* = re^{-i\theta} \]

so then
\[ \int {z^*} \ dz = \int r e^{-i\theta} r i e^{i \theta} \ d \theta \]
\[ = r^2 i \int_{-\pi/2}^{\pi/2} d \theta = r^2 \pi i \]
Alternatively,
\[ zz^* = |z|^2 = r^2  \]
\[ z^* = \frac{r^2}{z} \]
\[ \int {z^*} \ dz = r^2 \int \frac{1}{z} \ dz \]
Again
\[ z = re^{i \theta} \]
\[ dz = iz \ d \theta \]
So the integral is just
\[ = r^2 \int \frac{1}{z} \ iz \ d \theta   \]
\[ = r^2 i \int d \theta = r^2 \pi i \]
over the half-circle

\subsection*{example:  $\int z^2$}
Let $f(z) = z^2$.  

For the path, take the unit circle over the first quadrant from $(1,0)$ to $(0,1)$.  This is only a part of a circle, so the integral will be non-zero. 

There is are easy ways to do this, and there is a hard way.  

Let's start by re-checking that this function is analytic, and then do the hard way first.

Write $z$ in terms of $x$ and $y$:
\[ z = x + iy \]
\[ z^2 = (x + iy)^2 = x^2 - y^2 + i2xy \]
\[ u = (x^2 - y^2) \]
\[ v = 2xy \]
\[ u_x = 2x = v_y \]
\[ u_y = -2y = -v_x \]

The CRE hold.

Also
\[ dz = dx + i \ dy \]

So then
\[ \int z^2 \ dz = \int u \ dx - v \ dy + i \ [ \ v \ dx + u \ dy \ ] \]
\[ = \int (x^2 - y^2) \ dx - \int 2 xy \ dy + i \int 2xy \ dx + i \int (x^2-y^2) \ dy \]

As before, we must parametrize this using the relationship between $x$ and $y$ along the curve.
\[ x = \cos t \]
\[ y = \sin t \]
so
\[ x^2 - y^2 = \cos^2 t - \sin^2 t = 2 \cos^2 t - 1  \]
\[ 2xy = 2 \cos t \sin t  \]

and
\[ dx = - \sin t \ dt \]
\[ dy = \cos t \ dt \]

To keep things straight, write the integral in its four separate parts:
\[ I_1 = \int (2 \cos^2 t - 1) \  (- \sin t \ dt) \]
\[ I_2 = - \int 2 \cos t \sin t \ (\cos t \ dt) \]

leaving off the $i$
\[ I_3 = \int 2 \cos t \sin t (- \sin t \ dt) \]
\[ I_4 = \int (2 \cos^2 t - 1) \ (\cos t \ dt) \]

$I_4$ can also be written as $(\cos^2 t - \sin^2 t)(\cos t \ dt)$.

It looks pretty wild, but really, these are fairly easy integrals, since we can use $u$ substitution.  Let's make a table for reference:
\[ \int \cos^2 t \ (- \sin t \ dt) = \frac{1}{3} \cos^3 t \]
\[ \int \sin^2 t \ (\cos t \ dt) = \frac{1}{3} \sin^3 t \]
\[ \int \cos^3 t \ dt = \int (1 - \sin^2 t) \ \cos t \ dt = \sin t - \frac{1}{3} \sin^3 t \]

So now we can then just copy the results into the integrals we set up:
\[ I_1 = \frac{2}{3} \cos^3 t - \cos t \]
\[  I_2 = \frac{2}{3} \cos^3 t \]
The real part is then
\[  \frac{4}{3} \cos^3 t - \cos t \]
and
\[ I_3 = -\frac{2}{3} \sin^3 t \]
\[  I_4  = 2 \sin t - \frac{2}{3} \sin^3 t - \sin t \]
The imaginary part is then
\[ i \ [ \ -\frac{4}{3} \sin^3 t + \sin t \ ] \]
There is some symmetry.

At the upper limit, the cosine is zero and the sine is $1$ so the imaginary part is only non-zero at the upper limit and we get just $-1/3 \ i$.  

The real part is only non-zero at the lower limit, that gives $1/3$ but it's the lower limit so we subtract, and the answer is
\[ -\frac{1}{3} - \frac{1}{3} i = -\frac{1}{3} (-i - 1) \]

\subsection*{easy way}
One is to just treat $z$ as if it were a real variable
\[ \int z^2 \ dz = \frac{1}{3} \ z^3 \ \bigg |_1^i = \frac{1}{3} (-i - 1) \]
If we go all the way around the unit circle the integral is zero.

\subsection*{update}

Going back to the first example in the previous chapter we had
\[ \int z \ dz = \frac{z^2}{2} \ \bigg |_{1 + i}^{3 + 3i} \]
\[ = \frac{9 - 9 + 18i - \ [ \ 1 - 1 + 2i \ ] }{2} \]
\[ = 8i \]

\subsection*{example:  $1/z$}
\[ \int_0^{2\pi} \frac{1}{z} \ dz \]

Examining the inverse function, you might want to first confirm that it is analytic by calculating the partial derivatives.  We did this already so I'll skip it.

If we are on the unit circle, then 
\[ z = e^{i\theta} \]
\[ dz = ie^{i\theta} d \theta = iz\ d \theta \]
\[ \int \frac{dz}{z} = \int \frac{iz}{z} \ d \theta = i \int d \theta = 2 \pi i \]

Pretty and pretty easy!

The inverse is an example of a function that \emph{is} analytic, yet the integral around a closed curve that includes the origin is not equal to zero, it is instead equal to $2 \pi i$.  The reason is connected to the fact that the function is not defined at the $z=0$.

If we're centered on the origin but we don't have a unit circle, there will be an $R$ in both the numerator and the denominator, which cancel.

\emph{The result is thus independent of the radius of the circle.}

Other powers:

In general
\[ \oint_C \frac{dz}{(z - z_0)^n} = 
\begin{cases}
0, & n \ne 1 \\
2 \pi i, & n = 1 
\end {cases}
\]
We will see examples for $n \ne 1$ below.

\subsection*{inverse in terms of $x$ and $y$}

We can also integrate the inverse function in terms of $x$ and $y$:
\[ \frac{1}{z} = \frac{z*}{zz*} = \frac{x - iy}{x^2 + y^2} \]

So
\[ u = \frac{x}{x^2 + y^2} \]
\[ v =  \frac{-y}{x^2 + y^2} \]

The integral is
\[ \int u \ dx - v \ dy + i \ [ v \ dx + u \ dy \] 
\[ = \int \frac{1}{x^2+y^2} \ [ \  x \ dx + y \ dy + i \ ( \ -y \ dx + x \ dy) \ ] \]

Suppose we go on a circle of radius $R$ centered on the origin and parametrize in terms of $\theta$.  We obtain:
\[ x = R \cos \theta \]
\[ y = R \sin \theta \]
\[ x^2 + y^2 = R^2 \]
and
\[ dx = - R \sin \theta \ d \theta \]
\[ dy = R \cos \theta \ d \theta \]

We have for the integral
\[ \frac{1}{R^2} \int x \ dx + y \ dy + i \ [ \ -y \ dx + x \ dy \ ] \]

Each term $x$, $y$, $dx$ and $dy$ has a factor of $R$.  So factor that out and cancel what's in front.  Then substitute the trig functions:
\[ \int \cos \theta \ (- \sin \theta \ d \theta) + \sin \theta \ (\cos \theta \ d \theta) + i \ [ \ - \sin \theta \ (- \sin \theta \ d \theta) + \cos \theta \ (\cos \theta \ d \theta)\ ] \]

The real part has a big cancelation:
\[ \int   i \ [ \ - \sin \theta \ (- \sin \theta \ d \theta) + \cos \theta \ (\cos \theta \ d \theta)\ ]\]
and so does the imaginary part:
\[ = \int i \ d \theta = 2 \pi i \]

Note that if we integrate the same function around a unit square, we run into problems.  First let's  do $[0,0 \times 1,1]$.  We have
\[ \int u \ dx - \int v \ dy + i \ [ \ \int v \ dx + \int u \ dy \ ]  \]
Along $C1$, $y = 0$ and $dy = 0$ so:
\[ \int \frac{x}{x^2 + y^2} \ dx + i \ [ \ \int \frac{-y}{x^2 + y^2} \ dx \]
\[ = \int_0^1 \frac{1}{x} \ dx = \ln x \ \bigg |_0^1 \]
Since $\ln 0$ is not defined, we can't do this.

Logarithms are tricky, no doubt.  

If the complex logarithm $Log$ $z$ is defined and differentiable along the curve (and it is, along the semicircle from $-i$ to $i$), we can do this:
\[ I = \int_{-i}^i \frac{1}{z} \ dz = \text{ Log } z  \]

To evaluate this, recall that $z = e^{i\theta}$ recognize that at $z = -i$, $\theta = - \pi/2$, while for $z = i$, $\theta = \pi/2$:
\[ = i \theta \ \bigg |_{-\pi/2}^{\pi/2} =  i \ \frac{\pi}{2}  -  i \frac{-\pi}{2}  = 2i \ \frac{\pi}{2} = \pi i \]

If we're on the unit circle, then we need not worry about the real part of Log $z = \ln r + i \theta$, since $\ln 1$ is equal to zero.  Also, if we're centered on the origin, then $r$ is a constant and we get the same number at both bounds, so the difference is zero even if the path is not a full circle.

\subsection*{powers}
We can extend this to 
\[ \int \frac{1}{z^2} \ dz \]
As before
\[ dz = i z \ d \theta \]
so this is
\[ \int \frac{1}{z^2} \ i z \ d \theta = i \int \frac{1}{z} \ d \theta \]

On the unit circle
\[ z = e^{i\theta} \]
The integral is
\[  i \int_0^{2 \pi} e^{-i\theta} \ d \theta \]
Now from Euler
\[ e^{i \theta} = \cos \theta + i \sin \theta \]
For any closed path, the difference between the bounds is $2\pi$ which is also the period of these functions, so the difference is always zero.

In fact, for any negative integer power of $z$
\[ \int z^{-n} \ dz \]

around the unit circle $z=e^{i\theta}$ we have
\[ i \int e^{-i(n-1)\theta} \ d \theta \]
\[ = -\frac{1}{n-1} \ e^{-i(n-1)\theta} \ \bigg |_0^{2 \pi}  \]
It doesn't matter what $n-1$ is, the value of the integral is just zero.

\subsection*{example: square root}
Consider
\[ \int \sqrt{z} \ dz \]
along the half-circle of radius $3$ starting from the point $z = R$ on the $x$-axis and proceeding counter-clockwise.
We can do this integral even if the "branch" of the square root function that we're using is only defined for $\theta > 0$.  

We have that 
\[ z = Re^{i\theta}, \ \ \ \theta = 0 \rightarrow \pi \]
\[ dz = iz \ d \theta \]
\[ = iRe^{i\theta} \ d \theta \]

\[ \sqrt{z} = \sqrt{R} e^{i\theta/2} \]
so
\[ I = \int_0^{\pi} iR \sqrt{R} e^{i3\theta/2} \ d \theta \]

We need
\[ \int e^{i3\theta/2} \ d \theta = \frac{2}{3i} e^{i3\theta/2} \ \bigg |_0^{\pi} \]
easiest to write it out as
\[ e^{i3\theta/2} \ \bigg |_0^{\pi} = \cos \frac{3\pi}{2} + i \sin  \frac{3\pi}{2} - \cos 0 - i \sin 0 \]
\[ = 0 + i(-1) - 1 - 0 = -(1+i) \]

Going back to pick up all the factors we left behind:
\[ I = -iR \sqrt{R} \ \frac{2}{3i} \ (1+i) = -R \sqrt{R} \ \frac{2}{3} \ (1+i) \]
In the problem, $R$ was actually specified as $3$, leading to the cancellation:
\[ I = - 2 \sqrt{3} \ (1+i) \]

We can also do this problem by antiderivatives:
\[ \int_R^{-R} \sqrt{z} \ dz = \frac{2}{3} \ z^{3/2} \ \bigg |_R^{-R}  \]
\[ = \frac{2}{3} ( R^{3/2} e^{i3\pi/2} - R^{3/2} e^0) \]
\[ = \frac{2}{3} R^{3/2} ( e^{i3\pi/2} - 1) \]
and, as we showed above:
\[ e^{i3\pi/2} = -i \]
If $R=3$ we get the same answer as before.

\subsection*{summary}

For an analytic function, we can compute the integral by analogy with the real numbers:  $\int z \ dz = z^2/2$.

For any closed path, the result is just zero, with some special exceptions.

$z^*$ is special because it's not analytic.  $1/z$ is said to be special because it's not defined at $z = 0$, though it is analytic, but this begs the question, what about $1/z^2$?

I would rather say that $1/z$ is special because it is $z^*$ in disguise since:
\[ \frac{1}{z} \ \frac{z^*}{z^*} = \frac{1}{r^2} \ z^* \]
with $r$ just a constant.

Another reason is that (on the unit circle) we have
\[ z = e^{i \theta} \]
so
\[ dz = i z \ d \theta \]
and
\[ \int \frac{1}{z} \ dz = \int i \ d \theta \]
The $z$ cancels.

For any other power of $z$ we will have some factor of $e^{ki\theta}$ at the end.  Since this is a combination of sine and cosine, it will give zero when integrated over a closed path since $\theta = 0 \rightarrow 2 \pi$ and the trig functions have a period of $2 \pi$ no matter where you pick for the starting point (no matter what the value of $k$ is).



\end{document}