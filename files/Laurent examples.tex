\documentclass[11pt, oneside]{article} 
\usepackage{geometry}
\geometry{letterpaper} 
\usepackage{graphicx}
	
\usepackage{amssymb}
\usepackage{amsmath}
\usepackage{parskip}
\usepackage{color}
\usepackage{hyperref}

\graphicspath{{/Users/telliott/Github/figures/}}

\title{Writing Laurent Series}
\date{}

\begin{document}
\maketitle
\Large

%[my-super-duper-separator]

Laurent series are very helpful in certain problems.  However, they can really complicate your life.  Writing a series is not too difficult, with practice.  Look at the cheatsheet, and the short chapter on change of variables.

However, what you \emph{must remember} with Laurent series is where is the center, where are the singularities, and where is the contour?.  

Depending on these things, the series that is required will change.  You must match the series with the region.  The next example is shows that clearly.

\subsection*{example 16 (Boas)}

\label{sec:ex16L}

\[ f(z) = \frac{12}{z(2 - z)(1 + z)} \]

This function has three isolated singularities (at $z = 0, 2, -1$).

Expanded around $z_0 = 0$, there will be three regions in which we have \emph{different} series:  namely $0 < |z| < 1$, $1 < |z| < 2$ and $|z| > 2$.

\begin{center} \includegraphics [scale=0.5] {Boas_14_4_1b.png} \end{center}

For each region, there will usually be two series adding up to the complete Laurent series.

Also, in this problem, we have two integrands, obtained by partial fractions.
\[ = \frac{4}{z} \cdot (\frac{1}{1 + z} + \frac{1}{2 - z}) \]
So it's even more complicated!

\subsection*{reference series}
$A$
\[ \frac{1}{1-z} = 1 + z + z^2 + z^3 + z^4 \dots \]
$P$
\[ \frac{1}{1 - z} = - \frac{1}{z} - \frac{1}{z^2} - \frac{1}{z^3}  - \frac{1}{z^4} \dots \]

\subsection*{first term}

\[ \frac{1}{1 + z} \]
We need to substitute $-z$ for $z$ everywhere:

$A$
\[ \frac{1}{1 + z} = 1 - z + z^2 - z^3 + z^4 \dots \]
$P$ 
\[ \frac{1}{1 + z} = \frac{1}{z} - \frac{1}{z^2} + \frac{1}{z^3}  - \frac{1}{z^4} \dots \]

\subsection*{second term}

\[ \frac{1}{2 - z} \]
The sign is correct but we have to factor out $2$ on the bottom and remember it, and then have standard series in terms of $z/2$ and $2/z$:

$A$
\[ \frac{1}{1-z/2} = 1 + z/2 + (z/2)^2 + (z/2)^3 + (z/2)^4 \dots \]
with the factor of $1/2$
\[ = \frac{1}{2} + \frac{z}{4} + \frac{z^2}{8} + \frac{z^3}{16} \dots \]

$P$
\[ \frac{1}{1 - z/2} = - \frac{2}{z} - \frac{4}{z^2} - \frac{8}{z^3}  - \frac{16}{z^4} \dots \]
with the factor of $1/2$
\[ = - \frac{1}{z} - \frac{2}{z^2} - \frac{4}{z^3}  - \frac{8}{z^4} \dots \]

\subsection*{inner disk}

Start with the inner punctured disk.  We want convergence for the region $|z| < 1$.  Since it's less than, we use standard geometric series (analytic) for each of the two terms.

\[  1 - z + z^2 - z^3 + z^4 \dots \]
\[ \frac{1}{2} + \frac{z}{4} + \frac{z^2}{8} + \frac{z^3}{16} + \frac{z^4}{32} \dots \]

Add together
\[ = \frac{3}{2} - \frac{3}{4}z + \frac{9}{8}z^2  - \frac{15}{16}z^3 + \frac{33}{32}z^4 \dots \]

and finally, multiply by $4/z$ to obtain:
\[ = \frac{6}{z} - 3 + \frac{9z}{2} - \frac{15z^2}{4} \dots \]

This is the Laurent series valid in the innermost region.

\subsection*{outer region}

For the outer region, we need the principal part of the Laurent series only.  The first one is good as it is.
\[ \frac{1}{z} - \frac{1}{z^2} + \frac{1}{z^3}  - \frac{1}{z^4} \dots \]

For the second, we have the rearranged version:
\[ - \frac{1}{z} - \frac{2}{z^2} - \frac{4}{z^3}  - \frac{8}{z^4} \dots \]

Add together
\[ -\frac{3}{z^2} - \frac{3}{z^3} - \frac{9}{z^4} \dots \]

Recall the leading factor of $4/z$
\[ - \frac{12}{z^3} - \frac{12}{z^4} - \frac{36}{z^5} \dots )  \]
\[ = \frac{12}{z^3} \ (-1 -\frac{1}{z} - \frac{3}{z^2} \dots )\]

\subsection*{middle region}

The third part is the annulus in the middle.  For this we want convergence for $|z| > 1$ and also for $|z| < 2$.  Hence we want the principal part for $1/(1+z)$ and the analytic part for $1/(2-z)$.

\[ \frac{1}{z} - \frac{1}{z^2} + \frac{1}{z^3}  - \frac{1}{z^4} \dots \]

\[ \frac{1}{2} + \frac{z}{4} + \frac{z^2}{8}  + \frac{z^3}{16} + \frac{z^4}{32} \dots \]

We could add them together, but it isn't really necessary since there are no shared powers.

Finally, there is the leading factor of $4/z$, which gives $2$ as the cofactor of $z^{-1}$.  Each comes from term 1 of an analytic series.

\subsection*{summary}

What these results mean is that if we evaluate a contour integral in one of these regions, we should get the answer (within a factor of $2 \pi i$) given by the cofactor of $z^{-1}$ by series convergent for that region.

For example, in the inner disk, I get $6/z$ from the series above.  Looking ahead to residue theory, \emph{in that region} there is only a single pole (at $z = 0$) and the value of the residue is
\[ 12 \cdot \frac{1}{(1+z)(2-z)} \bigg |_{z=0} = 6 \]
This matches the cofactor of $z^{-1}$ there, which was $6$ for the series in the inner ring.

In the middle ring, we should get the same $6$ from above \emph{plus}
\[ 12 \cdot \frac{1}{z(2-z)} \bigg |_{z=-1} = - 4 \]
that's a total of $6 - 4 = 2$, which matches the series.

In the outer ring, we get for the newly added one:
\[ 12 \cdot \frac{1}{z(1+z)} \bigg |_{z=2} \]
\[ 12 \cdot \frac{1}{6} = 2 \]
That's a grand total of $6 - 4 + 2 = 4$.  Possibly we're missing a $1$ in the series which would give $4/z$ when multiplied by the leading factor, but I can't find the mistake.

\subsection*{example 3}

\label{sec:ex3L}

\[ f(z) = \frac{1}{z(z+2)} \]
Suppose the region of interest is an annulus centered on $z = 1$ with $1 < |z-1| < 3$.
\begin{center} \includegraphics [scale=0.5] {writeseries1.png} \end{center}

The first thing to do is make a substitution that translates the region so that it becomes centered on the origin:  $w = z - 1$.  Then the function becomes
\[  \frac{1}{(w + 1)(w + 3)} \]
The next thing is to write partial fractions.  For the numerator we get
\[ A(w+3) + B(w+1) = 1 \]
\[ A = - B = \frac{1}{2} \]

Hence
\[ \frac{1}{2} \cdot \ [ \ \frac{1}{w+1} - \frac{1}{w+3} \ ] \]
The third step is to convert each of these fractions into something like $1/1-x$.
\[ \frac{1}{w+1} = \frac{1}{1 - (-w)} \]
\[ \frac{1}{w+3}  = \frac{1}{3} \cdot \frac{1}{1 - (-w/3)} \]
And then the fourth step is to write the series, recalling that we want different forms depending on whether we are in a circle or an annulus.

\[ \frac{1}{1 - (-w)} \]
\[ = \sum_{n=0}^{\infty} (-w)^n = \sum_{n=0}^{\infty} (-1)^n (w)^n , \ \ \ |w| < 1 \]
\[ = -\sum_{n=1}^{\infty} \frac{1}{(-w)^n} =  -\sum_{n=1}^{\infty} \frac{(-1)^n}{w^n}, \ \ \ |w| > 1 \]
We pick the second form because our region is $1 < |z-1| < 3$

A similar thing can be done for the other term.  We show only the first series since we are inside the circle.
\[  \frac{1}{3} \cdot \frac{1}{1 - (-w/3)} \]
\[ = \frac{1}{3} \cdot  \sum_{n=0}^{\infty} (-1)^n (\frac{w}{3})^n , \ \ \ |w| < 3 \]
\[ = \sum_{n=0}^{\infty} (-1)^n \ \frac{1}{3^{n+1}} \ w^n , \ \ \ |w| < 3 \]

Add the two series together (remembering the minus sign on the second term)
\[ -\sum_{n=1}^{\infty} \frac{(-1)^n}{w^n} - \sum_{n=0}^{\infty} (-1)^n \ \frac{1}{3^{n+1}} \ w^n \]
and then picking up the leading factor from 
\[ \frac{1}{2} \cdot \ [ \ \frac{1}{w+1} - \frac{1}{w+3} \ ] \]
so
\[ \frac{1}{2} \ [ \ -\sum_{n=1}^{\infty} \frac{(-1)^n}{w^n} - \sum_{n=0}^{\infty} (-1)^n \ \frac{1}{3^{n+1}} \ w^n \ ] \]

The last step is to reverse the substitution:  $w = z - 1$ and bring the minus sign out front
\[ f(z) = - \frac{1}{2} \ [ \ \sum_{n=1}^{\infty} \frac{(-1)^n}{(z-1)^n} \sum_{n=0}^{\infty} (-1)^n \ \frac{1}{3^{n+1}} \ (z-1)^n \ ] \]

I don't know if I could ever learn to do this well, but at least the explanations make sense.

Now, if we were to integrate $f(z)$, we would have only one term that gives a non-zero result, namely the first term with $n=1$
\[ - \frac{1}{2} (-1) \frac{1}{z-1} \]
The residue is the cofactor of that term:
\[ \text{Res }(1) = \lim_{z \rightarrow 1} \frac{1}{2} = \frac{1}{2} \]
Multiply by $2 \pi i$ to obtain $\pi i$.


\end{document}