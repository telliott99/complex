\documentclass[11pt, oneside]{article} 
\usepackage{geometry}
\geometry{letterpaper} 
\usepackage{graphicx}
	
\usepackage{amssymb}
\usepackage{amsmath}
\usepackage{parskip}
\usepackage{color}
\usepackage{hyperref}

\graphicspath{{/Users/telliott/Github/figures/}}

\title{Summary of residues}
\date{}

\begin{document}
\maketitle
\Large

%[my-super-duper-separator]

Cauchy's residue formula was
\[ \oint \frac{f(z)}{z - z_0} \ dz = 2 \pi i \cdot f(z_0) \]

Two corollaries:
\[ \oint \frac{f(z)}{(z - w)^2} \ dz = 2 \pi i \cdot f'(z_0) \]

generally
\[ \oint \frac{f(z)}{(z - z_0)^{n+1}} \ dz = \frac{1}{n!} \ 2 \pi i \cdot f^{n}(z_0) \]

The residue of $f(z)$ at a singularity $z_0$ is the coefficient of the $(z - z_0)^{-1}$ term in the Laurent series for $f(z)$, if we can write it.

It is also given by
\[ R = \lim_{z \rightarrow z_0} (z - z_0) \cdot f(z_0) \]
\[ = \frac{1}{2 \pi i} \oint f(z) \ dz \]


\end{document}