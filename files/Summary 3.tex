\documentclass[11pt, oneside]{article} 
\usepackage{geometry}
\geometry{letterpaper} 
\usepackage{graphicx}
	
\usepackage{amssymb}
\usepackage{amsmath}
\usepackage{parskip}
\usepackage{color}
\usepackage{hyperref}

\graphicspath{{/Users/telliott/Github/figures/}}

\title{Summary of residues}
\date{}

\begin{document}
\maketitle
\Large

%[my-super-duper-separator]

Cauchy's residue formula was
\[ \oint \frac{f(z)}{z - z_0} \ dz = 2 \pi i \cdot f(z_0) \]

Two corollaries:
\[ \oint \frac{f(z)}{(z - w)^2} \ dz = 2 \pi i \cdot f'(z_0) \]

generally
\[ \oint \frac{f(z)}{(z - z_0)^{n+1}} \ dz = \frac{1}{n!} \ 2 \pi i \cdot f^{n}(z_0) \]

The residue of $f(z)$ at a singularity $z_0$ is the coefficient of the $(z - z_0)^{-1}$ term in the Laurent series for $f(z)$, if we can write it.

It is also given by
\[ R = \lim_{z \rightarrow z_0} (z - z_0) \cdot f(z_0) \]
\[ = \frac{1}{2 \pi i} \oint f(z) \ dz \]

\subsection*{example:  $1/1-z$}

I can get wrapped up in knots thinking about this one.

This is $-1/(z-z_0)$, with $z_0 = 1$.  Residue theorem says to multiply by $z - z_0$ and evaluate the function, i.e. $-1$, at the point $z_0$.  I get $I = -2 \pi i$.

Or:  substitute $w = 1-z$, then $dw = -dz$ so this is $- \int 1/w \ dw = -2 \pi i$.  

Or:  write the Laurent series
\[ 1 + z + z^2 \dots + (-\frac{1}{z})(1 + \frac{1}{z} + ( \frac{1}{z})^2 \dots \]
The $1/z$ term has coefficient $b_1 = -1$.  So it's $2 \pi i$ times $b_1$.

The problem is to think about convergence.  Where is my $z_0$ and where is my contour...?  Luckily, we have the other answers.

\subsection*{example:  $1/1+z$}

This is $1/(z - z_0)$ with $z_0 = -1$.  Residue theory says to multiply by $z - z_0$ and obtain $1$, evaluated at the point $z_0$ just gives $1$.  So I  get $2 \pi i$.




\end{document}