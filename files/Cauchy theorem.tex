\documentclass[11pt, oneside]{article} 
\usepackage{geometry}
\geometry{letterpaper} 
\usepackage{graphicx}
	
\usepackage{amssymb}
\usepackage{amsmath}
\usepackage{parskip}
\usepackage{color}
\usepackage{hyperref}

\graphicspath{{/Users/telliott/Github/figures/}}
% \begin{center} \includegraphics [scale=0.4] {gauss3.png} \end{center}

\title{Cauchy's Integral Theorem}
\date{}

\begin{document}
\maketitle
\Large

%[my-super-duper-separator]

Cauchy's theorem says that the integral of an analytic function over a closed path is equal to zero:
\[ \oint_C f(z) \ dz = 0 \]
There is an important restriction:  the enclosed region must not contain a singularity.

This will turn out to be a consequence of Green's Theorem, which you may remember from multivariable calculus.  

Let
\[ z = x + i y \]
\[ dz = dx + i dy \]
\[ z = f(x,y) = u(x,y) + iv(x,y) \]

Our integral is
\[ =  \oint u \ dx - v \ dy + i v \ dx + i u \ dy \]

\subsection*{proof of Cauchy's theorem}
Back in vector calculus we proved Green's theorem, which says that for two real functions of $x$ and $y$:  $M(x,y)$ and $N(x,y)$:
\[ \oint_C M dx + N dy = \iint_R (N_x - M_y) \ dx \ dy \]

Back then, $M$ and $N$ were components of a vector field $\mathbf{F}$ and we wrote the shorthand for curl:
\[ = \iint_R \nabla \times \mathbf{F} \ dA\]

but the important thing is that the theorem applies to real-valued functions of two real variables $f: \mathbb{R}^2 \rightarrow \mathbb{R}^1$, and so it applies to functions like $u(x,y)$ and $v(x,y)$.

Consider the real part of the integral above:
\[ I_{Re} = \oint u \ dx - v \ dy \]

Let $M = u$ and let $N = - v$ (notice the minus sign!).  Then
\[ I_{re} = \oint M \ dx + N \ dy \]

This is equal to a double integral containing:
\[ N_x - M_y \]

We have that $N = -v$ so $N_x = - v_x$.  And this is equal to $u_y$ by the CRE.

But $M = u$ so $M_y = u_y$. The two terms in the subtraction are equal so the result is zero.  Hence, this integral is zero.

For the imaginary part 
\[ I_{Im} = i \oint v \ dx + u \ dy \]

Let $N = u$ and let $M = v$ (no minus sign!).  Then
\[ I_{re} = \oint M \ dx + N \ dy \]

This is equal to a double integral containing:
\[ N_x - M_y \]

But $N_x = u_x$ and $M_y = v_y$ and these terms are equal by the CRE.  Therefore this expression is zero.

So the integral for the imaginary part is also zero, and thus the whole thing is zero as well:
\[ \oint u \ dx - v \ dy + i \ [ \  v \ dx + u \ dy \ ] \ = 0 \]

Remember how important it was (for Green's theorem) that the function being integrated be defined everywhere in the region.  Well, it's true here as well.

\[ \oint_C \frac{1}{z} \ dz \stackrel{?}{=} , \ \ \ \ \ \ \ne 0 \]

We've already seen by direct calculation that this integral is \emph{not} zero when the curve $C$ includes the origin, although it zero otherwise.

To repeat the demonstration for the former case we use the unit circle centered at the origin.  Write
\[ z = r e^{i\theta} \]
\[  \frac{dz}{d\theta} = r i e^{i\theta} = iz \]
Hence
\[ \oint_C \frac{1}{z} \ dz = \oint_C \frac{1}{z} \ iz \ d \theta \]
\[ = i   \oint_C  d \theta = 2 \pi i \]

\subsection*{Path independence}
The theorem that says the integral of an analytic function over a closed path (over a region without a singularity), is equal to zero.
\[ \oint_C f(z) \ dz = 0 \]

This result means, in turn, that the integral of an analytic function between two points $z_1$ and $z_2$ is independent of the path taken.  Call the two paths $C1$ and $C2$.  

Form the closed path by going from $z_1$ to $z_2$ over $C1$ and then return to $z_1$ by going backward over $C2$.  The total integral is equal to zero by Cauchy's Theorem.
\[ \int_{C1} f(z) \ dz + \int_{-C2} f(z) \ dz = 0 \]
But the integral over the path $C_2$ in the forward direction is just minus the integral over the reverse path $-C2$.
\[ \int_{-C2} f(z) \ dz = - \int_{C2} f(z) \ dz \]
Thus
\[ \int_{C1} f(z) \ dz - \int_{C2} f(z) \ dz = 0 \]
and
\[ \int_{C1} f(z) \ dz = \int_{C2} f(z) \ dz \]

\end{document}