\documentclass[11pt, oneside]{article} 
\usepackage{geometry}
\geometry{letterpaper} 
\usepackage{graphicx}
	
\usepackage{amssymb}
\usepackage{amsmath}
\usepackage{parskip}
\usepackage{color}
\usepackage{hyperref}

\graphicspath{{/Users/telliott/Github/figures/}}
% \begin{center} \includegraphics [scale=0.4] {gauss3.png} \end{center}

\title{Differentiate}
\date{}

\begin{document}
\maketitle
\Large

This section contains a general discussion of differentiation of complex functions.  A complex function takes as input a complex number, and emits as output another complex number.  Often $w$ is used for the output:
\[ w = f(z) \]

More concretely, a complex number is simply an ordered pair of real numbers, and a complex function is a pair of functions defined on the real numbers:

\[ f(z) = f(x + iy) = u(x,y) + i \cdot v(x,y) \]

The functions $u$ and $v$ each take a pair of real numbers and emit one real number.  They are connected in $f$ through the fact that $u$ and $v$ have the same input.

Finally, the output of $v$ is multiplied by $i$.  Here is an example:
\[ z = x + iy \]
\[ z^2 = (x + iy)(x + iy) \]
\[ = x^2 - y^2 + 2ixy \]
So
\[ u(x,y) = x^2 - y^2 \]
\[ v(x,y) = 2xy \]

Another one is:
\[ \frac{1}{z} = \frac{1}{z} \cdot \frac{z*}{z*} \]
\[ = \frac{x - iy}{(x + iy)(x - iy)} \]
\[ = \frac{x}{x^2 + y^2} + i \frac{-y}{x^2 + y^2} \]

Often a simplified notation is employed:

\[ f(z) = u + i \cdot v \]

Since the inputs cover the entire complex plane, we cannot plot graphs as with real functions.  Instead, one version of the complex plane is \emph{mapped} by the function into a different version of the complex plane.

\subsection*{Cauchy-Riemann}

This chapter gives us a first glimpse of the important Cauchy-Riemann conditions and justifies one of the formulas for calculating the derivative

\[ f'(z) = u_x + i v_x \]

As an example of its use, consider the complex exponential
\[ f(z) = e^z \]
If we write $z = x + iy$ then
\[ f(z) = e^{x + iy} \]
\[ = e^x e^{iy} \]
and (from Euler):
\[ e^{iy} = \cos y + i \sin y \]
so
\[ f(z) = e^x \cos y + i e^x \sin y \]

Using the formula, it can be shown easily that the derivative is the same as the function itself, just as for the case of real numbers.
\[ u(x,y) = e^x \cos y \]
\[ u_x = e^x \cos y = u \]
\[ v(x,y) = e^x \sin y \]
\[ v_x = e^x \sin y = v \]
Hence
\[f'(z) = u_x + iv_x = z \]

\subsection*{definition}
We define the derivative $f'(z)$ of a complex function $f(z)$ similarly to the derivative of a real function:
\[ f'(z) = \lim_{w \rightarrow z} \ \frac{f(w) - f(z)}{w-z} \]
if the limit exists.

Alternatively, with $\Delta$ notation, we might write:
\[ f'(z) = \lim_{\Delta z \rightarrow 0} \ \frac{f(z + \Delta z) - f(z)}{\Delta z} \]

A crucial difference from real functions is that there are only two directions from which to approach a given real number $x$, while there is an infinite number of ways of approaching $z$ in the Argand plane.  The limit is over \emph{all} possible ways of approaching $z$.  

If the limit exists, the function $f$ is called differentiable and $f'(z)$ is the derivative.
Consider
\[ f(z) = u(x,y) + i v(x,y) \]
Then
\[ f'(z) = \frac{f(z + \Delta z) - f(z)}{\Delta z} \]
\[ = \frac{u(x + \Delta x, y + \Delta y) + i v (x + \Delta x, y + \Delta y) - u(x,y) - i v(x,y)}{\Delta x + i \Delta y} \]

\subsection*{fixed y}
We tame this beast by looking at two specific paths.

Looking at the special path along the $x$-axis where $\Delta y = 0$ we obtain
\[ f'(z) = \frac{u(x + \Delta x, y) + i v (x + \Delta x, y) - u(x,y) - i v(x,y)}{\Delta x} \]
Rearrange the numerator
\[ = \frac{u(x + \Delta x, y) - u(x,y)}{\Delta x} + \frac{i v (x + \Delta x, y) - i v(x,y)}{\Delta x} \]
The first term is
\[ u_x = \frac{\partial u}{\partial x} \]
and the second term is 
\[ i v_x \]
Hence we conclude that
\[ f'(z) = u_x + i v_x \]

\subsection*{fixed x}
Now look at the special path along the $y$-axis where $\Delta x = 0$:
\[ f'(z) = \frac{u(x, y + \Delta y) + i v (x, y + \Delta y) - u(x,y) - i v(x,y)}{i \Delta y} \]
Rearrange the numerator
\[ = \frac{u(x, y + \Delta y) - u(x,y)}{i \Delta y} + \frac{i v (x, y + \Delta y) - i v(x,y)}{i \Delta y} \]
\[ = \frac{1}{i} u_y + v_y \]
Recall that $1/i = -i$
\[ f'(z) = v_y - i u_y \]

\subsection*{Putting it together}
We require that the limit be the same regardless of the direction of approach to $z$, so these two expressions for the difference quotient must be equal:
\[ f'(z) = u_x + i v_x = - i u_y + v_y \]

Both the real and the imaginary parts must be equal so
\[ u_x  = v_y \]
\[ u_y = - v_x \]

Once differentiability is established, we can use whichever path we want to evaluate the derivative.

As we said at the beginning, in looking at various complex functions we can use this fact:
\[ f'(z) = u_x + i v_x \]

One consequence is that
\[ \frac{df}{dz} = \frac{\partial f}{\partial x} \]
and since
\[ = u_x + i v_x = v_y - i u_y \]
\[ = - i u_y + v_y \]
\[ = -i (u_y + i v_y) \]
\[ = -i \frac{\partial f}{\partial y} \]

We conclude that 
\[ \frac{df}{dz} = \frac{\partial f}{\partial x} = -i \frac{\partial f}{\partial y} \]

\subsection*{looking ahead}
When we get to integration in a later section we will find that the integral of a complex function is computed as a line integral along a specified curve (often a circle centered either on the origin or on a point $z_0$).

This curve relates the values of $x$ and $y$ and allows us to parametrize either $y$ in terms of $x$ or more generally, both $x$ and $y$ in terms of a single real variable or parameter $t$.

When we have a function of such a variable like
\[ f(t) = u(t) + i v(t) \]
then the derivative is defined to be
\[ f'(t) = u'(t) + i v'(t) \]
where $u$ and $v$ are real-valued functions of a single real variable and so follow the standard rules from introductory calculus.  In particular if
\[ w(t) = z_0 f(t) \]
then
\[ w'(t) = z_0 f'(t) \]
The derivative of a constant times a function is the constant times the derivative of the function.

\subsection*{derivative of $z_0$ times a function}

We can show this by using a little algebra:
\[ \frac{d}{dt} \ z_0 f(t) = \ [ \ (x_0 + i y_0) (u + iv) \ ]' \]
\[ = \ [ \ (x_0 u - y_0 v) + i (y_0 u + x_0 v) \ ]' \]
\[ = (x_0 u - y_0 v)' + i (y_0 u + x_0 v)' \]
\[ = (x_0 u' - y_0 v') + i (y_0 u' + x_0 v') \]
\[ = (x_0 + i y_0)(u' + iv') \]
\[ = z_0 \frac{d}{dt} \ f(t) \]
Thus
\[ \frac{d}{dt} \ z_0 f(t) = z_0 \frac{d}{dt} \ f(t) \]
which is what we just said.

\subsection*{derivative of exp $z_0$ t }

Another expected result is
\[ \frac{d}{dt} \ e^{z_0 t} = z_0 e^{z_0 t} \]
where $z_0$ is a complex constant and $t$ is a real variable.

To do this one, refer to the definition
\[ f'(t) = u'(t) + i v'(t) \]

And now we need to break up the exponential into its real and imaginary parts.  

By Euler's equation, we wrote above
\[ e^z = e^{x + iy} = e^x \cos y + i e^x \sin y \]

For the exponential of a real variable, but containing a complex constant we have
\[ e^{z_0 t} = e^{(x_0 + iy_0) t} \]
\[ = e^{x_0t} \ e^{i y_0 t} \]
\[ = e^{x_0t} \ (\cos y_0 t + i \sin y_0 t) \]
\[ = e^{x_0t} \cos y_0 t + i e^{x_0t} \ \sin y_0 t \]

\subsection*{Substitution}

I find this calculation very confusing.  Especially the subscripts.  Rather than change letters, we will drop the subscripts on $x_0$ and $y_0$ but tell ourselves repeatedly:  these are constants.  Also, $t$ is a \emph{real} variable.

\[ e^{xt} \cos y t + i e^{xt} \ \sin yt \]

Using the definition above we get that the derivative is $u'(t) + i v'(t)$ so the derivative of a sum is the sum of the derivatives.

The first term ($u'$) is (by the product rule):
\[ \ [ \ e^{xt} \cos y t \ ]' = x e^{xt} \cos yt  - y e^{xt} \sin yt \]
and the second:
\[ \ [ \ e^{xt} \ \sin yt \ ]' =  x e^{xt} \sin yt  + y e^{xt} \cos yt \]
Remember that each term in that second one gets an $i$!
\[ \ i[ \ e^{xt} \ \sin yt \ ]' =  ix e^{xt} \sin yt  + iy e^{xt} \cos yt \]

Combine the first term from each and factor out the $x$:
\[ x(e^{xt} \cos yt +  ie^{xt} \sin yt) \]
Do the same with the second term:
\[ y(ie^{xt} \cos yt - e^{xt} \sin yt) \]
the tricky part
\[ = iy(e^{xt} \cos yt + ie^{xt} \sin yt) \]

Putting everything together we have just
\[ (x + iy)(e^{xt} \cos yt + ie^{xt} \sin yt) \]

Restoring the original naughts, we have just
\[ z_0 e^{z_0 t} \]
As promised.

$\square$

\end{document}