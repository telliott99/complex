\documentclass[11pt, oneside]{article} 
\usepackage{geometry}
\geometry{letterpaper} 
\usepackage{graphicx}
	
\usepackage{amssymb}
\usepackage{amsmath}
\usepackage{parskip}
\usepackage{color}
\usepackage{hyperref}

\graphicspath{{/Users/telliott/Github/figures/}}

\title{Introduction}
\date{}

\begin{document}
\maketitle
\Large

%[my-super-duper-separator]

\subsection*{geometric series}

The initial examples are all based on the geometric series:
\[ a + ar + ar^2 + ar^3 \dots \]
The ratio test shows that this infinite series is convergent for $|r| < 1$.  In that case it has a sum and we can write:
\[ S = a(1 + r + r^2 + r^3 \dots \]
\[ \frac{S}{a} r - \frac{S}{a} = 1 \]
\[ S = \frac{a}{1-r} \]
So any time you see this form think geometric series.
\[ \frac{1}{1 - r} \]
Switch to $z$ as the variable:
\[ \frac{1}{1 - z} \]

There is a trick to getting a \emph{different} geometric series.  Just write
\[ \frac{1}{1 - z} = (-\frac{1}{z}) \cdot \frac{1}{1 - 1/z} \]
That's also a geometric series, but in $1/z$.  It converges when $|1/z| < 1$ which means $|z| > 1$!!  

And it is equal to
\[ (-\frac{1}{z}) \cdot (1 + \frac{1}{z} + \frac{1}{z^2} + \frac{1}{z^3} \dots \]
\[ - \frac{1}{z} - \frac{1}{z^2} - \frac{1}{z^3} \dots \]

Note that we can substitute $-w = z$ and get
\[ \frac{1}{1+w} =  \frac{1}{w} \ [ \ 1 - \frac{1}{w} + \frac{1}{w^2} - \frac{1}{w^3} + \dots \ ] \]
\[ \frac{1}{1+w} =  \frac{1}{w} - \frac{1}{w^2} + \frac{1}{w^3} + \dots \]

go back to $z$ as the variable
\[ \frac{1}{1+z} =  \frac{1}{z} - \frac{1}{z^2} + \frac{1}{z^3} - \frac{1}{z^4} + \dots \]
Check by multiplying the right-hand side by $z$ and see all the cancellations after the first term.

\subsection*{Boas example}
\[ f(z) = \frac{12}{z(2 - z)(1 + z)} \]

So this function has three isolated singularities (at $z = 0, 2, -1$). And expanded around $z_0 = 0$, there will be three regions in which we have \emph{different} series:  namely $0 < |z| < 1$, $1 < |z| < 2$ and $|z| > 2$.  There will be three series all together.

Start by using partial fractions to obtain:
\[ = \frac{4}{z} \cdot (\frac{1}{2 - z} + \frac{1}{1 + z}) \]

Start with the inner punctured disk.  We want convergence for the region $|z| < 1$.  Since it's less than, we use standard geometric series.

\[ \frac{1}{1 + z} = 1 - z + z^2 - z^3 \dots \]
\[ \frac{1}{2 - z} = \frac{1}{2} \cdot \frac{1}{1 - z/2} = \frac{1}{2} \ [ \ 1 + \frac{z}{2} + \frac{z^2}{4} + \frac{z^3}{8} \dots \ ] \]
Add them together 
\[ = \frac{3}{2} - \frac{3z}{4} +  \frac{9z^2}{8} - \frac{15z^3}{16} \dots \]
and multiply by $4/z$ to obtain:
\[ = \frac{6}{z} - 3 + \frac{9z}{2} - \frac{15z^2}{4} \dots \]

This is the Laurent series valid in the innermost region.

For the outer region, manipulate each fraction:
\[ \frac{1}{1 + z} = \frac{1}{z} \cdot \frac{1}{1 + 1/z} \]
\[ \frac{1}{2 - z} = -\frac{1}{z} \cdot \frac{1}{1 - 2/z} \]
We do this so that the geometric series will be convergent for $|z| > 2$ (the ratio is $2/z$ in the second one).

Again, geometric series.  Write them separately:
\[ \frac{1}{z} \ [ \ 1 - \frac{1}{z} + \frac{1}{z^2} - \frac{1}{z^3} + \frac{1}{z^4} + \dots \ ] \]
\[ -\frac{1}{z} \ [ \ 1 + \frac{2}{z} + \frac{4}{z^2} + \frac{8}{z^3} + \frac{16}{z^4} \dots \ ] \]
Move the minus sign inside:
\[ \frac{1}{z} \ [ \ -1 - \frac{2}{z} - \frac{4}{z^2} - \frac{8}{z^3} - \frac{16}{z^4} \dots \ ] \]

Add
\[ \frac{1}{z} \ [  -\frac{3}{z} - \frac{3}{z^2} - \frac{9}{z^3} - \frac{15}{z^4} + \dots \]
Recall the leading factor of $4/z$, and get another factor of $-3/z$ giving what it has in the book:
\[ -\frac{12}{z^3} \ [ \ 1 + \frac{1}{z} + \frac{3}{z^2} + \frac{5}{z^3} + \dots \ ] \]

The last part is the annulus in the middle.  For this we want convergence for $|z| > 1$ and for $|z| < 2$.  Hence we want 

\[ \frac{1}{1 + z} = \frac{1}{z} \cdot \frac{1}{1 + 1/z} \]
\[ = \frac{1}{z} \ [ \ 1  -\frac{1}{z} + \frac{1}{z^2} - \frac{1}{z^3} + \frac{1}{z^4} + \dots \ ] \]
and 
\[ \frac{1}{2 - z} = \frac{1}{2} \cdot \frac{1}{1 - z/2} = \frac{1}{2} \ [ \ 1 + \frac{z}{2} + \frac{z^2}{4} + \frac{z^3}{8} \dots \ ] \]
I need a factor of $1/z$ on the latter (and move the factor of $2$ inside:
\[ \frac{1}{z} \ [ \ \frac{z}{2} + \frac{z^2}{4} + \frac{z^3}{8} + \frac{z^4}{16} \dots \ ] \]

We can add them and remember the leading factor of $4/z$ so it's
\[ \frac{4}{z^2} \ [ \frac{z}{2} + \frac{z^2}{4} + \frac{z^3}{8} + \frac{z^4}{16} \dots + 1  -\frac{1}{z} + \frac{1}{z^2} - \frac{1}{z^3} + \frac{1}{z^4} + \dots \ ] \]

I try to verify one power, pick $z^{-1}$  
\[ \frac{6}{z} + \frac{2}{z} \]
and that, unfortunately is not a match.  But I know we're close, and you can see the general idea.

\subsection*{example}

Consider:
\[ f(z) = \frac{z}{(z-1)(z-3)} \]
and say we need the series around $0 \le | z - 1 | \le 2$, also written as $C[1,2]$, a circle of radius $2$ around the point $z_0 = 1$.

One way to do this is to make substitution $x = z - 1$, so $z = x + 1$ and we have
\[  = \frac{x+1}{(x)(x-2)} \]
Factor out the $1/x$
\[ = \frac{1}{x} \ ( \frac{x+1}{x-2} ) \]
(this is easy to restore at the end by multiplying by $1/x$).

And then our goal is to get something like $1/1-r$.  

That means we want $x - 2$ on top
\[ = \frac{1}{x} \ ( \frac{x - 2 +  3}{x - 2} ) \]
\[ = \frac{1}{x} \ ( 1 + \frac{ 3}{x-2} ) \]
\[ = \frac{1}{x} \ ( 1 - \frac{ 3}{2-x} ) \]
\[ = \frac{1}{x} \ ( 1 - \frac{ 3/2}{1 - x/2} ) \]
\[ = \frac{1}{x} \ ( 1 - \frac{3}{2} \cdot \frac{1}{1 - x/2} ) \]

and now the 
\[ \frac{1}{1 - x/2} \]
can be expanded because that's a geometric series
\[ \frac{1}{1 - r} = 1 + r + r^2 + r^3 + \dots \]
so 
\[ \frac{1}{1 - x/2} = 1 + \frac{x}{2} + (\frac{x}{2} )^2 + \dots \]
which gives
\[ = \frac{1}{x} \ [ 1 - \frac{3}{2} \cdot (1 + \frac{x}{2} + (\frac{x}{2} )^2 + \dots) \ ] \]

Now, multiplying through by $1/x$ gives
\[ -\frac{1}{2x} + \dots \]
and \emph{nothing else matters}.  Reverse the change of variable:
\[ = -\frac{1}{2(z-1)} + \dots \]

which we will integrate as
\[ \oint \frac{-1/2}{z-1} \ dz \]
 over $C[1,2]$.
 
Recall the formula for residues:
\[ b_1 = \lim_{z \rightarrow z_0} (z-z_0) \ f(z)  \]

So
\[ \text{Res }(1) = \lim_{z \rightarrow 1} (z-1) \ \frac{-1/2}{z-1} = -\frac{1}{2} \]
Multiply by $2 \pi i$ to obtain $I = -\pi i$.

As a check on this go back to 
\[ f(z) = \frac{z}{(z-1)(z-3)} \]
\[ \text{Res }(1) = \lim_{z \rightarrow 1} (z-1) \ \frac{z}{(z-1)(z-3)} \]
\[ =  \lim_{z \rightarrow 1} \ \frac{z}{z-3} \]
\[ = -\frac{1}{2} \]

\subsection*{example}
These examples can get complicated.  Here is one from

\url{http://zimmer.csufresno.edu/~doreendl/128.13f/handouts/Lseriesex.pdf}

\[ f(z) = \frac{1}{(z-2)(z-1)} \]

This function has poles at $z = 1$ and $z = 2$.  If we are asked to write expansions around $z_0 = 0$, then we have three regions of interest and three different expansions.

The first region is the circle of radius $1$:  $|z| < 1$, the second is $1 < |z| < 2$ and then finally $|z| > 2$.

\subsection*{region 1}
Use partial fractions to write:
\[ \frac{1}{(z-2)(z-1)} = \frac{1}{z-2} - \frac{1}{z-1} \]
Considering the second term, we bring the minus sign inside
\[ = \frac{1}{z-2} + \frac{1}{1-z} \]
We have the classic
\[ \frac{1}{1-z} = 1 + z + z^2 = \sum_{n=0}^{\infty} z^n \]
which we know this is valid for $|z| < 1$, the region of interest.

For the other term
\[ \frac{1}{z - 2} = - \frac{1}{2 - z} \]
Our goal is to convert this into something like the geometric series.  Factor out the $2$ on the bottom like so
\[ = - \frac{1}{2} \ [ \  \frac{1}{1 - z/2} \ ]  \]

We can do a formal substitution or recognize that this is the geometric series 
\[ = - \frac{1}{2} \ [ \   \sum_{n=0}^{\infty} (z/2)^n \ ]  \]
We can rewrite this slightly by pulling out the factor of $2^n$ on the bottom and combining it with the factor of $2$ out front:
\[ = \sum_{n=0}^{\infty} \ [ \ \frac{-1}{2^{n+1}} \ ] \  z^n \  \]
which converges for $0 < |z/2| < 1 \Rightarrow 0 < |z| < 2$.

Our series is the sum of these two series, which can be combined as
\[ \sum_{n=0}^{\infty} \ [ \ 1 - \frac{1}{2^{n+1}} \ ] \  z^n \  \]

\subsection*{region 2}
This is the annulus $1 < |z| < 2$.  Thus
\[ | \frac{1}{z} | < 1 \  \ \ \text{ and } \ \ \  |\frac{z}{2} | < 1 \]

What they do is to work on the right-hand term of
\[ \frac{1}{(z-2)(z-1)} = \frac{1}{z-2} - \frac{1}{z-1} \]
and, as we saw in the previous section transform it into something containing $1/z$, which will be valid in the region $|z| > 1$.

So let's do it:
\[ \frac{1}{1 - z} = - \frac{1}{z - 1}  \]
\[ = - \frac{1}{z} \cdot \frac{1}{1 - 1/z}  \]
leaving aside the leading factor this is
\[ = 1 + \frac{1}{z} + \frac{1}{z^2} \dots \]
\[ = \sum_{n=0}^{\infty} \frac{1}{z^n} \]
add back that factor
\[ -\frac{1}{z} \ \sum_{n=0}^{\infty} \frac{1}{z^n} \]

The left-hand term is exactly what we had before:
\[ \sum_{n=0}^{\infty} \ [ \ \frac{-1}{2^{n+1}} \ ] \  z^n \  \]
so we combine them
\[ \sum_{n=0}^{\infty} \ [ \ \frac{-1}{2^{n+1}} \ ] \  z^n -\frac{1}{z} \ \sum_{n=0}^{\infty} \frac{1}{z^n} \]
and then just bring that $z$ in the second term inside
\[ \sum_{n=0}^{\infty} \ [ \ \frac{-1}{2^{n+1}} \ ] \  z^n - \ \sum_{n=0}^{\infty} \frac{1}{z^{n+1}} \]
or change the index
\[ = \sum_{n=0}^{\infty} \ [ \ \frac{-1}{2^{n+1}} \ ] \  z^n - \ \sum_{n=1}^{\infty} \frac{1}{z^{n}} \]

\subsection*{region 3}
We do the $1/z$ trick with both terms
\[ \frac{1}{z-2} - \frac{1}{z-1} \]
Start with the first one:
\[ \frac{1}{z-2} = \frac{1}{z} \cdot \frac{1}{1 - 2/z} \]
The series is
\[ \frac{1}{z} \ \cdot \ \sum_{n=0}^{\infty} \ [ \ \frac{2}{z} \ ]^n \]
\[ = \sum_{n=0}^{\infty} \ \frac{2^n}{z^{n+1}} \]

The second term is (leaving off the factor of $-1$)
\[ \frac{1}{z-1} = \frac{1}{z} \cdot \frac{1}{1 - 1/z} \]
The series is
\[ \frac{1}{z} \ \cdot \ \sum_{n=0}^{\infty} \ [ \ \frac{1}{z} \ ]^n \]
\[ = \sum_{n=0}^{\infty} \ \frac{1}{z^{n+1}} \]

Combining the two results and bringing back the factor we get
\[ \sum_{n=0}^{\infty} \ \frac{2^n}{z^{n+1}} -  \sum_{n=0}^{\infty} \ \frac{1}{z^{n+1}}  \]
\[ = \sum_{n=0}^{\infty} \ (2^{n} - 1) \ \frac{1}{z^{n+1}} \]
adjust the index
\[ = \sum_{n=1}^{\infty} \ (2^{n-1} - 1) \ \frac{1}{z^{n}} \]

\subsection*{example}
\[ f(z) = \frac{1}{z(z+2)} \]
Suppose the region of interest is an annulus centered on $z = 1$ with $1 < |z-1| < 3$.
\begin{center} \includegraphics [scale=0.5] {writeseries1.png} \end{center}

The first thing to do is make a substitution that translates the region so that it becomes centered on the origin:  $w = z - 1$.  Then the function becomes
\[  \frac{1}{(w + 1)(w + 3)} \]
The next thing is to write partial fractions.  For the numerator we get
\[ A(w+3) + B(w+1) = 1 \]
\[ A = - B = \frac{1}{2} \]
Hence
\[ \frac{1}{2} \cdot \ [ \ \frac{1}{w+1} - \frac{1}{w+3} \ ] \]
The third step is to convert each of these fractions into something like $1/1-x$.
\[ \frac{1}{w+1} = \frac{1}{1 - (-w)} \]
\[ \frac{1}{w+3}  = \frac{1}{3} \cdot \frac{1}{1 - (-w/3)} \]
And then the fourth step is to write the series, recalling that we want different forms depending on whether we are in a circle or an annulus.

\[ \frac{1}{1 - (-w)} \]
\[ = \sum_{n=0}^{\infty} (-w)^n = \sum_{n=0}^{\infty} (-1)^n (w)^n , \ \ \ |w| < 1 \]
\[ = -\sum_{n=1}^{\infty} \frac{1}{(-w)^n} =  -\sum_{n=1}^{\infty} \frac{(-1)^n}{w^n}, \ \ \ |w| > 1 \]
We pick the second form because our region is $1 < |z-1| < 3$

A similar thing can be done for the other term.  We show only the first series since we are inside the circle.
\[  \frac{1}{3} \cdot \frac{1}{1 - (-w/3)} \]
\[ = \frac{1}{3} \cdot  \sum_{n=0}^{\infty} (-1)^n (\frac{w}{3})^n , \ \ \ |w| < 3 \]
\[ = \sum_{n=0}^{\infty} (-1)^n \ \frac{1}{3^{n+1}} \ w^n , \ \ \ |w| < 3 \]

Add the two series together (remembering the minus sign on the second term)
\[ -\sum_{n=1}^{\infty} \frac{(-1)^n}{w^n} - \sum_{n=0}^{\infty} (-1)^n \ \frac{1}{3^{n+1}} \ w^n \]
and then picking up the leading factor from 
\[ \frac{1}{2} \cdot \ [ \ \frac{1}{w+1} - \frac{1}{w+3} \ ] \]
so
\[ \frac{1}{2} \ [ \ -\sum_{n=1}^{\infty} \frac{(-1)^n}{w^n} - \sum_{n=0}^{\infty} (-1)^n \ \frac{1}{3^{n+1}} \ w^n \ ] \]

The last step is to reverse the substitution:  $w = z - 1$ and bring the minus sign out front
\[ f(z) = - \frac{1}{2} \ [ \ \sum_{n=1}^{\infty} \frac{(-1)^n}{(z-1)^n} \sum_{n=0}^{\infty} (-1)^n \ \frac{1}{3^{n+1}} \ (z-1)^n \ ] \]

I don't know if I could ever learn to do this well, but at least the explanations make sense.

Now, if we were to integrate $f(z)$, we would have only one term that gives a non-zero result, namely the first term with $n=1$
\[ - \frac{1}{2} (-1) \frac{1}{z-1} \]
\[ \text{Res }(1) = \lim_{z \rightarrow 1} \frac{1}{2} = \frac{1}{2} \]
Multiply by $2 \pi i$ to obtain $\pi i$.

\subsection*{simpler view}

On the other hand, I would just write the partial fraction:
\[ \int \frac{1}{2} \ [ \ \frac{1}{z} - \frac{1}{z + 2} \ ] dz \]

The curve $C[1,3]$ includes the singularity at $z = 0$, but $z = -2$ is on the boundary, not inside the region.  The curve $C[1,1]$ includes neither point.

So for the second curve the integral is zero and for the first one only the $\int 1/z \ dz$ matters.  It's an old friend, with value $2 \pi i$, the value of the integral is thus $\pi i$.

\subsection*{example}

Here are two examples from Brown and Churchill.

\[ \int \frac{1}{z(z-2)^4} \ dz \]
with singularities at $z = 0$ and $z = 2$.

Use the $1/z$ part to get a geometric series:
\[ \frac{1}{z(z-2)^4} = \frac{1}{(z-2)^4} \cdot \frac{1}{(2 + z - 2)} \]
\[ = \frac{1}{(z-2)^4} \cdot \frac{1}{2} \cdot \frac{1}{(1 - (-(z-2))/2 )} \]

The third term gives the geometric series with common ratio $(z-2)/2$.  Those $z-2$ terms will cancel the leading factor.  The only term that matters is the cube, which gives:
\[  \frac{1}{(z-2)} \cdot \frac{1}{2} \cdot \frac{1}{(-2)^3} \]

We have that $b_1 = -1/16$ so $I = -\pi i/8$.

The second one is:
\[ \int e^{1/z^2} \ dz \]
We use the standard series for $e^z$
\[ e^z = 1 + z + \frac{z^2}{2!} + \frac{z^3}{3!} + \dots \]
substituting $1/z^2$
\[ 1 + \frac{1}{z^2} + \frac{z^r}{2!} + \frac{z^6}{3!} + \dots \]

Since there's no $z$ with a power $n=-1$, the value of the integral is just zero.





\end{document}