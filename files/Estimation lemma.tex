\documentclass[11pt, oneside]{article} 
\usepackage{geometry}
\geometry{letterpaper} 
\usepackage{graphicx}
	
\usepackage{amssymb}
\usepackage{amsmath}
\usepackage{parskip}
\usepackage{color}
\usepackage{hyperref}

\graphicspath{{/Users/telliott/Github/figures/}}

\title{Estimation Lemma}
\date{}

\begin{document}
\maketitle
\Large

There are actually two kinds of complex functions that seem different.  

One type takes a complex number $x + iy$ as input and outputs a complex number $w$.  We could write $f: \mathbb{C} \rightarrow \mathbb{C}$.

This hides what's really going on, that $f$ is a composition of two functions with $u,v: \mathbb{R}^2 \rightarrow \mathbb{R}$.
\[ f(z) = u(x,y) + i v(x,y) = w \]

There is another kind of complex function which makes this opaque.  

$f(t)$ takes one real parameter as the variable, obtains a value for $x$ and $y$ from $\gamma(t) = x + iy$, and then generates $w$ by operating with something like $u$ and $v$ on that intermediate result.

Sometimes, to save letters, one writes for a parametrized function:
\[ \gamma(t) = x(t) + i y(t) \]
and $\int \gamma(t)$ over some path uses $x$ and $y$ generated by $\gamma$, with $a \le t \le b$:
\[ I = \int_a^b \gamma(t) \ \gamma'(t) \ dt \]
We did some examples which show that it's not necessarily to explicitly generate $x$ and $y$.

%[my-super-duper-separator]

\subsection*{Lemma:  Triangle inequality for integrals (Karkhar)}

\label{sec:tri_inequality_integrals}

\[ | \int_a^b f(t) \ dt | \le \int_a^b |f(t)| \ dt \]

\emph{Proof}.

With the right conditions on $f$ and the path, the integral on the left-hand side $\int_a^b f(t) \ dt$ exists.  It is some complex number.

Let's call it $K$, where  
\[ \int_a^b f(t) \ dt = K = |K| e^{i \theta} \]
\[ |K| = K e^{-i \theta} \]

We will assume that $K \ne 0$ to make things simpler.  (The result is still valid for $K = 0$, since on the right-hand side, $|f(t)| \ge 0$ and therefore so is its integral, and it follows that $|0| = 0 \le \int |f(t)| \ dt$).

Written in terms of $f(t)$
\[ 0 \le |K| = \frac{K}{e^{i \theta}}  = K e^{-i \theta} \]
\[ = e^{-i \theta} \int_a^b f(t) \ dt \]

(This would not be possible if $K = 0$).

Now, $K$ is some fixed value, so $ \theta$ is fixed and $e^{-i \theta}$ is a constant, which allows us to bring it inside the integral
\[ = \int_a^b e^{-i \theta} f(t) \ dt \]

For the moment, let's hide all that by defining: 
\[ g(t) = e^{-i \theta} f(t) \]

This integral is positive:
\[ 0 \le |K| = \int_a^b g(t) \ dt  \]

It is also entirely real, since it is equal to a real number, $|K|$.  Therefore:
\[ \int g(t) \ dt = Re \{ \int g(t) \ dt \} \]

Now, consider the right-hand side.  Since $g$ is just two functions added together
\[ \int g(t) \ dt = \int (x(t) \ dt + i \int y(t) \ dt \]

the real part of the right-hand side is $\int (x(t) \ dt$:
\[ = \int_a^b Re \{ g(t) \} \ dt \]

This is really a matter of definition of the integral (see Brown and Churchill sect 38).

This is the crucial step:
\[ \int_a^b Re \{ g(t)\} \ dt \le \int_a^b |g(t)| \ dt  \]

For any given $t$, $g(t)$ is also some complex number and the real part of a complex number is always less than or equal to the length of the modulus, by the triangle inequality.

We have established that 
\[ |K| \le \int_a^b |g(t)| \ dt \]

We go back to $f(t)$.  The right-hand side is
\[ = \int_a^b | e^{-i\theta} f(t) | dt = \int_a^b | e^{-i\theta} |  | f(t) | dt \]

by the Lemma on the product of moduli (\hyperref[sec:product_of_moduli]{\textbf{here}}). But $| e^{-i\theta} | = 1$ so
\[ = \int_a^b |f(t)| dt \]

Going back to pick up $|K|$
\[ |K| \le \int_a^b |f(t)| dt \]
and then restating the initial definition of $|K|$:
\[  | \int_a^b f(t) \ dt | = |K| \le \int_a^b |f(t)| \ dt \]

$\square$

We can use that result for the proof of the Estimation lemma.

\subsection*{Estimation Lemma:  (Brown and Churchill)}

\label{sec:estimation_lemma}

The statement is that, for a contour $C$ of length $L$, with $f(z)$ piecewise continuous on $C$ and if $M$ is a non-negative constant such that
\[ |f(z)| \le M \]
then 
\[ |\int_C f(z) \ dz | \le ML \]

\emph{Proof}.

Let $\gamma = \gamma(t)$ with $a \le t \le b$ be a parametrization of $C$.  

According to the above lemma,
\[ |\int_C f(z) \ dz |  = |\int_a^b f[\gamma(t)] \ \gamma'(t)| \ dt  \le \int_a^b |f(\gamma(t)) \ \gamma'(t)| \ dt \]
But 
\[ |f[\gamma(t)] \ \gamma'(t)|  = |f(\gamma(t)]| \ |\gamma'(t)|  \le M |\gamma'(t)| \]
so it follows that
\[ |\int_C f(z) \ dz | \le M \int_a^b |\gamma'(t)| \ dt \]
But the integral in the last term is the length $L$ of $C$.

This is also true for $f$ \emph{piecewise} continuous on $C$.

$\square$

\subsection*{Estimation Lemma:  (Karkhar)}

Let $M$ be the maximum value of $v(\gamma(t))$, then
\[ | \int_\gamma v(z) \ dz | \le M \int_a^b \gamma'(t) \ dt = ML \]
where $L$ is the length of the curve.

\emph{Proof}.

By the Triangle inequality for integrals:
\[ | \int_\gamma v(z) \ dz | = | \int_a^b v(\gamma(t)) \ \gamma'(t) \ dt | \]
\[ \le \int_a^b |v(\gamma(t))| \ |\gamma'(t)| \ dt \]

Let $M$ be the maximum value of $v(\gamma(t))$, then
\[ | \int_\gamma v(z) \ dz | \le M \]

for all points on $\gamma$.  But then

\[ | \int_\gamma v(z) \ dz | = M \int_a^b \gamma'(t) \ dt = ML \]

where $L$ is the length of the curve.  The final step requires \hyperref[sec:length_of_curve]{\textbf{this}}, our lemma on the length of a curve parametrized by $\gamma$.

\subsection*{Estimation lemma (Beck)}

Suppose $\gamma$ is a piecewise smooth path, $f$ is a complex function which is continuous on $\gamma$, then
\[ | \int_\gamma f \ | \le \  \text{max}_{z \in \gamma} \ |f(z)| \cdot \ \text{length}(\gamma)   \]

What this says is that for all the values of $z$ on $\gamma$, pick the one with the largest $|f(z)|$, the biggest modulus.  Call that number $M$, and let $L$ be the length of the curve parametrized by $\gamma$.

Then, the value of the integral is a complex number, whose modulus is less than or equal to $M$.  The important thing is that it is bounded, so when multiplied by some small number that tends to zero, the whole thing will tend to zero.

\emph{Proof}.

Let $\phi = $Arg $(\int_\gamma f)$.  That is, $\int_\gamma f = | \int_\gamma f | e^{i\phi}$.  We know that $| \int_\gamma f |$ is a real number.

\[ \int_\gamma f = | \int_\gamma f | e^{i\phi} \]
\[ | \int_\gamma f | = e^{-i\phi} \int_\gamma f \]
\[ = Re \ ( e^{-i\phi} \int_a^b f(\gamma(t)) \ \gamma'(t) \ dt ) \]
\[ = \int_a^b Re \ (f (\gamma(t)) \ e^{-i\phi} \ \gamma'(t)) \ dt \]

and then since the expression in parentheses is a complex number, its real part is less than or equal to its modulus
\[ \le \int_a^b | f (\gamma(t)) \ e^{-i\phi} \ \gamma'(t) | \ dt \]
which we can break up into pieces, by the (\hyperref[sec:product_of_moduli]{\textbf{product of moduli}}), hence
\[ = \int_a^b | f (\gamma(t)) | \ |e^{-i\phi}| \ |\gamma'(t)| \ dt \]

Since $|e^{-i\phi}| = 1$
\[ = \int_a^b | f (\gamma(t)) | \ |e^{-i\phi}| \ |\gamma'(t)| \ dt \]
The first part is the maximum value of $f(\gamma(t))$ for $a \le t \le b$, and the second part is the length of the curve from \hyperref[sec:length_of_curve]{\textbf{here}}.

\[ \int_\gamma f \le M \cdot L \]

$\square$

\subsection*{Another proof}

I think this is from Beck, but I'm not sure where it is in the book.

We have seen the (\hyperref[sec:tri_inequality]{\textbf{Triangle inequality}}), which says that for two complex numbers $z$ and $w$:
\[  |z + w| \le |z| + |w| \]

We can write a similar inequality for integrals
\[ | \int_a^b g(t) \ dt | \le \int_a^b | g(t) | \ dt \]

\emph{Proof.}

Since we approximate the integral as a Riemann sum.

\[ |\int_a^b g(t) \ dt| \approx | \sum g(t_k) \ \Delta t | \le \sum | g(t_k) | \ \Delta t \approx \int_a^b |g(t)| \ dt \]

Using that result, we can apply it to the integral along a curve or contour:
\[ | \int_\gamma f(z) \ dz |  \le \int_{\gamma} | f(z) | \ dz \]

\emph{Proof.}

\[ | \int_{\gamma} f(z) \ dz | = | \int_a^b f(\gamma(t)) \gamma'(t) \ dt | \]
\[ \le \int_a^b | f(\gamma(t)) | | \gamma'(t) | \ dt \]

So what that means is that if $|f(z)| < M$ along $C$, then
\[ | \int_C f(z) \ dz | \le M \cdot L \]
where $L$ is the length of $C$.

\emph{Proof.}

where we have used the fact that
\[ |\gamma'(t) | \ dt = ds \]
is the arc length element.


\end{document}