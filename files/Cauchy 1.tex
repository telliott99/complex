\documentclass[11pt, oneside]{article} 
\usepackage{geometry}
\geometry{letterpaper} 
\usepackage{graphicx}
	
\usepackage{amssymb}
\usepackage{amsmath}
\usepackage{parskip}
\usepackage{color}
\usepackage{hyperref}

\graphicspath{{/Users/telliott_admin/Tex/png/}}
% \begin{center} \includegraphics [scale=0.4] {gauss3.png} \end{center}

\title{Cauchy's Integral Theorem}
\date{}

\begin{document}
\maketitle
\Large

Cauchy's first theorem says that the integral of an analytic function over a closed path is equal to zero, when the enclosed region does not contain a singularity.
\[ \oint_C f(z) \ dz = 0 \]
This will turn out to be a consequence of Green's Theorem, which you should remember from multivariable calculus.  Let
\[ z = x + i y \]
\[ dz = dx + i dy \]
\[ z = f(x,y) = u(x,y) + iv(x,y) \]
Our integral is
\[ \oint z \ dz = \oint (u(x,y) + iv(x,y)) \ (dx + i dy) \]
\[ =  \oint u(x,y) \ dx - \oint v(x,y) \ dy + i \oint v(x,y) \ dx + i \oint u(x,y) \ dy \]
As before, because we are moving along a curve there is a relationship between $x$ and $y$, so we can either express that relationship (for say, $y$ in terms of $x$), or parametrize the curve in terms of $t$ or $\theta$.  In either case, these become integrals in a single variable.  We suppress the $(x,y)$ notation and the extra integral signs:
\[ =  \oint u \ dx - v \ dy + i v \ dx + i u \ dy \]

\subsection*{proof of Cauchy 1}
Back in vector calculus we proved Green's theorem, which says that for two real functions of $x$ and $y$:  $M(x,y)$ and $N(x,y)$:
\[ \oint_C M dx + N dy = \iint_R (\frac{\partial N}{\partial x} - \frac{\partial M}{\partial y}) \ dx \ dy \]
Back then, $M$ and $N$ were components of a vector field $\mathbf{F}$ and we wrote the shorthand for curl:
\[ = \iint_R \nabla \times \mathbf{F} \ dA\]
but the important thing is that they are real-valued functions of two real variables $f: \mathbb{R}^2 \rightarrow \mathbb{R}^1$.

In terms of $u$ and $v$ we have for the real part of Cauchy's Theorem that $M=u$ and $N = -v$ (notice the minus sign!).  

So:
\[ \oint u \ dx - v \ dy = \iint_R (-\frac{\partial v}{\partial x} -  \frac{\partial u}{\partial y}) \ dx \ dy \]
\[ = - \iint_R (v_x + u_y) \ dx \ dy \]
But, according to the CRE
\[ u_y = -v_x \]
Hence, this integral is zero.

For the imaginary part we use Green's Theorem again with $N=u$ and $M = v$ (no minus sign here):
\[ \oint v \ dx + u \ dy =  \iint_R (\frac{\partial u}{\partial x} - \frac{\partial v}{\partial y}) \ dx \ dy \]
\[ =  \iint_R (u_x - v_y) \ dx \ dy \]
But, again, according to the CRE
\[ u_x = v_y \]
So the integral for the imaginary part is also zero, and thus the whole thing is zero as well:
\[ \oint u \ dx - \oint v \ dy + i \oint v \ dx + i \oint u \ dy = 0 \]

Remember how important it was (for Green's theorem) that the function being integrated be defined everywhere in the region.  Well, it's true here as well.

\[ \oint_C \frac{1}{z} \ dz \stackrel{?}{=} 0 \]

We've already seen by direct calculation that this integral is \emph{not} zero when the curve $C$ includes the origin, but it \emph{is} zero otherwise.

To repeat the demonstration for the former case we use the unit circle centered at the origin.  Write
\[ z = r e^{i\theta} \]
\[  \frac{dz}{d\theta} = r i e^{i\theta} = iz \]
Hence
\[ \oint_C \frac{1}{z} \ dz = \oint_C \frac{1}{z} \ iz \ d \theta \]
\[ = i   \oint_C  d \theta = 2 \pi i \]

\subsection*{Path independence}
The theorem that says the integral of an analytic function over a closed path (over a region without a singularity), is equal to zero.
\[ \oint_C f(z) \ dz = 0 \]

This result means, in turn, that the integral of an analytic function between two points $z_1$ and $z_2$ is independent of the path taken.  Call the two paths $C1$ and $C2$.  

Form the closed path by going from $z_1$ to $z_2$ over $C1$ and then return to $z_1$ by going backward over $C2$.  The total integral is equal to zero by Cauchy's Theorem.
\[ \int_{C1} f(z) \ dz + \int_{-C2} f(z) \ dz = 0 \]
But the integral over the path $C_2$ in the forward direction is just minus the integral over the reverse path $-C2$.
\[ \int_{-C2} f(z) \ dz = - \int_{C2} f(z) \ dz \]
Thus
\[ \int_{C1} f(z) \ dz - \int_{C2} f(z) \ dz = 0 \]
and
\[ \int_{C1} f(z) \ dz = \int_{C2} f(z) \ dz \]

\subsection*{example}

We will integrate the function $f(z) = z$ over a rectangle ($R = [0,a] \times [b,0]$.  Write
\[ z = x + i y \]
\[ dz = dx + i dy \]
\[ f(x,y) = u(x,y) + iv(x,y) \]
Our integral is
\[ \int z \ dz = \int (u + iv) \ (dx + i dy) \]
\[ = \int u \ dx - \int v \ dy + i \int v \ dx + i \int u \ dy \]
Since the whole thing is equal to zero over our closed path, both parts are equal to zero:
\[ \int u \ dx - \int v \ dy = 0 \]
\[ \int v \ dx + \int u \ dy = 0 \]
Does this look familiar?

\end{document}  