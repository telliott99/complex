\documentclass[11pt, oneside]{article} 
\usepackage{geometry}
\geometry{letterpaper} 
\usepackage{graphicx}
	
\usepackage{amssymb}
\usepackage{amsmath}
\usepackage{parskip}
\usepackage{color}
\usepackage{hyperref}

\graphicspath{{/Users/telliott/Github/figures/}}

\title{Taylor series}
\date{}

\begin{document}
\maketitle
\Large

%[my-super-duper-separator]

\url{https://math.mit.edu/~jorloff/18.04/notes/topic7.pdf}

A standard geometric series with a finite number of terms is
\[ S_n = a(1 + r + r^2 + r^3 + \dots + r^n) \]
\[ S_n = a + ar + ar^2 + ar^3 + \dots + ar^n) \]
where $a$ is the \emph{first term} and $r$ is the \emph{common ratio}.

Because the series is \emph{finite}, we know the sum exists, which allows us to write $S_n$.  The standard trick is
\[ S_n \ r = ar + ar^2 + ar^3 + \dots + ar^n + ar^{n+1}) \]
\[ S_n (1 - r) = ar - ar^{n+1} \]
\[ S_n = \frac{a(1 - r^{n+1})}{1 - r} \]

A series is understood to have an infinite number of terms.
\[ S = a + ar + ar^2 + ar^3 + \dots = \lim_{n \rightarrow \infty} S_n \]
Provided that $|r| < 1$, the term $r^{n+1}/(1-r)$ goes to zero in the limit, and then
\[ S = \frac{a}{1-r} \]

\subsection*{expansions}
We will often see the expression 
\[ \frac{1}{w - z} \]

$w$ is a complex variable and $z$ is a complex number.

This looks a lot like the sum of a geometric series and can be converted to one in two different ways.
\[ \frac{1}{w - z} = \frac{1}{w} \cdot \frac{1}{1 - z/w} \]
The second term on the right-hand side is a geometric series in $z/w$:
\[ =  \frac{1}{w} (1 + \frac{z}{w} + (\frac{z}{w})^2 + (\frac{z}{w})^3 + \dots ) \]
which converges when $|z/w| < 1$, that is, when $|w| > |z|$.

The other way is
\[ \frac{1}{w - z} = -\frac{1}{z} \cdot \frac{1}{1 - w/z} \]
The second term on the right-hand side is a geometric series in $w/z$:
\[ =  -\frac{1}{z} (1 + \frac{w}{z} + (\frac{w}{z})^2 + (\frac{w}{z})^3 + \dots ) \]
which converges when $|w/z| < 1$, that is, when $|w| < |z|$.

\subsection*{power series in $z$}

Here is a power series in $z$, where $z_0$ is a fixed point:
\[ f(z) = \sum_{n=0}^{\infty} a_n(z - z_0)^n \]

Theorem (without proof):

There is a number $R \ge 0$ such that

$\circ$ \ if $R > 0$, then the series converges absolutely to an analytic function, for $|z - z_0| < R$.

$\circ$ \ The series diverges for $|z - z_0| > R$. $R$ is called the radius of convergence. The disk $|z - z_0| < R$ is called the disk of convergence.

$\circ$ \ The derivative is given by term-by-term differentiation.  The series for $f'$ has the same radius of convergence.

$\circ$ \ If $\gamma$ is a bounded curve inside the disk of convergence, the integral is given by term-by-term integration:
\[ \int_{\gamma} f(z) \ dz = \sum_{n=0}^{\infty} \ \int_{\gamma} a_n(z - z_0)^n \ dz \]

If $R = \infty$ the function $f(z)$ is called \emph{entire}.  Often, $R$ may be determined by the ratio test.

\subsection*{ratio test}
Consider the series $\sum_0^{\infty} c_n$.  If $L = \lim_{n \rightarrow \infty} |c_{n+1}/c_n|$ exists, then:

$\circ$ \ if $L < 1$ the series converges absolutely.

$\circ$ \ if $L > 1$ the series diverges.

$\circ$ \ if $L = 1$ we don't know.

\subsection*{examples}

Consider the geometric series
\[ 1 + z + z^2 + z^3 + \dots \]

The ratio of the absolute values of consecutive terms is $|z|$ and the limit is:
\[ L = \lim_{n \rightarrow \infty} \frac{|z+1|}{|z|} = |z| \]

Convergence follows when $L < 1$, that is when $|z| < 1$.

Consider
\[ f(z) = \sum_{n=0}^{\infty} \ \frac{z^n}{n!} \]

\[ L =  \lim_{n \rightarrow \infty} \frac{|z^{n+1}|}{(n+1) |z^n|} =  \lim_{n \rightarrow \infty} \frac{|z|}{n+1} \]

Convergence follows when $L < 1$, this happens for all values of $z$.

Of course, this is the exponential function $f(z) = e^z$, which is \emph{entire}.  It is analytic for the whole complex plane.

To summarize, 

\begin{quote}we've seen that a power series converges to an analytic function inside its disk of convergence. Taylor’s theorem completes the story by giving the converse: around each point of analyticity an analytic function equals a convergent power series.\end{quote}


\end{document}