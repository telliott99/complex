\documentclass[11pt, oneside]{article} 
\usepackage{geometry}
\geometry{letterpaper} 
\usepackage{graphicx}
	
\usepackage{amssymb}
\usepackage{amsmath}
\usepackage{parskip}
\usepackage{color}
\usepackage{hyperref}

\graphicspath{{/Users/telliott/Github/figures/}}

\title{Laurent series}
\date{}

\begin{document} 
\maketitle
\Large

%[my-super-duper-separator]

Any function that is analytic inside a disk has a power series (a Taylor series) convergent inside that disk.  

We know that
\[ e^z = 1 + z + \frac{z^2}{2!} + \frac{z^3}{3!} \dots \]
which in fact converges everywhere.  So then consider:
\[ \frac{e^z}{z} = \frac{1}{z} \ [ \ 1 + z + \frac{z^2}{2!} + \frac{z^3}{3!} \dots \ ] \]
\[ = \frac{1}{z} + 1 + z + \frac{z^2}{2!} + \frac{z^3}{3!} \dots \]
This isn't defined at $z = 0$ but is defined everywhere else.

We can integrate a power series term by term.  We know by Cauchy's integral theorem that the only term with a non-zero contour integral is $1/z$ so
\[ \oint \frac{e^z}{z} = \oint \frac{1}{z} = 2 \pi i \]

As the power of $z$ in the denominator increases, eventually we'll see factorial terms:
\[ \oint \frac{e^z}{z^3} = \pi i \]

And it's good that we can do this type of problem with a series, because a parametrization of it rapidly leads to not fun stuff like $e^{e^{i \theta}}$ or ($\int e^t/t$).

So what about a circular region, excluding the center.  That is called a punctured disk.
\[ 0 < |z-z_0| < R \]
Or what about a donut or annulus?
\[ 0 < r < |z-z_0| < R \]
Then the applicable series is a Laurent series.

The HELM guys say: 
\begin{quote}One of the shortcomings of Taylor series is that the circle of convergence is often only a part of the region in which $f(z)$ is analytic.\end{quote}

\url{https://learn.lboro.ac.uk/archive/olmp/olmp_resources/pages/workbooks_1_50_jan2008/Workbook26/26_6_snglrts_n_resdus.pdf}

An example is
\[ f(z) = \frac{1}{1 - z} \]
This function is analytic everywhere except at the singularity $z = 1$.  The Taylor series expanded around $z = 0$ is
\[ 1 + z + z^2 + z^3 + \dots \]
which converges to $f(z)$ only for $|z| < 1$.

\begin{quote}The radius of convergence for a series centered on $z = z_0$ is the distance between $z_0$ and the nearest singularity.\end{quote}

Boas:

Let $C1$ and $C2$ be two circles with center at $z_0$.  Let $f(z)$ be analytic in the region between the circles.  Then $f(z)$ can be expanded in a Laurent series:
\[ f(z) = a_0 + a_1 (z - z_0) + a_2 (z - z_0)^2 + \dots + \]
\[ \ \ \ \ \ + \frac{b_1}{z - z_0} + \frac{b_2}{(z - z_0)^2} + \dots \]

The Taylor part of the series (called the analytic part) usually converges everywhere inside a disk of radius $R$, while the $b$ part, the principal part, usually converges everywhere outside a disk of radius $r$, so the combined series is convergent in the annulus, the area between $r$ and $R$.

Note:  if there are several isolated singularities , then there are several annular rings, each with a different Laurent series.  We will work an example of that in the next chapter.

\subsection*{Laurent's Theorem}
If $f(z)$ is analytic through a closed annulus $D$ centered at $z = z_0$, then at any point $z$ inside $D$ we can write:
\[ f(z) = a_0 + a_1(z-z_0) + a_2(z-z_0)^2 + \dots \]
\[ \hspace{27mm} + \ b_1(z-z_0)^{-1} + b_2(z-z_0)^{-2} + \dots \]

where the coefficients are given by
\[ a_n = \frac{1}{2 \pi i} \ \oint_C \frac{f(z)}{(z - z_0)^{n+1}} \ dz \]
\[ b_n = \frac{1}{2 \pi i} \ \oint_C \frac{f(z)}{(z - z_0)^{1-n}} \ dz \]
(no, they don't match the powers of $(z - z_0)$).

Any polynomial of $z$ is analytic, and quotients of analytic functions are also analytic.  

The end result will be that the integral $\int f(z) \ dz$ may be obtained by integrating the right-hand side, where all the terms except one will have an integral equal to zero.

Out of this entire series given above, only one term matters:
\[ b_1(z-z_0)^{-1}  \]

This is a consequence of Cauchy's Integral theorem.

\subsection*{derivation of Laurent series}
We follow

\url{https://www.youtube.com/watch?v=2GC26rJB2L0&list=PLvcbYUQ5t0UFmFXOLwC9BdZ5qghykd4gR&index=22&t=0s}

Fix some particular $z$ in the annulus $0 < r < |z-z_0| < R$.

Choose $r_1$ and $R_1$ just inside (or outside, respectively) the boundaries of the donut:
\[ 0 < r < r_1 < |z-z_0| < R_1 < R \]

Let $\gamma_1$ go along (let $w$ take on the values) $|w - z_0| = R_1$ counter-clockwise, with the interior on the left, and let $\gamma_2$ go along $|w - z_0| = r_1$, in the opposite direction to $\gamma_1$.  

Set up a keyhole contour.  Then $f$ is analytic in the domain which the line integral encloses, which is required to use Cauchy's formula:
\[ f(z) = \frac{1}{2 \pi i} \ [ \ \int_{\gamma_1} \frac{f(w)}{w-z} \ dw +  \int_{\gamma_2} \frac{f(w)}{w-z} \ dw \ ]  \]

The path that links the two circles cancels because we traverse it in opposite directions.

Now rewrite 
\[ \frac{1}{w - z} = \frac{1}{(w - z_0) - (z - z_0)} \]
\[ = \frac{1}{w - z_0} \ [ \ \frac{1}{1 - (z-z_0)/(w-z_0)} \ ] \]
This is a geometric series with initial term
\[ \frac{1}{w - z_0} \]

divided by (or multipied by the inverse of) one minus the common ratio
\[ \frac{z-z_0}{w - z_0} \]
Since 
\[ | \frac{z - z_0}{w - z_0} | < 1 \]
For the big circle $\gamma_1$ we have that this ratio is less than $1$ ($w$ runs along $R$), so the series converges absolutely.

The corresponding series is
\[ \sum_{k=0}^{\infty} \ \frac{(z - z_0)^k}{(w - z_0)^{k+1}} \]

For the small circle, $\gamma_2$:
the ratio $1/w-z$ is equal to
\[ -\frac{1}{z - z_0} \ [ \ \frac{1}{1 - (w-z_0)/(z-z_0)} \ ] \]

For the small circle $\gamma_2$ the flipped ratio is less than $1$ since $w$ runs along $r$, so again the series converges absolutely.  

The series is
\[ - \sum_0^{\infty} \ \frac{(w - z_0)^j}{(z - z_0)^{j+1}} \]

Rewriting 
\[ f(z) = \frac{1}{2 \pi i} \ [ \ \int_{\gamma_1} f(w) \ \sum_{k=0}^{\infty} \frac{(z - z_0)^k}{(w - z_0)^{k+1}} \ dw - \int_{\gamma_2} f(w) \ \sum_{j=0}^{\infty} \frac{(w - z_0)^j}{(z - z_0)^{j+1}} \ dw \ ] \]
The sums can come out from under the integral so
\[ f(z) = \frac{1}{2 \pi i} \ [ \ \sum_{k=0}^{\infty} (z - z_0)^k \int_{\gamma_1} \frac{f(w)}{(w - z_0)^{k+1}} \ dw  \]
\[ \ \ \ \ - \sum_{j=0}^{\infty} z - z_0)^{-j-1} \int_{\gamma_2} f(w) \ (w - z_0)^{j}  \ dw \ ]  \]

We can clean up the indices for $f_2$.  Instead of running from $j = 0 \rightarrow \infty$ with power $ - j -1$, let it be $k = - \infty \rightarrow -1$ with power $k$.

We now have $f_1(z) + f_2(z)$.  Now we can write the coefficients for both in the same way:
\[ a_k = \frac{1}{2 \pi i} \ \int_{\gamma} \frac{f(w)}{(w - z_0)^{k+1}} \ dw \]
and
\[ f_1(z) = \sum_0^{\infty} a_k(z-z_0)^k \]
$f_1$ is convergent when $|z-z_0| < R$.

\[ f_2(z) = \sum_{-\infty}^{-1} a_k(z-z_0)^k \]
$f_2$ is convergent when $|z-z_0| > r$.

A Laurent series is the combination.
\[ \sum_{-\infty}^{\infty1} a_k(z-z_0)^k \]

It is convergent when both criteria are met.

The principal part of the Laurent series consists of the terms with negative exponents.  We can also write that part as
\[ \sum_{n=1}^{\infty} a_{-n} (z - z_0)^{-n} \]

Although it seems like we've made things \emph{really} complicated, they aren't, because when such a series is integrated, the only non-zero term is the $k = -1$ term.

Let's sidestep the problem of determining the coefficients using the formulas given above.

Instead, just say that we seek a series expansion using negative powers of $z$, and try to make sure that it will be valid in the region $|z| > 1$.

\end{document}