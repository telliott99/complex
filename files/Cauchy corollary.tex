\documentclass[11pt, oneside]{article} 
\usepackage{geometry}
\geometry{letterpaper} 
\usepackage{graphicx}
	
\usepackage{amssymb}
\usepackage{amsmath}
\usepackage{parskip}
\usepackage{color}
\usepackage{hyperref}

\graphicspath{{/Users/telliott/Github/figures/}}
% \begin{center} \includegraphics [scale=0.4] {gauss3.png} \end{center}

\title{Cauchy corollary}
\date{}

\begin{document}
\maketitle
\Large

%[my-super-duper-separator]

In this section we follow Beck, so I've used their notation, which is slightly different.  In particular, $w$ rather than $z_0$ is the fixed point inside the region.

The Cauchy Integral formula is then:
\[ f(w) = \frac{1}{2 \pi i} \oint_C \frac{f(z)}{(z - w)} \ dz \]

If $f$ is differentiable for all points in some open disk centered at $w$ then $f$ is holomorphic at $w$.  For a holomorphic function $f$, a specific extension of the Cauchy formula is

\[ f'(w) = \frac{1}{2 \pi i} \oint_C \frac{f(z)}{(z - w)^2} \ dz \]

One way this can be obtained is by just differentiating the original formula under the integral sign on the right-hand side. 

\subsection*{Cauchy corollary}

The Cauchy formula can also be written:
\[ \oint_C \frac{f(z)}{z - w} \ dz = 2 \pi i f(w) \]

Note that while we will fix $w$ and then vary $z$ along $\gamma$, before that, this can be viewed as a function of both $w$ and $z$, so we can take the partial with respect to $w$ of both sides:

\[ \frac{\partial}{\partial w} ( \frac{f(z)}{z - w} ) = \frac{f(z)}{(z - a)^2} \]
so then
\[ \oint_C \frac{f(z)}{(z - w)^2} \ dz = 2 \pi i f'(w) \]
Thus
\[ f'(w) = \frac{1}{2 \pi i} \ \oint_C \frac{f(z)}{(z - w)^2} \ dz \]

More generally
\[ f^n(w) = \frac{n!}{2 \pi i} \ \oint_C \frac{f(z)}{(z - w)^{n+1}} \ dz \]
So
\[ \frac{2 \pi i}{n!} f^n(w) = \oint_C \frac{f(z)}{(z - w)^{n+1}} \ dz \]

\subsection*{more carefully} 

A more formal proof is the following (from Beck).

\[ f'(w) = \frac{f(w + \Delta w) - f(w)}{\Delta w} \]
so
\[ = \frac{1}{\Delta w} \ [ \ \frac{1}{2 \pi i} \ \int_{\gamma} \frac{f(z)}{z - (w + \Delta w)} \ dz - \frac{1}{2 \pi i} \int_{\gamma} \frac{f(z)}{(z - w)} \ dz \ ] \]
\[ =  \frac{1}{2 \pi i} \ \int_{\gamma} \frac{f(z)}{(z - w - \Delta w)(z-w)} \ dz  \]

In putting the two fractions over a common denominator we get a factor of $\Delta w$ on top which cancels the leading one.

It is now possible to show that the value of this integral approaches what we seek as $\Delta w \rightarrow 0$.  

We will next show that the difference of integrals goes to zero as $\Delta w \rightarrow 0$.

That difference is
\[ = \frac{1}{2 \pi i} \int_{\gamma} (\frac{f(z)}{(z - w - \Delta w)(z - w)} - \frac{f(z)}{(z - w)^2} \ dz \]
\[ = \frac{\Delta w}{2 \pi i} \int_\gamma \frac{f(z)}{(z - w - \Delta w)(z - w)^2} \ dz \]

Very similar to what we just did.

As $\Delta w \rightarrow 0$, the $\Delta w$ on top will make the whole thing go to zero, \emph{provided that the integral remains bounded}.

Let $M$ be the maximum value of the function over the curve.  They write:
\[ M := \text{max}_{z \in \gamma} |f(z)| \]

Choose $\delta > 0$ such that 
\[ |z - w| \ge \delta \]
for all $z$ on $\gamma$.  

Then the \hyperref[sec:rev_tri_inequality]{\textbf{reverse triangle inequality}} says that
\[ |(z - w - \Delta w)(z - w)^2 | \ge ( |z - w| - |\Delta w|)|z - w|^2 \]
\[ \ge (\delta - |\Delta w|) \delta^2 \]
so
\[ \frac{f(z)}{|(z - w - \Delta w)(z - w)^2} | \le \frac{|f(z)|}{( |z - w| - |\Delta w|)|z - w|^2} \]
\[ \le \frac{M}{(\delta - |\Delta w|) \delta^2 } \]
which certainly stays bounded as $\Delta w \rightarrow 0$.

This proves the Cauchy Integral formula for $f'$.

$\square$

The formula for $f''$ is
\[ f''(w) = \frac{1}{\pi i} \int_C \frac{f(z)}{(z - w)^3} \ dz \]
Notice the extra factor of $2$

The general rule is:
\[ f^n(z) = \frac{n!}{2 \pi i} \int_C \frac{f(w)}{(w - z)^{n+1}} \ dw \]
which can be proved by induction.

\subsection*{example}

\label{sec:ex8C}

\[ \int \frac{1}{z^2(z-1)}  \ dz \]
If the region includes both of the singularities $z = 0$ and $z = 1$, we can split that path into two parts as shown in the figure:
\begin{center} \includegraphics [scale=0.5] {Beck_5_1.png} \end{center}

Rewrite the integral as
\[ \int_{\gamma_1} \frac{1/(z-1)}{z^2} \ dz + \int_{\gamma_2} \frac{1/z^2}{z - 1} \ dz \]

By the corollary
\[ 2 \pi i f'(w) =  \int_{\gamma_1} \frac{f(z)}{(z - w)^2} \ dz \]
So 
\[ f(z) = \frac{1}{z - 1} \ dz \]
\[ f'(z) = -\frac{1}{(z - 1)^2} \ dz \]
\[ f'(z = 0) = -1 \]
The integral is:
\[ = 2 \pi i f'(w) = -2 \pi i  \]

For the second path
\[ 2 \pi i f(w) =  \int_{\gamma_2} \frac{f(z)}{z - w} \ dz \]
with $w = 1$, and 
\[ f(z) = \frac{1}{z^2} \]
so
\[ = 2 \pi i \ [ \  \frac{1}{z^2} \bigg |_1 \ ] = 2 \pi i \]
The total is zero.


\end{document}