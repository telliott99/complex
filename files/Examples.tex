\documentclass[11pt, oneside]{article} 
\usepackage{geometry}
\geometry{letterpaper} 
\usepackage{graphicx}
	
\usepackage{amssymb}
\usepackage{amsmath}
\usepackage{parskip}
\usepackage{color}
\usepackage{hyperref}

\graphicspath{{/Users/telliott/Github/figures/}}

\title{Examples}
\date{}

\begin{document}
\maketitle
\Large

%[my-super-duper-separator]

\subsection*{example 1}
\[ I = \oint \frac{1}{1 - z} \ dz \]

\hyperref[sec:series_cheatsheet]{\textbf{Laurent}}

Change of variables.  Let $w = 1 - z$, $dw = - dz$
\[ I = \oint \frac{1}{w} \ (- dw) = - (2 \pi i) \]

\subsection*{example 2}
\[ I = \oint \frac{1}{1 + z} \ dz \]

\hyperref[sec:series_cheatsheet]{\textbf{Laurent}}.

Change of variables.  As above, except there's no minus sign.  $I =  2 \pi i$.

\subsection*{example}
\[ I = \oint \frac{1}{a + z} \ dz \]

\hyperref[sec:a_plus_z]{\textbf{Laurent}}.

Factor and change of variables 
\[ I = \frac{1}{a} \oint \frac{1}{1 + z/a} \ dz \]
Let $w = z/a$, $a dw = dz$.
\[ = \frac{1}{a} \oint \frac{1}{1 + w} \ a dw \]

$ I = 2 \pi i$ (example 2).

\subsection*{example 3}
\[ \oint \frac{1}{z(z + 2)} \ dz \]

\hyperref[sec:ex3L]{\textbf{Laurent}}.

Partial fractions.
\[ \oint \frac{1}{2} \ [ \ \frac{1}{z} - \frac{1}{z + 2} \ ] dz \]

The curve $C[1,3]$ includes the singularity at $z = 0$, but $z = -2$ is on the boundary, not inside the region.  The curve $C[1,1]$ includes neither point.

So for the second curve the integral is zero and for the first one only the $\int 1/z \ dz$ matters.  It's an old friend, with value $2 \pi i$, the value of the integral is thus $\pi i$.

Residues.

At $z_0 = 0$ we have $1/(z+2)$ evaluated at $z_0 = 0$ which is $1/2$.  At $z_0 = -2$ we have $1/z$ evaluated at $z_0 = -2$ which is $-1/2$.  Either one contributes its value times $2 \pi i$ if included in the contour.

\subsection*{example 4}
\[ \oint \frac{1}{z^2 - 1} \ dz \]

Residues.  Factor as
\[ \frac{1}{(z + 1)(z - 1)} \]

We have poles at $z = \pm 1$.  The first term is $z_0 = -1$ so evaluate $1/(z - 1)$ at $z_0$ and obtain $R(-1) = -1/2$.

For the other one $z_0 = 1$ and that is $z - z_0$ as written, so evaluate $1/z+1$ at $z_0$ and obtain $R(1) = 1/2$.

\subsection*{example 5}
\[ \oint \frac{1}{z(z - 2)^2} \ dz \]

\hyperref[sec:ex5L]{\textbf{Laurent}}, \hyperref[sec:ex5PF]{\textbf{Partial fractions}}.

Residues.  There is a pole of first order at $z_0=0$ and one of second order at $z_0=2$.  

At the first, multiply by $(z - 0)$ and evaluate the limit of what's left:
\[ \text{Res }(0) = \frac{1}{(z-2)^2} \bigg |_{z_0 = 0} = \frac{1}{4} \]

For the other, multiply by $(z-2)^2$ and compute the $N-1$ (first) derivative of what's left
\[ [ \ \frac{1}{z} \ ]' \ =  - \frac{1}{z^2} \]
\[ \text{Res } (2) = \lim_{z \rightarrow 2} -\frac{1}{z^2} = -\frac{1}{4} \]
Don't forget to divide by $(N-1)!$, which is just $1$ in this case.  The total is zero.

Change of variables to allow series.
\[ \frac{1}{z(z-2)^2} \]
Let $w = z - 2$, $dw = dz$, $z = w + 2$
\[ \oint \frac{1}{w^2(w+2)} dw \]

This allows the possibility of a series solution, but we must ask where is the contour?

Suppose the contour includes both of the original poles.  Because of that factor of $1/w^2$, we need the analytic $A$ series for $1/(w+2)$, from \hyperref[sec:a_plus_z]{\textbf{here}}.

\[ \frac{1}{2} \cdot \ [ \ 1 - \frac{w}{2} + \frac{w^2}{4} \dots \ ]    \]
With the factor, the relevant term is $b = -1/4$, which matches what we had above.

We would have a different series and a different answer for a smaller contour only enclosing one pole.

\subsection*{example 6}
\[ \oint \frac{1}{z(z - 2)^4} \ dz \]

Laurent series.  There are singularities at $z = 0$ and $z = 2$.

Use the $1/z$ part to get a geometric series:
\[ \frac{1}{z(z-2)^4} = \frac{1}{(z-2)^4} \cdot \frac{1}{(2 + z - 2)} \]
\[ = \frac{1}{(z-2)^4} \cdot \frac{1}{2} \cdot \frac{1}{(1 - (-(z-2))/2 )} \]

The third term gives the geometric series with common ratio $(z-2)/2$.  Those $z-2$ terms will cancel the leading factor.  The only term that matters is the cube, which gives:
\[  \frac{1}{(z-2)} \cdot \frac{1}{2} \cdot \frac{1}{(-2)^3} \]

We have that $b_1 = -1/16$ so $I = -\pi i/8$.

Residues.  We have a pole of first order at $z=0$ and one of fourth order at $z=2$.  At the first
\[ \text{Res } [f(z),z=0] = \lim_{z \rightarrow 0}   \frac{1}{(z-2)^4} \]
\[ =  \lim_{z \rightarrow 0} \ \frac{1}{(z-2)^4}  = \frac{1}{(-2)^4} =  \frac{1}{16} \]

For the other pole recall that
\[ \frac{2 \pi i}{n!} f^n(a) = \oint_C \frac{f(z)}{(z-a)^{n+1}} \ dz \]

Remove the factor of $1/(z-2)^4$ leaving $f(z) = 1/z$ and then compute the $N-1$ (third) derivative of what's left

\[ \text{Res } [f(z),z=2] = \frac{1}{n!} \ \lim_{z \rightarrow 2} \ [ \ \frac{1}{z} \ ]''' \ \]
\[ f(z) = z^{-1} \]
\[ f'(z) = - z^{-2} \]
\[ f''(z) = 2 z^{-3} \]
\[ f'''(z) = -6 z^{-4} \]
\[  \lim_{z \rightarrow 2} \ [ \ \frac{1}{z} \ ]''' = - \frac{6}{16} \]

Don't forget to divide by $(N-1)!$, which is $3! = 6$ in this case.  That leaves
\[ \text{Res } [f(z),z=2] = - \frac{1}{16} \]

The total of the residues is just zero.  

This problem is from Brown and Churchill, which they work by doing Laurent series.  They get a different answer, namely $-\pi i/8$.  

The reason is that they integrate over the contour $0 < | z - 2 | < 2$, that is, $C[2,2]$, which includes the second pole but not the first.  

Multiplying by $2 \pi i$ gives their result.

Going back to example 6, what I did was
\[ f(z) = \frac{1}{z(z-2)^4} \]

To expand this, write:
\[ -\frac{1}{16z} \cdot \frac{1}{(1 - z/2)^4} \]
which gives a geometric series in $z/2$.  
\[ = -\frac{1}{16z} (1 + \frac{z}{2} + (\frac{z}{2})^2 \dots)^4 \]

We want the term with $1/z$, which is just the first one of the power series to the fourth power:  $1$.  Thus that term is
\[   -\frac{1}{16z}  \]
The residue is $-1/16$.

We do need to think about the disk of convergence.  We are on $C[2,2]$.  The series has $r = z/2$ which converges if $|z - z_0/2| < 1$, so $|z - z_0| < 2$, which is fine.

Note:  the example is number 3 in the section on Residues from the 8th edition, which I found online, but it is not available in my paper copy (6th ed.).

Change of variables.

\[ \oint \frac{1}{z(z - 2)^2} \ dz \]
Let $w = z - 2$, $dw = dz$, $z = w + 2$.
\[ \oint \frac{1}{w^4(w + 2)} \ dz \]

Expand
\[ \frac{1}{2 + w} = \frac{1}{2} \ \frac{1}{(1 + w/2)} \]
Because of the factor of $1/w^4$ we need the third term of the analytic $A$ series:
\[ \frac{1}{2} \ (\dots + (-\frac{w}{2})^3 + \dots \]
The cofactor is $b_1 = -1/16$.

\subsection*{example 7}
\[ \oint \frac{1}{z^4 - 1} \ dz \]

Residues.  We can factor the denominator as
\[ z^4 - 1 = (z^2 - 1)(z^2 + 1) \]
\[ = (z+1)(z-1)(z+i)(z-i) \]

There are four poles, and each will have a residue.
\[ \text{Res}(1) =  \frac{1}{(z+1)(z+i)(z-i)} \bigg |_{z_0 = 1} \]
\[ = \frac{1}{2(1+1)} = \frac{1}{4} \]

\[ \text{Res}(i) = \frac{1}{(z+1)(z-1)(z+i)} \bigg |_{z_0 = i}\]
\[ = \frac{1}{(-2)(2i)} = -\frac{1}{4i} = \frac{i}{4}  \]

\subsection*{example 8}
\[ \oint \frac{1}{z^2(z - 1)} \ dz \]

\hyperref[sec:ex8C]{\textbf{Cauchy}} or \hyperref[sec:ex8PF]{\textbf{Partial fractions}}.

\subsection*{example 9}

\subsection*{example 10}
\[ \oint \frac{1}{z^2 + 1} \ dz \]

\hyperref[sec:ex10PF]{\textbf{Partial fractions}}.

Residues. This can be factored
\[ f(z) = \frac{1}{(z + i)(z - i)} \]
So there are two simple poles, at $z = \pm \ i$.

The formula is:
\[ b_1 = \lim_{z \rightarrow z_0} (z-z_0) \ f(z)  \]

Evaluate the formula.  Our path includes $i$ but not $-i$.  We have for $z_0 = i$:
\[ b_1 = \lim_{z \rightarrow i} (z-i) \  \frac{1}{(z+i)(z-i)} \]
\[ = \lim_{z \rightarrow i}  \  \frac{1}{z+i} = \frac{1}{2i} \]

Multiplied by $2 \pi i$:
\[ I = \pi \]

If the unit circle had been centered at $-i$, rewrite the function as
\[ f(z) = \frac{1/z-i}{z+i} \]
The value of the function is

\[ \frac{1}{z-i} \bigg |_{-i} = -\frac{1}{2i} \]
and the integral is then $- \pi$.

A contour that includes both singularities integrates to zero.

\subsection*{example 12}
\[ \oint \frac{5z - 2}{z(z - 1)} \ dz \]

There are two simple poles at $z_0 = 0$ and $z_0 = 1$ and the residues are
\[ \text{Res }(0) = \lim_{z \rightarrow 0} (z - 0) \ \frac{5z-2}{z(z-1)} \]
\[ = \lim_{z \rightarrow 0} \ \frac{5z-2}{(z-1)} \]
\[ = \frac{5 \cdot 0 - 2}{0 - 1} = 2 \]

\[ \text{Res }(1) = \lim_{z \rightarrow 1} (z - 1) \ \frac{5z-2}{z(z-1)} \]
\[ = \lim_{z \rightarrow 1} \ \frac{5z-2}{z} \]
\[ = \frac{5 \cdot 1 - 2}{1} = 3 \]
Hence the total of all the residues is $5$ and $I = 10 \pi i$.

\subsection*{example 13}
\[ \oint \frac{z}{(2z + 1)(5 - z)} \]

\[ f(z) = \frac{z}{(2z + 1)(5 - z)} \]

The poles are at $z_0 = -1/2$ and $z_0 = 5$.

First multiply the function by $z - z_0$  That gives
\[  R(-1/2) = \frac{z}{2 (5 - z)} \bigg |_{z = -1/2}  = \frac{-1/2}{2(11/2)} = -\frac{1}{22} \]

And
\[ R(5) = -\frac{z}{2z + 1} \bigg |_5 = - \frac{5}{11} \]

For
\[ f(z) = \frac{\cos z}{z} \]
The pole is at $z_0 = 0$ so multiply by $z$:
\[ R(0) = \cos z \bigg |_0 = 1 \]

\subsection*{example 14}
\[ \oint \frac{z^2}{4 - z^2} \ dz \]

\hyperref[sec:ex14PF]{\textbf{Partial fractions}}.

\subsection*{example 15}
\[ \oint \frac{z^2}{z^2 + 2z + 2} \ dz \]

\hyperref[sec:ex15PF]{\textbf{Partial fractions}}.

Residues.  Use the quadratic equation to factor.  The zeroes are at $z_0 = -1 \pm i$.  
\[ \frac{1}{z^2 + 2z + 2} = \frac{1}{(z - (-1 + i))(z - (-1 - i) )}  \]

At the first
\[ R(-1 + i) = \frac{z^2}{z - (-1 - i)} \bigg |_{z_0 = -1 + i} = \frac{-2i}{2i} = - 1 \]

\subsection*{example 16}
\[ \oint \frac{12}{z(2 - z)(1 + z)} \ dz \]

\hyperref[sec:ex16L]{\textbf{Laurent}}.

\subsection*{example 17 (Orloff 7.2.1)}
\[ \int \frac{z}{z^2 + 1} \ dz \]

Factoring the denominator we have that $z^2 + 1 = (z + i)(z - i)$.  So there are singularities at $z = \pm i$.  We are asked to expand the series around $z =i$.  There are actually two different regions with two different solutions.  We will compute the inner punctured disk with $|z - i| < 2$.  This extends as far as the singularity at $z = i$.

Using partial fractions write:
\[ z \frac{1}{(z + i)(z - i)} = \frac{z}{2i} \ [ \frac{1}{z - i} - \frac{1}{z + i}  \  ] \]
Keep that $z/2i$ in your back pocket for now.

We need the $A$ series around the first term and the $P$ series around the second.  These are:
\[ \frac{1}{z - i} = - \frac{1}{i} \cdot \frac{1}{1 - z/i} = - \frac{1}{i} \cdot \ [ \ 1 + (\frac{z}{i}) + (\frac{z}{i})^2 + (\frac{z}{i})^3 \dots \ ] \]

The other one is
\[ - \frac{1}{z} \cdot \frac{1}{1 + i/z} = - \frac{1}{z} \cdot \ [ \ 1 + (-\frac{i}{z}) + (-\frac{i}{z})^2 + (-\frac{i}{z})^3 \dots \]

Since we have $z$ (in our pocket), the only term that matters is
\[ -\frac{1}{z} \cdot (- \frac{i}{z}) = \frac{i}{z^2} \]
from the second one.  That becomes, digging into our pockets:
\[ \frac{i}{z^2} \cdot \frac{z}{2i} = \frac{1}{2z} \]

So a contour integral around the point $z = i$ and inside the radius $|z - i| < 2$ will have a value of $\pi i$.  We can check that:
\[ \frac{1}{2i}  \oint \frac{z}{z - i} \ dz = (2 \pi i) \ \frac{z}{2 i} \bigg |_{z = i} = \pi i \]

Also, we can check this by change of variables.  The singularity in the region of the contour integral is the first term:
\[ \oint \frac{1}{2i} \cdot \frac{z}{z - i} \ dz \]
Let $w = z - i$, $dw = dz$, then
\[ = \frac{1}{2i} \oint \frac{w + i}{w} \ dw \]

The integral of the first part is $\int dw$ which is zero.  The second part is
\[ \frac{1}{2i} \oint \frac{i}{w} \ dw = \frac{1}{2} \cdot (2 \pi i) = \pi i \]

ex 18:  \hyperref[sec:ex18L]{\textbf{Laurent}}
\[ \oint \frac{z}{(z - 1)(z - 3)} \ dz \]

\subsection*{example 19 (Brown and Churchill)}
\[ \oint e^{1/z^2} \ dz \]

Laurent series.

We use the standard series for $e^z$
\[ e^z = 1 + z + \frac{z^2}{2!} + \frac{z^3}{3!} + \dots \]
substituting $1/z^2$
\[ 1 + \frac{1}{z^2} + \frac{1}{2! \ z^4} + \frac{1}{3! \ z^6} + \dots \]

Since there's no $z$ with a power $n=-1$, the value of the integral is zero.

\subsection*{example 20}
\[ f(z) = \frac{e^z}{z^2} \]

There is obviously a double pole at $z = 0$.  (1) Multiply by $z^2$.  (2) Take the derivative, obtaining $e^z$.  Evaluate at $z = 0$, obtaining $1$ for the residue.

Another way is to write a Laurent series:
\[ = \frac{1}{z^2} (1 + z + \frac{z^2}{2!} + \frac{z^3}{3!} \dots ) \]
The only term with a non-zero integral is\
\[ \frac{1}{z} \]
so the cofactor $b_1$ is just $1$, which is also the residue, and the value of the integral is $2 \pi i$.

\subsection*{example 21}

\[ f(z) = \frac{1 + e^z}{z^2} + \frac{2}{z} \]
We can break this up into its two component parts.  For the first term, the pole is of order $m = 2$ at $z_0 = 0$.  We remove the $z^2$ term and take the derivative
\[ (1 + e^z)' = e^z \]

The factorial term is just $1$, leaving $e^z$ which is evaluated at the pole giving a residue
\[ \text{Res }(0) = e^0 = 1 \]

The other term is just $2$ times the standard
\[ \oint \frac{1}{z} \ dz = 2 \pi i \]
Here $I = 4 \pi i$ and the residue is $2$.  Alternatively just use
\[ I = 2 \pi i f(z_0) = 4 \pi i \]
where $f = 2$.

The total of the residues is $3$ and the value of the integral is $6 \pi i$.

Still another approach
\[ \frac{1 + e^z}{z^2} \]
has a pole at $z_0 = 0$, so we \emph{can} use the series for $e^z$ at $0$:
\[ = \frac{1}{z^2} \cdot 1 + 1 + z + \frac{z^2}{2!} + \frac{z^3}{3!} \dots \]

The relevant term is 
\[ = \frac{1}{z^2} \cdot  z = \frac{1}{z} \]
So the residue is $1$.

\subsection*{example 22}  
\[ \oint \frac{e^z}{z^3} \ dz \]

Series.  The relevant term is 
\[ = \frac{1}{z^3} \cdot  \frac{z^2}{2!} = \frac{1}{2z} \]
So the residue is $1/2$.

\subsection*{example 23}
\[ \oint \frac{e^z}{z - 1} \ dz \]

Laurent series.

\[ I = \oint \frac{e^z}{z - 1} \ dz \]

There's a trick to writing the Laurent series:
\[ \frac{e^z}{z - 1} = \frac{e}{z - 1} \cdot e^{z - 1} \]
So now the expansion of $e^{z -1}$ gives:
\[ = \frac{e}{z - 1} \ [ \ 1 + (z - 1) + \frac{(z-1)^2}{2!} + \dots \ ]  \]
\[ = e \ [ \ \frac{1}{z - 1} + 1 + \frac{(z-1)}{2!} + \dots \ ]  \]
The coefficient of $1/(z-1)$ is $e$, so the value of the integral is
\[ I = 2 \pi i \cdot e \]

Extension:

For $z - n$ in the denominator, in a similar way:
\[ \frac{e^z}{z - n} = \frac{e^n e^{z -n}}{z - n} \]
\[ = \frac{e^n}{z - n} \ ( 1 + (z - n) + \frac{(z - n)^2}{2!} + \dots ) \]
The residue is $e^n$.

There's another simple dodge, and that is change of variables.  Let
\[ w = z - 1\]
\[ dw = dz \]
\[ e^z = e^{w + 1} = e^w \cdot e \]
Then f is
\[ e \cdot \frac{e^w}{w} \]
so the series is 
\[ \frac{1}{w} \ (1 + w + w^2 \dots ) \]
The cofactor of $w^{-1}$ is $e$ and the result is the same as before.

Of course, for $z - n$ we get a power of $e^n$ multiplying the series.

Also, you might wonder about 
\[ \oint \frac{e^z}{1 - z} dz \]
With the substitution, $dw = - dz$, so that's where we pick up the minus sign that we need.  Or just manipulate before the substitution:
\[ \frac{1}{1 - z} = - \frac{1}{z - 1} \]

Residues.

The pole is at $z = 1$.  At that point we have
\[ e^z \bigg |_1 = e \]

\subsection*{example 24}
\[ \int \frac{e^z}{z(z - 1)^2} \ dz \]

\hyperref[sec:ex24R]{\textbf{Residues}}.

\subsection*{example 25}

Residues.  The denominator can be factored
\[ z^2 - 2z - 3 = (z + 1)(z - 3) \]

If the disk is $|z| \le 2$ then it includes only $z_0 = -1$ and the formula is
\[ b_1 = \lim_{z \rightarrow z_0} (z-z_0) \ f(z)  \]
so
\[ b_1 = \lim_{z \rightarrow -1} (z+1) \ \frac{e^z}{(z + 1)(z - 3)}  \]
\[ = \lim_{z \rightarrow -1} \ \frac{e^z}{(z - 3)}  \]
\[ = \frac{e^{-1}}{-1 - 3} = - \frac{1}{4 e} \]
and
\[ I = 2 \pi i \ b_1 = 2 \pi i \ (- \frac{1}{4 e}) \]
\[ = - \frac{\pi i}{2 e} \]

\subsection*{example 26}
\[ \oint_C \frac{e^z}{z^3 - z^2 - 5z - 3} \ dz \]

The factorization was given in the original problem, but suppose we don't have it.  Guess:
\[ (z ...)(z^2 ... ) \]
The next term is $-z^2$ so maybe 
\[ (z -3)(z^2 + 2z ... ) \]
Then we'll need $-3$ at the end:
\[ (z - 3)(z^2 + 2z + 1) \]
and we get $-5z$, so that works!

Nicely, the quadratic also factors:
\[ \oint_C \frac{e^z}{(z+1)^2 (z-3)} \]

A double pole at $z = -1$ and a single one at $z = 3$.

Recall the general approach for a double pole, construct
\[ g(z) = (z - z_0)^2 \ f(z) \]
Here
\[ = \frac{e^z}{(z-3)} \]

Then compute the first derivative:
\[ \frac{e^z \cdot (z-3) - (e^z \cdot 1)}{(z - 3)^2} \]
\[ = \frac{(z-4)e^z}{(z-3)^2} \]

Evaluate the limit of that as $z \rightarrow z_0$.
\[ \lim_{z \rightarrow -1} \ \frac{(z-4)e^z}{(z-3)^2} = -\frac{5}{16} \ \frac{1}{e} \]

Remember the extra factor of $1/(n-1)!$ to get the residue, and $2 \pi i$ to get the value of the integral.
\[ I = -\frac{5 \pi i}{8 e} \]

\subsection*{example 27}
\[ \oint \frac{z e^z}{z^2 - 1} \ dz \]

\hyperref[sec:ex27R]{\textbf{Residues}}.

\subsection*{example 28}

\[ f(z) = \frac{e^z}{z(z-1)^2} \]
The denominator is one we saw above, but now there is an extra factor of $e^z$.

We have a pole of first order at $z=0$ and one of second order at $z=1$.  At the first
\[ \text{Res } [f(z),z=0] = \lim_{z \rightarrow 0} \frac{e^z}{(z-1)^2} = 1 \]

For the other one, remove the factor of $1/(z-1)^2$ and compute the $N-1$ (first) derivative of what's left
\[ \text{Res } [f(z),z=1] = \lim_{z \rightarrow 1} \ [ \ \frac{e^z}{z} \ ]' \ \]
\[ = \frac{e^z z - e^z}{z^2} = \frac{e^z(z - 1)}{z^2} \ \bigg |_1 =  0 \]
Hence
\[ \oint f(z) \ dz = 2 \pi i \ [ \ \sum  \text{Res } \ ] \ = 2 \pi i \]

\subsection*{example 29 (Boas)}
\[ \oint \cot z \ dz \]

Find $R(0)$ for $f(z) = \cot z$.
\[ R = \lim_{z \rightarrow z_0} (z - z_0) \frac{\cos z}{\sin z} \]
\[ = \lim_{z \rightarrow 0} \frac{z}{\sin z} \ \cos z = \cos 0 = 1 \]

\subsection*{example 30}
\[ \oint \frac{\sin \pi z}{z^2 - 1} \ dz \]

\hyperref[sec:ex30R]{\textbf{Residues}}  

\subsection*{example 31}
\[ \oint \frac{\sin z}{1 - z^4} \ dz \]

\hyperref[sec:ex31R]{\textbf{Residues}}.

\end{document}